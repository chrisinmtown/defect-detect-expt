
Beachten Sie bitte auch das Erg"anzungsblatt "`Funktionales Testen mit 
Programm $x$"', was im Ihnen vorliegenden Material enthalten ist.

\begin{enumerate}

\item Lesen Sie sich den Fragebogen "`E22: Funktionales Testen"'
durch. Tragen Sie Ihre Kennung ein und f"ullen Sie die Rubrik
"`Vorbetrachtungen"' aus. 

\subsection*{Test generieren}

\item Tragen Sie die Uhrzeit zu der Sie mit dem Experiment beginnen 
unter Frage 4 ein.

\item Lesen Sie sich die Spezifikation durch. Leiten Sie daraus
die \"Aquivalenzklassen ab.
Tragen Sie deren Anzahl unter Frage 5 des Fragebogens ein.
Bitte beantworten Sie zugleich Frage 6.

\item Generieren Sie Testf\"alle, indem Sie aus den \"Aquivalenzklassen 
die Grenzf\"alle w"ahlen.
%\item W\"ahlen Sie die Grenzf\"alle aus den \"Aquivalenzklassen aus
%und generieren Sie die Testf\"alle. 
Verzichten Sie dabei auf die oberen Grenzen f\"ur Anzahl der
Parameter o.\"a.~(siehe auch Erg\"anzungsblatt).
Wenn Sie damit fertig sind, tragen Sie die Anzahl der Testf\"alle
unter Frage 7 und Ihren Zeitbedarf unter Frage 8 ein.

\item Loggen Sie sich bei einem Rechner bzw.~Xstation ein.  
Tragen Sie den Rechnernamen und Zeitpunkt unter Frage 9 ein.  
Holen Sie sich das notwendige Dateipaket (siehe Erg\"anzungsblatt).
\"Uberzeugen Sie sich davon, da{\ss} alle Dateien bzw.~Verzeichnisse
vorhanden sind! 

\item Tippen Sie Ihre vorbereitete Testf\"alle in Dateien im
Verzeichnis "`test-dir"' ein. Um Ihnen die Arbeit zu erleichtern, ist die
Komponente mit einer Kommandozeilenschnittstelle versehen worden. Diese
Schnittstelle wird von einem automatischen Tester angesteuert. Der Tester
lie"st {\em Parameterdateien\/} und ruft die Komponente mit den darin
spezifizierten Parametern auf. Die Resultate werden in eine Datei geschrieben.
Auf diese Weise wird ein kompletter Test der Komponente durchf"uhrt. 

Sie m"ussen die Parameter- und ggf.~{\em Eingabedateien} anlegen. Ein 
Testfall $=$ eine Parameterdatei. Die 
Ausdr"ucke in einer Parameterdatei spezifizieren einen Aufruf der
Komponente f"ur einen Lauf des automatischen Testers. Parameterdateien werden
mit ``.test'' als Dateikennung benannt. Beispiel: Soll die Komponente 'komp'
mit {\tt komp -i in\_file1} aufgerufen werden, dann geh"ort in
die Parameterdatei der Ausdruck {\tt -i in\_file1}.

{\em Eingabedateien\/} werden f"ur Komponenten angelegt, die aus
Dateien lesen. Eingabedateien sollten sinnvoll benannt werden,
z.B. ``leer'', sie d"urfen {\bf nicht} die Dateikennung ``.test'' aufweisen.

\"Uberzeugen Sie sich, da{\ss} in den Dateien auch wirklich das
enthalten ist, was Sie wirklich wollen!

\subsection*{Test durchf"uhren}

\item Wenden Sie die Testf\"alle auf die Komponente an indem Sie das
Kommando "`run-suite"' eingeben. 
Falls Sie sich vertippt haben und einige Testf\"alle deswegen
nicht das getestet haben, was Sie getest haben wollten, korrigieren
Sie jetzt die Fehler, tippen sie "`make clean"' und lassen sie
dann "`run-suite"' wieder laufen.   
Aber bitte generieren Sie keine weiteren Testdaten! 

\item Haben Sie w\"ahrend des Eintippens bzw.~Durchf\"uhrens der
Testf\"alle irgendwelche Pausen gemacht, so vermerken Sie nur den aktiven
Anteil an der gesamten Zeit (d.h. der Anteil ohne Pausen, etc.) unter
Punkt 10. 

\item Die Resultate stehen sowohl in einzelnen Dateien im Verzeichnis 
"`test-dir"' als auch in einer zusammenh\"angende Datei
"`test-results.summary"', worin die einzelnen Resultate getrennt sind.
Drucken Sie die Resultate zweimal auf dem Zeilendrucker aus. 

\item Loggen Sie sich aus und holen Sie Ihre Ausgabe vom Zeilendrucker.

\subsection*{Fehlverhalten diagnostizieren}

\item Tragen Sie die Startzeit unter Frage 11 ein.
Schauen Sie sich die Testresultate genau an.
Finden Sie etwaige Fehlverhalten, indem sie die Ergebnisse
lt.~Spezifikation mit den Ausgaben Ihrer Testf\"alle vergleichen. 
Markieren Sie die entdeckten Fehlverhalten in der Ausgabe 
durch Umkreisen, o."a. Tragen Sie die Fehlverhalten in das Formblatt f\"ur
Fehlverhalten ein. Nummieren Sie die Fehlverhalten von 1 bis $n$ in der
Spalte "`Fv-Nr."' (Fehlverhaltensnummer) des Formblatts und entsprechend
die Umkreisungen im Ausdruck.

\item Wenn Sie Ihrer Meinung nach alle Fehlverhalten gefunden haben,
so tragen Sie bitte Ihren Zeitbedarf daf\"ur unter Frage 12 ein.


\item Notieren Sie auf einem Ausdruck Ihrer Resultate Ihre Praktikumskennung,
mit welcher Testmethode Sie gearbeitet haben und in welchem Verzeichnis
Ihre Testf"alle zu finden sind. Geben Sie diesen Ausdruck 
beim Betreuer ab und nehmen Sie den Quellkode entgegen. Bitte l"oschen Sie
nicht Ihre Testf"alle !

\subsection*{Fehler lokalisieren}

\item Bitte tragen Sie die Uhrzeit unter Frage 13 ein.
Versuchen Sie die Fehler, die zu den von Ihnen erkannten
Fehlverhalten gef\"uhrt haben zu isolieren. Tragen Sie die Ergebnisse
in das Formblatt f\"ur Fehler ein. 

Geben Sie an welche Fehler Sie entdeckt haben [Zeilennummer].
Klassifizieren Sie dabei nach den
vorgestellten Fehlertypen [{Omission, Comission}, {Initialization, Control,
Interface, Data, Computation, Cosmetic}]. Charakterisieren Sie dann 
kurz den Fehler, z.B. `Initialwert 1 statt 0'.  Geben Sie auch
die Nummer des Fehlverhaltens an, zu dem der Fehler f"uhrt.

Sollten Sie zuf\"alligerweise auch Fehler erkennen, die nicht zu
Fehlverhalten bei Ihren Testergebnisse gef\"urt haben, d\"urfen
Sie sie auch auflisten.  In diesem Fall geh"ort in die Spalte
"`Fv-Nr."' ein Strich.

\item Wenn Sie mit der Suche nach Fehler fertig sind, tragen Sie unter
Punkt 14 Ihren Zeitbedarf ein.

\item F"ullen Sie bitte den Rest des Fragebogens (Fragen 15 bis 17) aus.
Geben Sie den Fragebogen und alle Formbl"atter bei Ihrem Betreuer ab,
	sch\"onen Feierabend.

\end{enumerate}
