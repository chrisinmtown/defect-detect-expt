
Dies ist das Erg\"anzungsblatt bez\"uglich des Programms "`nametbl"'
f\"ur funktionales Testen.

\subsection*{Notwendige Eingaben}

\begin{itemize}

\item Welche Dokumente geh"oren zu dieser Aufgabe?

\begin{enumerate}
\item Dokument ES9, die Spezifikation der Komponente
\item Dokument EQ9, der Quellcode der Komponente (erhalten Sie, nachdem Sie
	Testf"alle erstellt und Fehlverhalten diagnostiziert haben)
\end{enumerate}

\item Wie hole ich mir das Dateipaket, das ich brauche?

Gehen Sie wie folgt vor:

\begin{enumerate}

\item Zuerst legen Sie ein neues Verzeichnis daf\"ur an mittels des 
"`mkdir"'-Kommandos.
\begin{verbatim}
    mkdir ft-nametbl
\end{verbatim}

\item Dann wechseln Sie in das neue Verzeichnis mittels des 
"`cd"'-Kommandos.
\begin{verbatim}
    cd ft-nametbl
\end{verbatim}

\item Zuletzt geben Sie folgendes Kommando ein:
\begin{verbatim}
    tar xf ~uebung00/Experiment/ft-nametbl.tar
\end{verbatim}

\end{enumerate}

\item Was mu"s vorhanden sein ?

Die folgenden Dateien m\"ussen vorhanden sein:
\begin{verbatim}
    Makefile        nametbl         run-suite       test-dir
\end{verbatim}

\end{itemize}

\subsection*{Bildung von "Aquivalenzklassen und Testf"allen}

Die Kommandos mit ihren Parametern werden von der Testumgebung zur Verf"ugung
gestellt, um die Funktionen des Programms m"oglichst einfach ansprechen
zu k"onnen. Bilden Sie keine "Aquivalenzklassen f"ur syntaktisch
falsche Kommandos ! Sie w"urden nur die Testumgebung testen.
Beispiele f"ur unsinnige Testf"alle:
\begin{verbatim}
delete all
\end{verbatim}
Kommando nicht definiert.
\begin{verbatim}
ins
\end{verbatim}
Argument fehlt.

\newpage
\section*{E29, Blatt 2 -- Wird mit dem Quellkode ausgegeben !}


\subsection*{Kurze Beschreibung der verwendeten Bibliotheks-Funktionen}

\begin{itemize}
	\item {\tt assert(expression)}\\
		Makro, der erwartet, da"s {\tt expression} zum Zeitpunkt der Ausf"uhrung
		wahr ist und in diesem Fall nichts tut. Ansonsten wird eine 
		Fehlermeldung mit der {\tt expression} ausgegeben und das Programm wird
		beendet.
	\item {\tt char *malloc(unsigned size)}\\
		Liefert ein Pointer auf einen Speicherblock zur"uck, der mindestens 
		{\tt size}
		Bytes gro"s ist. Bei Mi"serfolg wird NULL zur"uckgegeben.
	\item {\tt int free(char *ptr)}\\
		Gibt einen zuvor allokierten Block an das System zur"uck. {\tt ptr} 
		mu"s mit 
		{\tt malloc} o."a. allokiert worden sein. Bei Erfolg wird 1, sonst
 		0 zur"uckgegeben.
	\item {\tt int strcmp(char *s1, char *s2)}\\
		Vergleicht zwei Strings. Liefert einen Wert gr"o"ser als, gleich, oder
		kleiner als 0 zur"uck, abh"angig davon, ob {\tt s1} lexikographisch
		(d.h.~ASCII-Zeichenwert) gr"o"ser, gleich, oder 
		kleiner als {\tt s2} ist.	
	\item {\tt char *strdup(char *s1)}\\
		Dupliziert den String {\tt s1}. Alloziiert zu diesem Zweck Speicher
		mit {\tt malloc()} und gibt bei Erfolg von {\tt malloc} den 
		Pointer darauf zur"uck, sonst NULL.
	\item {\tt char *tsearch(char *key, char **rootp, int (*compar)())}
	\item {\tt char *tfind(char *key, char **rootp, int (*compar)())}
	\item {\tt void twalk(char *root, void (*action)())}\\
		Diese Funktionen implementieren einen bin"aren Suchbaum. Alle n"otigen
		Vergleiche werden mit der Funktion {\tt compar} durchgef"uhrt, die
		vom Benutzer der Baumfunktionen zur Verf"ugung gestellt werden mu"s. Die
		Funktion {\tt compar} wird den Baumfunktionen mit Hilfe eines
		Funktionspointers bekannt gemacht. {\tt compar} wird mit 
		zwei Argumenten
		aufgerufen, die auf die zu vergleichenden Elemente zeigen. Der 
		R"uckgebewert mu"s kleiner, gleich oder gr"o"ser 0 sein, abh"angig
		davon, ob das erste Element kleiner, gleich oder gr"o"ser dem zweiten
		Element ist (analog zu {\tt strcmp}).

		{\tt tsearch} kann einen Baum aufbauen und auf ihn zugreifen. Wenn
		ein Element gefunden wird, da"s {\em gleich\/} dem ist, auf das
		{\tt key} zeigt, wird ein Pointer zur"uckegegeben. Dieser Pointer
		zeigt auf einen Pointer der auf das gefundene 
		Baumelement zeigt. Ansonsten wird das Element, auf das {\tt key}
		zeigt, in den Baum eingef"ugt und ein Pointer-auf-einen-Pointer auf
		das neue Element zur"uckgegeben. Illustration:
		\begin{verbatim}
   Rueckgabe-   --            ------------- 
    ---------->|  |--------->| Baumelement |
    pointer     --  Pointer   ------------- 
                                  /    \
		\end{verbatim}
		Es werden nur Pointer kopiert, d.h.\ der Benutzer mu"s
		f"ur die Baumelemente selbst Speicher reservieren (z.B.~mit {\tt malloc}).
		{\tt rootp} zeigt auf eine Variable und diese Variable zeigt auf die
		Wurzel des Baums. Um den ersten Knoten im Baum (Wurzel) zu erzeugen,
		mu"s in der Variablen NULL stehen, danach enth"alt die Variable 
		den Zeiger auf die Wurzel des Baums.
		
		{\tt tfind} entspricht {\tt tsearch}, nur das bei Scheitern der
		Suche NULL zur"uckgegeben und kein neuer Knoten angelegt wird.

		{\tt twalk} traversiert einen Baum oder auch Teilbaum, denn als {\tt root}
		kann auch ein beliebiger Knoten angegeben werden. Beachten Sie, da"s nur
		ein Pointer auf {\tt root} angegeben wird, kein 
		Pointer-auf-einen-Pointer,	wie bei {\tt tsearch} und {\tt tfind}.
		{\tt action} ist die Funktion, die mit jedem Knoten aufgerufen wird.
		Sie mu"s auch von Benutzer zur Verf"ugung gestellt werden. Sie erh"alt
		drei Argumente: einen Pointer auf das Baumelement, eine 
		Information, wann das Baumelement besucht wurde und die Tiefe des 
		Baumelements (Wurzel = 0). Die Besuchsinformation ist ein
		Enumerationstyp: {\tt typedef enum \{preorder, postorder, 
		endorder, leaf \} VISIT;} {\tt preoder}, {\tt postoder} und 
		{\tt enoder} stehen 
		f"ur den ersten, zweiten und dritten Besuch w"ahrend einer
		"`depth-first, left-to-right"' Traversion des Baumes, {\tt leaf} f"ur
		das Blatt eines Baumes.

	\item {\tt int sscanf(s, format, pointer)}\\
		Lie"st Zeichen von String {\tt s} und wandelt sie entsprechend dem
		spezifizierten {\tt format} um und schreibt die Resultate in die Variablen
		auf die die {\tt pointer} zeigen. Dies ist die Umkehrfunktion 
		von {\tt printf}, die Formatspezifikationen sind identisch. 
	\item {\tt char *fgets(s, n, stream)}\\
		Lie"st Zeichen aus dem Stream {\tt stream} bis zum NEWLINE oder EOF und
		schreibt sie in den String {\tt s}, bis {\tt n-1} Zeichen gelesen wurden.
		Bei EOF wird NULL zur"uckgegeben, sonst {\tt s}.
\end{itemize}

