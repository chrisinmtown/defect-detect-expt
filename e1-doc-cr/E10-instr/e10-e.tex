Please follow the directions on the supplemental sheet entitled ``Code
reading with program $x$,'' which is included in this packet of materials.

\begin{enumerate}

\item Before beginning, read over the data-collection form ``E12: Code Reading.'' 
Enter your identifier and fill out the section entitled ``Before beginning.''

\end{enumerate}

\subsection*{Read the code}

\begin{enumerate}
\addtocounter{enumi}{1}

\item Enter the time when you began the experiment under question~4.

\item Briefly read through the code to gain an overview of this component.  
If you happen to notice faults during this or one of the following
three steps, mark  them.  However, please do not invest any time in a
precise analysis; i.e., don't isolate the faults.  Isolation of faults
begins in step 11. 

\item Determine the dependencies between the individual functions in
the source code, for example with a call tree.
Normally the functions are already ordered in the code such that the
``low-level'' functions (leaves) are at the beginning and the
``high-level'' functions (root) at the end.
Begin applying the code-reading technique with the leaf functions and
work towards the root.

\item Develop an understanding for the structure of each individual
function by identifying the elementary structures (sequences,
conditional assignments, loops) and by marking them, for example by
drawing a box around them. 
Combine the elementary structures together to form larger structures
(with corresponding boxes) until you have framed in the entire
function. 

\item Try to determine the meaning of each structure (box) by starting
with the innermost box and note the result on the worksheet given to
you for that purpose.
Use line numbers (lines x--y) when doing so.
Avoid using implicit knowledge that does not lie inside this box, for
example initial values, inputs, or parameter values.
Describe the parts as functionally as possible.
While doing so, use generally accepted principles of the application
domain to keep the description brief and understandable.  For example,
in the case of searching a tree, state ``breadth-first search'' instead
of a lengthy description of breadth-first search.

\item Enter the time you required to compose the abstractions 
in the data-collection form under question~5.
Also please enter the maximum number of abstraction levels and the
style of description you used under questions~6 and~7, respectively.

\item Make a copy of your notes (you will still need the original) and
write your identifier on the copy.  Give the copy to the exercise
leader.  You will then receive the specification of the component.

\end{enumerate}

\subsection*{Search for inconsistencies}

\begin{enumerate}
\addtocounter{enumi}{8}

\item Aswer question~8 on the data-collection form.
Read through the specification carefully.  The specification describes
the component as a whole, not the individual functions.  You must
derive the specification for the individual functions from the 
component's specification. 
(This is necessary so that the same specification can be used in all
exercises for reasons of comparability.)
Detect possible inconsistences by comparing the specification with your
description of the component.
Enter the inconsistencies which you detect in the worksheet, and don't
forget to enter your identifier there as well.   

\item When you believe that you have detected all of the
inconsistencies, enter the time you required for your search under
question~9 of the data-collection form.
Please number the inconsistencies which you found from 1 to $n$ in the
column labeled ``I. Nr.'' (inconsistency number) in the worksheet for
inconsistencies. 

\end{enumerate}

\subsection*{Isolate faults}

\begin{enumerate}
\addtocounter{enumi}{10}

\item Try to isolate the faults responsible for the inconsistencies
which you detected.
But please answer question~10 on the data-collection form first.
Enter the isolated faults in the worksheet provided for that purpose,
and again don't forget to enter your identifier on that sheet.

Record the faults which you isolated by using the line number.
Classify the faults according to the different types and classes of
faults [\{Omission, Comission\}, \{Initialization, Control,
Interface, Data, Computation, Cosmetic\}]. 
Then characterize the fault briefly; e.g., ``initial value was 1 instead
of 0.''
Also give the number(s) of the inconsistency or inconsistencies which
result(s) from this fault.

Also enter in the same way the faults which you found by chance.
These faults receive just a dash in the column ``I. Nr.''

\item Enter the time you required for isolating faults in the
data-collection form under question~11.

\item Please fill out the rest of the form (questions~12--14).

\item Give the data-collection form and all other worksheets to the
experiment leader, and you're all done.

\end{enumerate}
