
Dies ist das Erg\"anzungsblatt bez\"uglich des Programms "`tokens"'
f\"ur strukturelles Testen.

\subsection*{Notwendige Eingaben}

\begin{itemize}

\item Welche Dokumente geh"oren zu dieser Aufgabe?

\begin{enumerate}
\item Dokument TS3, die Spezifikation der Komponente (erhalten Sie, nachdem
   Sie Testf"alle erstellt und Fehlverhalten diagnostiziert haben)
\item Dokument TQ3, der Quellcode der Komponente
\end{enumerate}

\item Wie hole ich mir das Dateipaket, das ich brauche?

%Das Dateipaket befindet sich auf unseren Anlage in einer tar-Datei.
% Der folgende Proze{\ss} soll gefolgt werden.
Gehen Sie wie folgt vor:

\begin{enumerate}

\item Zuerst legen Sie ein neues Verzeichnis daf\"ur an mittels des 
"`mkdir"'-Kommandos.
\begin{verbatim}
    mkdir st-tokens
\end{verbatim}

\item Dann wechseln Sie in das neue Verzeichnis mittels des 
"`cd"'-Kommandos.
\begin{verbatim}
    cd st-tokens
\end{verbatim}

\item Zuletzt geben Sie folgendes Kommando ein:
\begin{verbatim}
    tar xf ~prakt00/Aufgabe5/st-tokens.tar
\end{verbatim}

\end{enumerate}

\item Was mu"s vorhanden sein ?

Die folgende Dateien m\"ussen vorhanden sein:
\begin{verbatim}
    Makefile     gct-map      run-suite    test-dir     tokens
\end{verbatim}

% 
% \item Wie wird die Umgebung initialisiert ?
% \begin{verbatim}
%     % make gct
% \end{verbatim}
% Es erscheint:
% \begin{verbatim}
%     gct-init
%     make tokens CC=gct
%     gct -g  -target sun4 -c  tokens.c
%     gct -o tokens tokens.o
% \end{verbatim}
% 

\end{itemize}


\subsubsection*{Schreiben von Testf"allen}

Getestet werden alle Funktionen des Programms grunds"atzlich durch den
Aufruf 
\begin{verbatim}
tokens
\end{verbatim}
Das Programm lie{\ss}t seine Eingabe von der Standard-Eingabe (stdin) und
kann "uber Optionen beeinflu"st werden. Eine Parameterdatei enth"alt folglich 
die Option(en), den Namen der Eingabedatei 
und das notwendige Zeichen, um der Inhalt der Datei zum Programm
umzulenken (``<'').
%Dies ist zwar etwas umst"andlich, aber aus Konsistenzgr"unden n"otig.
Das Format der Parameterdatei ist wie folgt:
\begin{verbatim}
    -Option1 -Option2 < file1
\end{verbatim}

Welche Optionen erlaubt sind und wie die Eingabedatei gestaltet sein
mu"s, entnehmen Sie den Funktionen der Komponente. 
