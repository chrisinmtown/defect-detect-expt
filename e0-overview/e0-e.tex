The experiment consists of two parts, namely a set of exercises to
introduce you to the techniques and experimental procedures (the
training phase) and the actual comparison (the live experiment).

\section*{The Training Phase}

The training phase offers you the chance to familiarize yourself
with the techniques and the procedures that will be used during the
experiment.

\paragraph{Procedure:}
All participants will apply the technique ``code reading'' to the
program ``count,'' will apply ``functional testing'' to ``series,''
and will apply  ``structural testing'' to ``tokens.''
Pick up the necessary materials at your convenience, follow the
instructions, and turn in the completed data collection forms.
{\bf Please work alone.}

\medskip

Each technique is fundamentally performed in four separate steps:
\begin{enumerate}
	\item Testing: create test data; Reading: create own specification.
	\item Testing: run tests; Reading: no such step.
	\item Testing: identify failures; Reading: identify inconsistencies.
	\item Isolate faults.
\end{enumerate}


\section*{The Live Experiment}

The experiment will be run over three days at fixed times.
All participants will see the same program on the same day,
but will apply different techniques.
The experiment is otherwise performed exactly the way the training
phase worked.


\paragraph{Procedure:} 
Information about the times, dates, and division of exercises is
listed on the next page.
Meetings will be held on the times and dates listed. to discuss the
experiment and to distribute the required materials.
You should plan to spend a minimum of three (3) hours per day on these
exercises.
{\bf Please arrive on time!}


\begin{quote}
\bf 
If it is simply impossible for you to participate at your preassigned
time, please contact the person responsible for the experiment
immediately.
\end{quote}

\bigskip

A group feedback session has been planned for 14 June 19xx at 10:00.

\bigskip

% An overview of the process to be followed while applying each
% defect-detection technique is repeated here.

%------------------------------------------------------------------------------

\newpage

\subsection*{Times, dates, and assignment of exercises}

\begin{center}
\begin{tabular}{|r|c|l|l|}
\hline
\bf Time and date         & \bf Groups  & \bf Technique        & \bf Program\\
\hline
Tuesday, 7 June 19xx      &  5, 6       &  Structural testing  &  ntree\\
$10^{00}$ or $15^{00}$    &  3, 4       &  Functional testing  & \\
                          &  1, 2       &  Code reading        & \\
\hline
Wednesday, 8 June 19xx    &  2, 4       &  Structural testing  & cmdline \\
$15^{00}$                 &  1, 6       &  Functional testing  & \\
                          &  3, 5       &  Code reading        & \\
\hline
Thursday, 9 June 19xx     &  1, 3       &  Structural testing  & nametbl \\
$14^{00}$                 &  2, 5       &  Functional testing  & \\
                          &  4, 6       &  Code reading        & \\
\hline
\end{tabular}
\end{center}


\subsection*{Division into groups}

\begin{center}

\begin{tabular}{|l|c|}
\hline
\bf Identifier &  \bf Group \\
\hline
prakt01       &      1      \\
prakt02       &      2      \\
prakt03       &      3      \\
prakt04       &      4      \\
prakt06       &      5      \\
\hline
prakt07       &      6      \\
prakt08       &      1      \\
prakt09       &      2      \\
prakt10       &      3      \\
prakt11       &      4      \\
\hline
prakt12       &      5      \\
prakt13       &      6      \\
prakt15       &      1      \\
prakt17       &      2      \\
prakt18       &      3      \\
\hline
prakt19       &      4      \\
prakt20       &      5      \\
prakt21       &      6      \\
prakt22       &      1      \\
prakt23       &      2      \\
\hline
prakt24       &      3      \\
prakt25       &      4      \\
prakt26       &      5      \\
prakt27       &      6      \\
prakt28       &      1      \\
\hline
prakt29       &      2      \\
prakt30       &      3      \\
prakt31       &      4      \\
\hline
\end{tabular}

\end{center}

\newpage

\section*{Classifying Faults}

This document gives a brief overview of the fault-classification
scheme defined by Basili \& Selby (1987) for their experiment and
reused in the Kamsties \& Lott experiment (1995).

The first facet for classifying faults is to distinguish whether
something was forgotten or incorrectly written.

\begin{itemize}

\item Omission: Something was forgotten.
Example faults of class ``omission'' are a missing assignment, a
missing variable initialization, or a missing parameter in a function
call. 

\item Comission: Something was written incorrectly.
Example faults of class ``comission'' are using ``+'' instead of
``-'', assigning an incorrect initial value, or checking the wrong
case in an if-condition.

\end{itemize}

Faults can also be classified according to the semantic problem
expressed by the fault, as follows.

\begin{itemize}

\item Initialization:  The value of a variable has a different value
than expected when it is first accessed.
For example, a control variable in a for-loop is initialized to 1
instead of 0.

\item Computation: The value of an express is computed incorrectly.
For example, a ``+'' is used instead of ``-'' or the wrong constant
value is used.

\item Control: The path of control differs from the expected path.
For example, a conditition was written incorrectly in an if-statement
or a loop.

\item Interface: The interface of an element or module was understood
and used incorrectly.
For example, many C library functions that process strings expect the
string variable to be null terminated; handing the function a string
that is null terminated (or omitting a parameter entirely) would be an
interface fault. 

\item Data: Data was used incorrectly.
For example, an element in a linked list is missed or the computation
of an array element is incorrect.

\item Cosmetic: The presentation of the component to the user is
false.
For example, an incorrectly spelled word that is printed in an error
is an example of a cosmetic fault.


\end{itemize}

With the exception of simple cases such as missing parameters in a
function call, classifying faults into these semantic classes is
frequently not obvious or repeatable.  
Classification depends strongly on some idea of how the code should
have been written, and therefore depends on the tester's subjectivity.
Different testers will sometimes identify different ways in which the
fault can be removed, which in turn changes the classification of the
fault.

