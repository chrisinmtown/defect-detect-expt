%
% specification for ntree
%

\subsection*{Name}

ntree -- Functions for managing a tree

\subsection*{Usage}
{\bf ntree} input-file

\subsection*{Description}
{\bf ntree} reads commands from a file and processes them in order to
test a few functions. 
Considered together, the functions implement a tree in which each node
can have any number of child nodes.
The number of children is not limited by the program.
A node holds two strings (key, content).

\subsection*{Options}
None.

\subsection*{Input Data}
Only the following commands are allowed in the input files.
Each command must start on a new line.
Lower and upper case is relevant in the keys.
The separator between commands and arguments is a space; this means that
spaces may not appear within a string.
In this implementation, command lines (command plus corresponding
parameters) read from the file are limited to a maximum length of
1024. 


\begin{itemize}

\item {\tt root <key> <content>}

Creates and initializes the root of the tree.
This command must be the first command in the input file.
If it does not appear as the first command, or if the command is
repeated, the behavior of the program is not defined.

\item {\tt child <father-key> <child-key> <child-content>}

Attaches the new child node with the given data onto the given father
node. 
If the father node is not found in the tree, a corresponding error
message is printed. 
If the key of the father-node appears more than once in the tree, the
new child is attached to the first node found (see also ``search''
below).

\item {\tt search <key>}

The tree is searched for the node with the key ``<key>''.
The tree is searched from left to right using depth-first search,
in which the key of a node is checked on the first visit (preorder
traversal). 
If the node with the given key is found, the contents are printed out;
if not, a corresponding error message is printed.
If multiple nodes have the same key, only the contents of the first
one found are printed. 

\item {\tt sibs <key1> <key2>}

Tests whether the nodes with the given keys are siblings; i.e.,
whether they share the same father node.
If one or the other key is not found in the tree, a corresponding
error message is printed.
Every node counts as its own sibling.  
See also ``search'' for a description of the search method.

\item {\tt print}

Prints out the complete contents of the tree.  A new line is printed
for each node with the key, content, and the depth in the tree.
Each line is indented to correspond with the depth of the
node (4 blanks per level).

\end{itemize}

\subsection*{Example}

Because the output of the program as presented here was formatted by a
typesetting program, the indentation shown here may not correspond
exactly with the actual indentation.

{\small
\begin{verbatim}
% cat eingabe
root Lamb SE:_Planning_for_Change
child Lamb Rombach Software_Specifications
child Lamb Brackett Software_Requirements
child Rombach Parnas Predicate_Logic_for_SE
print

% ntree eingabe

Eingabedatei `eingabe' wird bearbeitet.

Die Zeile `root Lamb SE:_Planning_for_Change' wird ausgewertet:

Die Zeile `child Lamb Rombach Software_Specifications' wird ausgewertet:

Die Zeile `child Lamb Brackett Software_Requirements' wird ausgewertet:

Die Zeile `child Rombach Parnas Predicate_Logic_for_SE' wird ausgewertet:

Die Zeile `print' wird ausgewertet:
Knoten (Ebene 0): Schluessel 'Lamb', Inhalt 'SE:_Planning_for_Change'
    Knoten (Ebene 1): Schluessel 'Rombach', Inhalt 'Software_Specifications'
        Knoten (Ebene 2): Schluessel 'Parnas', Inhalt 'Predicate_Logic_for_SE'
    Knoten (Ebene 1): Schluessel 'Brackett', Inhalt 'Software_Requirements'

Ende der Eingabedatei `eingabe'.
\end{verbatim}
}
\subsection*{Author}
Ch. Lott
\subsection*{Bugs}
The functionality for reading files is only for test purposes and is
therefore not especially fault-tolerant.  For example, one can assume
that incorrectly spelled commands, commands with missing arguments, or
commands with too many arguments will not be handled especially
carefully. 

