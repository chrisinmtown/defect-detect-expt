
\textbf{\large Name}

cmdline -- Syntaktische und semantische Analyse einer Kommandozeile


\medskip
\textbf{\large Verwendung}

{\bf cmdline} --hilfe

{\bf cmdline} --mass $<$MASS$>$ [ Suchoption ] Datei [ Datei ... ]


\medskip
\textbf{\large Beschreibung}

{\bf cmdline} analysiert eine Kommandozeile eines Messwerkzeugs auf
syntaktische und teilweise auch auf semantische Korrektheit. 
Der Benutzer mu{\ss} dem Programm genau ein Ma{\ss} und mindestens
einen Dateiname \"ubergeben.
Die Einheiten, die evtl. gemessen werden sollten, sind in Dateien
gespeichert. 
Zu den Argumenten (Ma{\ss}, Dateinamen usw.) darf noch genau eine
sogenannte Suchoption \"ubergeben werden.
Die Reihenfolge zwischen Ma{\ss}angabe und Suchoption ist beliebig.
Bei der Auswertung der Kommandozeile werden alle Argumente ab 
(und einschlie{\ss}lich) dem ersten Argument, welches keine Option ist,
als Dateinamen interpretiert.  
Ob Dateinamen zu existierenden Dateien korrespondieren wird aber nicht
gepr\"uft. 

Bei Erfolg (d.h., wenn sowohl die Syntax als auch die Semantik der
Argumente als zul\"assig befunden wurde) wird eine Zusammenfassung der
gepr\"uften Argumente ausgegeben, ansonsten eine erkl\"arende
Fehlermeldung.  Die Zusammenfassung besteht aus dem zu 
berechnenden Ma{\ss}, ggf.\ der Suchoption (mit Wert bei --unter und
--ueber) und einer Auff\"uhrung der einzulesenden Dateien.


\medskip
\textbf{\large Optionen}

Optionen und Argumente f\"ur Optionen k\"onnen bis auf ihren eindeutigen
Anfang gek\"urzt werden. 
Im folgenden Text steht der eindeutige Teil vor den eckigen Klammern.
Es kann maximal eine Suchoption verwendet werden. 

\begin{itemize}
\item --h[ilfe] 

Hilfeoption.  Es wird nur ein Hilfstext ausgegeben und sonst nichts gemacht.

\item --?

Hilfeoption. Siehe Option ``--hilfe''.

\item --mas[s] $<$MASS$>$

Zul\"assige Ma{\ss}angaben f\"ur $<$MASS$>$ sind GKO[M],  LKO[M], GKH[M],
LKH[M], GI[HE] und  LI[HE].

\item --a[lle]

Suchoption. Das Programm soll das Ma{\ss} f\"ur alle Einheiten bestimmen.

\item --max

Suchoption. Das Programm soll den maximalen Wert des Ma{\ss}es f\"ur alle
Einheiten bestimmen.

\item --mi[n]

Suchoption.  Das Programm soll den minimalen Wert des Ma{\ss}es f\"ur alle
Einheiten bestimmen.

\item --du[rchschnitt]

Suchoption.  Das Programm soll den durchschnittlichen Wert des Ma{\ss}es
f\"ur alle Einheiten bestimmen.

\item --un[ter] $<$Grenzwert$>$

Suchoption.  Das Programm soll alle Einheiten identifiziern,
deren Wert f\"ur das angegebene Ma{\ss} unter dem Grenzwert liegt.
Der angegebene Grenzwert kann eine beliebige Realzahl sein.

\item --ue[ber] $<$Grenzwert$>$

Suchoption.  Das Programm soll alle Einheiten identifiziern,
deren Wert f\"ur das angegebene Ma{\ss} \"uber dem Grenzwert liegt.
Der angegebene Grenzwert kann eine beliebige Realzahl sein.

\end{itemize}



\medskip
\textbf{\large Beispiel}

{\small
\begin{verbatim}
% cmdline -mass GKHM -alle datei1
Die Aufgabe ist:
Mass:  GKHM
Suche: -alle
Anzahl der zu lesenden Dateien: 1
Die Dateien sind:
   datei1
\end{verbatim}
}


\medskip
\textbf{\large Autoren}

Baumg\"artner, Cla{\ss}en, Gieseke, Lott.
