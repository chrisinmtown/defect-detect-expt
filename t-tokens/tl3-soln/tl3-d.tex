%
% sample solution for exercise tokens
%

Die Komponente ``tokens'' hat die folgenden Fehler und weist die
entsprechenden Fehlverhalten auf:

\begin{enumerate}

\item Fehler in Funktion ``treeprint'', Zeile 25: Das Zeichen `$>$'
sollte `$>=$' sein. 

F"uhrt zum Fehlverhalten:
Falls ein Grenzwert mit dem Argument ``-m'' 
angegeben wird (``n''), wird n + 1 verwendet.


\item Fehler in Funktion ``getword'', Zeile 64/77: Die L"ange der
Eingabe wird nicht kontrolliert. 

F"uhrt zum Fehlverhalten:
Falls ein "uberlanges Wort eingelesen wird,
bricht das Programm mit einem Core-Dump ab.


\item Fehler in Funktion ``main'', Zeile 97 (circa): Das Array
``MapAllowed'' wird nie initialisiert. 

F"uhrt zum Fehlverhalten:  Compiler-Abh"angig; nicht alphanumerische
Zeichen k"onnten akzeptiert werden.


\item Fehler in Funktion ``main'', Zeile 101: Die Variable ``Alpha''
wird mit 0 statt mit 1 initialisiert.  

F"uhrt zum Fehlverhalten:
Das Argument ``-a'' hat keine Auswirkung.


\item Fehler in Funktion ``main'', Zeile 110: Es wird nicht gepr\"uft,
ob eine g\"ultige Zahl angegeben wurde.

F"uhrt zum Fehlverhalten: 
Ung\"ultige Zahlen werden ohne Fehlermeldung akzeptiert.


\item Fehler in Funktion ``main'', Zeile 115: Das Argument ``-a''
fehlt in der Usage-Meldung. 

F"uhrt zum Fehlverhalten:
Die Hilfemeldung des Programs schweigt sich "uber das Argument ``-a'' aus. 

\end{enumerate}

\newpage

Die folgenden Testf"alle erzielen die maximal m"oglichen
"Uberdeckungwerte f"ur das Programm.  


\begin{enumerate}
\item Testfall: Unbekannte Option.

\begin{description}
\item[Aufruf mit:] -q
\item[Eingabe:] keine.
\item[Erwartete Ausgabe:] Fehlermeldung, Hilfemeldung f"ur das Programm.
\end{description}

\item Testfall: Keine Optionen, Leere Datei.

\begin{description}
\item[Aufruf mit:] $<$ /dev/null
\item[Eingabe:] keine.
\item[Erwartete Ausgabe:] keine.
\end{description}

\item Testfall: Eine Option, Leere Datei.

\begin{description}
\item[Aufruf mit:] -c '' $<$ /dev/null
\item[Eingabe:] keine.
\item[Erwartete Ausgabe:] keine.
\end{description}

\item Testfall: Wenige Optionen, sehr kurze Datei.

\begin{description}
\item[Aufruf mit:] -c / $<$ shortfile
\item[Eingabe:] Eine Datei namens ``shortfile'' mit dem einzelnen Wort
``nonl'' und keinen weiteren Zeichen vor dem Ende der Datei; d.h.,
kein sogen.\ `newline'. 
\item[Erwarte Ausgabe:]  Die folgende Einzelzeile:

	\begin{quote}
	1 nonl
	\end{quote}
\end{description}

\item Testfall: Viele Optionen, kurze Datei.

\begin{description}
\item[Aufruf mit:] -a -i -c /. -m 1 $<$ infile
\item[Eingabe:] Eine Datei namens ``infile'' mit folgenden Eintr"agen:

	\begin{quote}
	word1\\
	word1\\
	word1\\
	word2\\
	word2\\
	$*$\\
	b\\
	a
	\end{quote}

\item[Erwartete Ausgabe:] Die folgende Zeilen:

	\begin{quote}
	1 a\\
	1 b\\
	3 word1\\
	2 word2
	\end{quote}
\end{description}
\end{enumerate}
Ende der notwendigen Testf"alle.

\newpage

Nachdem die oben aufgelisteten Testf"alle ausgef\"uhrt werden,
werden die folgenden "Uberdeckungswerte erreicht.

{\small
\begin{verbatim}
% gsummary test-dir/GCTLOG 
BINARY BRANCH INSTRUMENTATION (38 conditions total)
2 ( 5.26%) not satisfied.
36 (94.74%) fully satisfied.

SWITCH INSTRUMENTATION (5 conditions total)
0 ( 0.00%) not satisfied.
5 (100.00%) fully satisfied.

LOOP INSTRUMENTATION (21 conditions total)
7 (33.33%) not satisfied.
14 (66.67%) fully satisfied.

OPERATOR INSTRUMENTATION (15 conditions total)
0 ( 0.00%) not satisfied.
15 (100.00%) fully satisfied.

SUMMARY OF ALL CONDITION TYPES (79 total)
9 (11.39%) not satisfied.
70 (88.61%) fully satisfied.
\end{verbatim}
}

\noindent
Die Werte, die 100\%  unterschreiten, k"onnen mit dem Kommando 
``greport'' n"aher erl"autert werden.  Das Kommando liefert die
folgende Ausgabe:
{\small
\begin{verbatim}
% greport test-dir/GCTLOG 
"tokens.c", line 38: if was taken TRUE 5, FALSE 0 times.
"tokens.c", line 64: loop zero times: 0, one time: 3, many times: 8.
"tokens.c", line 118: loop zero times: 0, one time: 0, many times: 3.
"tokens.c", line 120: loop zero times: 0, one time: 0, many times: 3.
"tokens.c", line 122: if was taken TRUE 3, FALSE 0 times.
"tokens.c", line 123: loop zero times: 0, one time: 0, many times: 3.
\end{verbatim}
}

\noindent
Diese Zeilen/Konstrukte k"onnen aus den folgenden Gr\"unden nicht
besser \"uberdeckt werden:

\begin{description}
\item[Zeile 38:] Diese ``if''-Bedingung "uberpr"uft, ob ``calloc''
tats\"achlich Speicher allokiert hat.  Es ist schwierig zu
provozieren, da{\ss} ``calloc'' keinen Speicher allokiert, insbesondere
f"ur Benutzer, die das ``limit''-Kommando der Shell nicht kennen.

\item[Zeile 64:] Es handelt sich hier um eine Endlos-Schleife (die
Terminierungskriterien der Schleife sind leer).  Daher wird diese
Schleife immer mindestens einmal ausgef"uhrt.

\item[Zeilen 118, 120, und 123:]  Diese ``for''-Schleifen haben
hartkodierte, fixierte Terminierungskriteren, die gr\"o{\ss}er als
eins sind und werden daher immer mehr als einmal ausgef\"uhrt.

\item[Zeile 122:] Diese ``if''-Bedingung ist immer wahr aufgrund eines
Fehlers im Programm (der ``Alpha''-Variable wird kein Wert zugewiesen).

\end{description}

