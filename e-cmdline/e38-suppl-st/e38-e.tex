
This is the supplemental sheet 
for applying technique ``structural testing''
to the program ``cmdline.''

\subsection*{Necessary inputs}

\begin{itemize}

\item Which documents belong to this exercise?

\begin{enumerate}
\item Document ES8, the specification of the component, which you will
receive after you have created test cases and attempted to reach 100\%
coverage.
\item Document EQ8, the source code of the component
\end{enumerate}

\item How do I fetch the files which I need?

Do the following:

\begin{enumerate}

\item First create a new directory for this exercise with the
``mkdir'' command.
\begin{verbatim}
    mkdir st-cmdline
\end{verbatim}

\item Then change to the new directory with the ``cd'' command.
\begin{verbatim}
    cd st-cmdline
\end{verbatim}

\item Finally, enter the following command:
\begin{path}
\begin{verbatim}
    tar xf ~prakt00/Exercise5/st-cmdline.tar
\end{verbatim}
\end{path}

\end{enumerate}

\item What should I have?

The following files must be available.
\begin{verbatim}
    Makefile     cmdline     gct-map    run-suite    test-dir
\end{verbatim}

\end{itemize}


\subsubsection*{Writing test cases}

All of the program's functions are fundamentally tested via 
the invocation 
\begin{verbatim}
cmdline [ -argument ... ]
\end{verbatim}
Because ``cmdline'' requires no input file, only parameter files
with the command-line arguments need to be created.

Which arguments are allowed on the command line and which functions
are invoked by those commands can be seen in the function 
{\tt process\_switches} in the file {\tt cmdline.c}.
The test scaffolding is transparent.
The function {\tt process\_switches} is given the same argument vector
({\tt argc, argv[]}) with which ``cmdline'' is invoked.

\subsection*{Brief description of the library functions used}

\begin{itemize}

\item {\tt int strlen(char *s)}

Returns the length of the string {\tt s} (the null character on the end
is not counted).

\item {\tt int strcmp(char *s1, char *s2)}

Compares two strings.  Returns a value greater than, equal to, or less
than 0 depending on whether {\tt st1} is lexicographically (i.e.,
ASCII value) greater than, equal, or less than {\tt s2}.

\item {\tt int strncmp(char *s1, char *s2, int n)}

Like {\tt strcmp}, but compares maximum {\tt n} characters.

\item {\tt int fprintf(stderr, "Text"...)}

Prints the text on the standard-error output stream.

\item {\tt int isdigit(char c)}
Returns {\em true} if the character is a digit, otherwise {\em false}.

\item {\tt int atoi(char *str)}

Converts the string {\tt str} to an integer.
Leading white space is ignored, and scanning of the string ends
with the first non-numeric character.
If no numeric characters could be read, 0 is returned.

\end{itemize}
