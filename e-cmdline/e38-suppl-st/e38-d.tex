
Dies ist das Erg\"anzungsblatt bez\"uglich des Programms "`cmdline"'
f\"ur strukturelles Testen.

\bigskip

\subsection*{Notwendige Eingaben}

\begin{itemize}

\item Welche Dokumente geh"oren zu dieser Aufgabe?

\begin{enumerate}
\item Dokument ES8, die Spezifikation der Komponente (erhalten Sie, nachdem
	Sie die Testf"alle erstellt haben)
\item Dokument EQ8, der Quellcode der Komponente
\end{enumerate}

\item Wie hole ich mir das Dateipaket, das ich brauche?

Gehen Sie wie folgt vor:

\begin{enumerate}

\item Zuerst legen Sie ein neues Verzeichnis daf\"ur an mittels des 
"`mkdir"'-Kommandos.
\begin{verbatim}
    mkdir st-cmdline
\end{verbatim}

\item Dann wechseln Sie in das neue Verzeichnis mittels des 
"`cd"'-Kommandos.
\begin{verbatim}
    cd st-cmdline
\end{verbatim}

\item Zuletzt geben Sie folgendes Kommando ein:
\begin{verbatim}
    tar xf ~uebung00/Experiment/st-cmdline.tar
\end{verbatim}

\end{enumerate}

\item Was mu"s vorhanden sein ?

Die folgende Dateien m\"ussen vorhanden sein:
\begin{verbatim}
    Makefile    cmdline     gct-map    run-suite    test-dir
\end{verbatim}

% \item Wie wird die Testumgebung initialisiert ?
% \begin{verbatim}
%     % make gct
% \end{verbatim}
% Es erscheint:
% \begin{verbatim}
%     gct-init
%     make cmdline CC=gct
%     gct -g  -target sun4 -c  cmdline.c
%     gct -o cmdline cmdline.o str-driver.o
% \end{verbatim}

\end{itemize}


\subsubsection*{Schreiben von Testf"allen}

Getestet werden alle Funktionen des Programms grunds"atzlich durch den
Aufruf 
\begin{verbatim}
cmdline [ -Argumente ... ] 
\end{verbatim}
Da "`cmdline"' keine Eingabedateien ben"otigt, sind nur Parameterdateien
mit Aufruf-Argumenten anzulegen.

Welche Argumente die Kommandozeile erlaubt und welche 
Funktionen dadurch aufgerufen werden, entnehmen Sie der Funktion 
{\tt process\_switches} in der Datei {\tt cmdline.c}.
Die Testumgebung aus {\tt driver.o} ist transparent, 
die Funktion {\tt process\_switches} wird mit dem gleichen Argumentvektor 
({\tt argc, argv[]}) aufgerufen, mit dem auch "`cmdline"' aufgerufen wurde. 

%Der Treiber erwartet, da{\ss} der von der Funktion zur\"uckgelieferte
%Wert die Anzahl der Switches entspricht, die am Anfang der Zeile 
%stehen und von der Funktion erkannt wurden.   Die Treiberfunktion
%druckt diesen Wert aus und versucht sie zu verwenden um die weitere
%Kommandozeilenargumente auszudrucken. 

\subsection*{Kurze Beschreibung der verwendeten Bibliotheks-Funktionen}

\begin{itemize}
   \item {\tt int strlen(char *s)}\\
      Liefert die L"ange des Strings {\tt s}
      zur"uck (Null-Zeichen am Ende wird nicht mitgez"ahlt).
   \item {\tt int strcmp(char *s1, char *s2)}\\
      Vergleicht zwei Strings. Liefert einen Wert gr"o"ser als, gleich, oder
      kleiner als 0 zur"uck, abh"angig davon, ob {\tt s1} lexikographisch
      (d.h.~ASCII-Zeichenwert) gr"o"ser, gleich, oder
      kleiner als {\tt s2} ist.
   \item {\tt int strncmp(char *s1, char *s2, int n)}\\
      Wie {\tt strcmp}, vergleicht maximal {\tt n} Zeichen.
   \item {\tt int fprintf(stderr,"Text"...)}\\
      Gibt den Text auf der Standardfehlerausgabe aus.
   \item {\tt int isdigit(char c)}\\
      Liefert {\em true\/}, wenn das Zeichen {\tt c} eine Ziffer ist,
      ansonsten {\em false\/}.
   \item {\tt int atoi(char *str)}\\
      Wandelt den String {\tt str} in einen Integer um. F"uhrende
      "`white-spaces"' werden ignoriert, das scannen das Strings wird mit dem
      ersten Zeichen, das keine Ziffer ist, abgebrochen. Konnte keine Ziffer
      gelesen werden, wird 0 zur"uckgegeben.
\end{itemize}

