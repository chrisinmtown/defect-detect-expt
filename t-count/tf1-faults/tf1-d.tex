
In Klammern ist die Kategorisierung des Fehlers angegeben.

\begin{enumerate}


\item Fehler in Zeile 10: 
Die Variablen m\"ussen mit 0 statt 1 initialisiert werden.

\{Comission, Initialisierng\} 

F\"uhrt zum Fehlverhalten:
Die Summen sind alle eins zu gro{\ss}.

\item Fehler in Zeile 14: Die Variable ``fp'' wird nicht initialisiert f\"ur den
Fall, da{\ss} die Eingabe von ``stdin'' gelesen werden soll.

\{Omission, Initialisierung\} 

F\"uhrt zum Fehlverhalten:
`count' kann nicht von der Standard-Eingabe (stdin) lesen.


\item Fehler in Zeile 15: Statt ``stderr'' wird in der fprintf-Aufruf
``stdout'' verwendet.

\{Comission, Interface\} 

F\"uhrt zum Fehlverhalten:
Fehlermeldungen erscheinen auf der Standard-Ausgabe (stdout) 
statt auf der Standardfehler-Ausgabe (stderr).


\item Fehler in Zeile 16: Komponente wird mit `exit (1)'
verlassen. Richtig w\"are hier die Verwendung des C-statements
`continue' gewesen, welches die n\"achste Iteration einer umgebenden
Schleife bewirkt.  

\{Comission, Kontrolle\}

F\"uhrt zum Fehlverhalten:
Ist ein als Parameter spezifiziertes File nicht im Dateisystem
vorhanden, so bricht das Programm ab.  Weitere Parameter werden nicht
ber\"ucksichtigt. Ebenso erfolgt keine Ausgabe der Summe.


\item Fehler in Zeile 19: Die Variable ``inword'' wird mit 1
initialisiert statt 0.

 \{Comission, Initialisierung\}

F\"uhrt zum Fehlverhalten:
`count' bestimmt bei n W\"ortern in einer Datei manchmal n,
manchmal n + 1 (abhaengig davon ob der erste Zeichen ein Leerzeichen
ist).  


\item Fehler in Zeile 34: *argv wird statt argv[i] verwendet.

\{Comission, Daten\}

F\"uhrt zum Fehlverhalten:
In der Ausgabe wird statt der Dateiname immer der Name des
Programmes ausgedruckt. 


\item Fehler in Zeile 41: Argc wird mit 1 verglichen; es mu{\ss} aber mit 2
verglichen werden.

\{Comission, Berechnung\}

F\"uhrt zum Fehlverhalten:
Das Programm druckt auch eine Zeile mit ``Total'' aus in den
F\"allen worin nur eine Datei gelesen wurde.


\item Fehler in Zeile 42: Statt ``tlinect'' wird lediglich ``linect''
verwendet. 

\{Comission, Daten\}

F\"uhrt zum Fehlverhalten:
Die Summen werden nicht korrekt ausgedruckt.
\begin{verbatim}
% ./count file2 file2 file2
      1       2      14 ./count
      1       2      14 ./count
      1       2      14 ./count
      1       7      43 total
\end{verbatim}



\end{enumerate}
