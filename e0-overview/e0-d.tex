\paragraph{Ziel:} Die Testtechniken "`code-reading"', 
funktionales und strukturelles Testen bez"uglich Effektivit"at, Effizienz
und erfa"sbaren Fehlerklassen experimentell zu vergleichen

\vspace*{5mm}
\centerline{\bf Diese Aufgabe wird in Einzelarbeit durchgef"uhrt.}

\paragraph{"Uberblick:} Diese Aufgabe besteht aus zwei Teilen, einer
	Einarbeitung in die Testtechniken und einem Experiment. Im 
	Gegensatz zum Experiment aus der SE-I Vorlesung sollen Sie 
	in diesem Experiment {\em alle drei Testtechniken\/}
	auf drei Programmausschnitte aus dem MWZ anwenden. Das Experiment 
	wird an drei Tagen in der n"achsten Woche durchgef"uhrt. Die Zeitplanung:
\begin{itemize}
	\item Einarbeitung in die Testtechniken: Montag, 30.5. -- Mittwoch 1.6.
	\item Vorbesprechung zum Experiment: Mittwoch, 1.6., $17^{30}$ im Plenum
	\item Experiment mit Programm 1: Dienstag, 7.6., $15^{30}$
	\item Experiment mit Programm 2: Mittwoch, 8.6., $15^{30}$
	\item Experiment mit Programm 3: Donnerstag, 9.6., $15^{30}$
	\item Nachbesprechung: Mittwoch, 15.6., $17^{30}$ im Plenum 
\end{itemize}
Die Termine f"ur das Experiment sind Vorschl"age, "uber die in der 
Vorbesprechung diskutiert werden kann.

Es folgt die Aufgabenbeschreibung f"ur den ersten Teil der Aufgabe 
(Einarbeitung), die f"ur den
zweiten Teil (Experiment) erhalten Sie am Montag, den 6. Juni.
\pagebreak

\section{Teil 1: Einarbeitung}

\paragraph{Ziel:} Durch die Einarbeitung sollen Sie Ihre Kenntnisse "uber 
die Testtechniken aus der SE-I Vorlesung und dem Experiment in der 
dazugeh"origen "Ubung vertiefen. 

\paragraph{Der Teilnehmer erh"alt:}  
\begin{itemize}
	\item Fehlerklassifikationsschema (E1) im Anhang
	\item Hinweise auf "`C"' Spezialit"aten (E2) im Anhang
\end{itemize}

\paragraph{Durchf"uhrung:} 
Lesen Sie sich zun"achst die allgemeinen Hinweise (E1, E2) durch.
Wenden Sie dann die einzelnen Testtechniken an, in welcher Reihenfolge
und wann Sie das tun, bleibt Ihnen "uberlassen (im Rahmen des Zeitplans) ! 
{\bf Bitte unterhalten Sie
sich nicht mit Ihren Kommilitonen, die die Einarbeitung noch nicht gemacht 
haben, "uber die Ergebnisse !} Die Einarbeitung wird nicht bewertet.
\bigskip

F"ur jede Testtechnik ist auf den folgenden Seiten --zur "Ubersicht--
ein Proze"smodell angegeben, eine detailierte Anleitung gibt es 
zus"atzlich im Anhang. Grunds"atzlich wird jede Testtechnik in vier
Schritten ausgef"uhrt:
\begin{enumerate}
	\item Testdaten erzeugen, code reading: Kode lesen
	\item Tests durchf"uhren (code reading: entf"allt)
	\item Fehlverhalten erkennen (code reading: Inkonsistenzen zwischen 
		Spezifikation und Abstraktion erkennen)
	\item Fehler lokalisieren
\end{enumerate}

Die Durchf"uhrung der Einarbeitung entspricht im Wesentlichen 
dem Experiment in der n"achsten Woche; wie auch dort sollen Sie hier alle 
drei Testtechniken auf drei
Programme anwenden (eine Testtechnik auf ein Programm). Unterschiede 
bestehen im Umfang der Testprogramme (deutlich geringer) und
in der freien Zeiteinteilung.

\paragraph{Bewertung:} 
	Teilnahme an Einarbeitung und Experiment (20 Punkte)

\paragraph{Plenum:} Mittwoch, 15. Juni, 17.30

%------------------------------------------------------------------------------
\newpage
\section*{Code reading}

\paragraph{Der Teilnehmer erh"alt:}  
\begin{itemize}
	\item Anleitung "`Code reading"' (= {\bf E10}) im Anhang
	\item Quellkode eines Programms (= {\bf E11}) s.o.
	\item Fragebogen "`Code reading"' (= {\bf E12}) s.o.
	\item Formblatt f"ur Inkonsistenzen (= {\bf E13}) s.o.
	\item Formblatt f"ur Fehler (= {\bf E14}) s.o.
	\item Spezifikation (= {\bf E15}) sp"ater (siehe Beschreibung unten)
\end{itemize}

\paragraph{Durchf"uhrung:} 
\begin{enumerate}
	\item Lesen
	\begin{itemize}
		\item Eingabe: Quellkode (E11), Fragebogen (E12)
		\item Aktivit"at: Lesen des Kodes nach der Methode "`code-reading by 
			stepwise abstraction"'
		\item Ausgabe: Notizen mit den Abstraktionen
	\end{itemize}
	Geben Sie eine Kopie Ihrer Notizen beim Betreuer ab und nehmen Sie die 
	Spezifikation (E15) entgegen
	\item entf"allt
	\item Entdecken von Inkonsistenzen zw. Spezifikation und abstrahierter 
		Funktion
	\begin{itemize}
		\item Eingabe: Quellkode (E11), Spezifikation (E15), Fragebogen (E12),
			Formblatt f"ur Inkonsistenzen (E13)
		\item Aktivit"at: Vergleich der abstrahierten Funktion mit der
			Spezifikation 
		\item Ausgabe: Formblatt mit erkannten Inkonsistenzen (= {\bf A13})
	\end{itemize}
	\item Lokalisieren von Fehlern
	\begin{itemize}
		\item Eingabe: Quellkode (E11), Spezifikation (E15), Fragebogen (E12),
			Formblatt f"ur Fehler (E14)
		\item Aktivit"at: Isolieren von Fehlern
		\item Ausgabe: Formblatt mit erkannten Fehlern (= {\bf A14}), 
			vollst"andig ausgef"ullter Fragebogen (= {\bf A12})
	\end{itemize}
\end{enumerate}

\paragraph{Der Teilnehmer gibt ab:}  
\begin{itemize}
	\item ausgef"ullten Fragebogen (A12)
	\item ausgef"ullte Formbl"atter  (A13, A14)
\end{itemize}

%------------------------------------------------------------------------------
\newpage
\section*{Funktionales Testen}

\paragraph{Der Teilnehmer erh"alt:}  
\begin{itemize}
	\item Anleitung "`Funktionales Testen"' (= {\bf E20}) im Anhang
	\item Spezifikation von Komponente "`series"' (= {\bf E21}) s.o.
	\item Fragebogen "`Funktionales Testen"' (= {\bf E22}) s.o.
	\item Formblatt f"ur Fehlverhalten (= {\bf E23}) s.o.
	\item Formblatt f"ur Fehler (= {\bf E24}) s.o.
	\item Dateipaket "`Funktionales Testen"' (= {\bf E25}) unter dem 
		Account "`prakt00"',
		Verzeichnis "`Aufgabe5"', Datei "`funk-test-einarb.tar"'
	\item Quellkode (= {\bf E26}) sp"ater (siehe Beschreibung unten)
\end{itemize}

\paragraph{Durchf"uhrung:} 
\begin{enumerate}
	\item Testdaten erzeugen
	\begin{itemize}
		\item Eingabe: Spezifikation (E21), Fragebogen (E22)
		\item Aktivit"at: Testdaten "Aquivalenzklassen- und Grenzwert-basiert
			ableiten
		\item Ausgabe: Testdaten
	\end{itemize}
	\item Tests Durchf"uhren
	\begin{itemize}
		\item Eingabe: ausf"uhrbares Programm (E25), Testdaten, Fragebogen (E22)
		\item Aktivit"at: Test durchf"uhren
		\item Ausgabe: Testergebnisse
	\end{itemize}
	Loggen Sie sich aus, verlassen sie den Rechnerraum.
	\item Entdecken von Fehlverhalten
	\begin{itemize}
		\item Eingabe: Spezifikation (E21), Testergebnisse, Fragebogen (E22), 
			Formblatt f"ur Fehlverhalten (E23)
		\item Aktivit"at: Fehlverhalten diagnostizieren
		\item Ausgabe: Formblatt mit erkannten Fehlverhalten (= {\bf A23})
	\end{itemize}
	Geben Sie einen Ausdruck Ihrer Testergebnisse 
	beim Betreuer ab und nehmen Sie den Quellkode (E26) entgegen
	\item Lokalisierung von Fehlern
	\begin{itemize}
		\item Eingabe: Spezifikation (E21), Quellkode (E26), 
			Fragebogen (E22), Formblatt f"ur Fehler (E24)
		\item Aktivit"at: Lokalisieren von Fehlern
		\item Ausgabe: Formblatt mit erkannten Fehlern (= {\bf A24}), 
			vollst"andig ausgef"ullter Fragebogen (= {\bf A22})
	\end{itemize}
\end{enumerate}

\paragraph{Der Teilnehmer gibt ab:}  
\begin{itemize}
	\item ausgef"ullten Fragebogen (A22)
	\item ausgef"ulltes Formbl"atter Fehlverhalten/Fehler (A23, A24)
\end{itemize}

%------------------------------------------------------------------------------
\newpage
\section*{Strukturelles Testen}

\paragraph{Der Teilnehmer erh"alt:}  
\begin{itemize}
	\item Anleitung "`Strukturelles Testen"' (= {\bf E30}) im Anhang
	\item Quellkode (= {\bf E31}) s.o.
	\item Fragebogen "`Strukturelles Testen"' (= {\bf E32}) s.o.
	\item Formblatt f"ur Fehlverhalten (= {\bf E33}) s.o.
	\item Formblatt f"ur Fehler (= {\bf E34}) s.o.
	\item Dateipaket "`Strukturelles Testen"' (= {\bf E35}) unter dem 
		Account "`prakt00"', Verzeichnis "`Aufgabe5"', Datei 
		"`strukt-test-einarb.tar"'
	\item Spezifikation (= {\bf E36}) sp"ater (siehe Beschreibung unten)
\end{itemize}

\paragraph{Durchf"uhrung:} 
\begin{enumerate}
	\item Testdaten erzeugen
	\begin{itemize}
		\item Eingabe: Quellkode (E31), Fragebogen (E32)
		\item Aktivit"at: Testdaten erzeugen
		\item Ausgabe: Testdaten
	\end{itemize}
	\item Tests durchf"uhren
	\begin{itemize}
		\item Eingabe: ausf"uhrbares Programm (E35), Testdaten, Fragebogen (E32)
		\item Aktivit"at: Kontrollkanten"uberdeckung mit dem Coverage-Tool GCT 
			bestimmen, Tests durchf"uhren
		\item Ausgabe: Testergebnisse
	\end{itemize}
	Loggen Sie sich aus, verlassen sie den Rechnerraum.
	Geben Sie einen Ausdruck Ihrer Testergebnisse beim Betreuer ab und nehmen 
	Sie die Spezifikation (E36) entgegen
	\item Entdecken von Fehlverhalten
	\begin{itemize}
		\item Eingabe: Spezifikation (E36), Quellkode (E31), 
			Testergebnisse, Fragebogen (E32), Formblatt f"ur Fehlverhalten (E33)
		\item Aktivit"at: Fehlverhalten diagnostizieren
		\item Ausgabe: Formblatt mit erkannten Fehlverhalten (= {\bf A33})
	\end{itemize}
	\item Lokalisierung von Fehlern
	\begin{itemize}
		\item Eingabe: Spezifikation (E36), Quellkode (E31), 
			Fragebogen (E32), Formblatt f"ur Fehler (E34)
		\item Aktivit"at: Lokalisieren von Fehlern
		\item Ausgabe: Formblatt mit erkannten Fehlern (= {\bf A34}), 
			ausgef"ullter Fragebogen (= {\bf A32})
	\end{itemize}
\end{enumerate}

\paragraph{Der Teilnehmer gibt ab:}  
\begin{itemize}
	\item ausgef"ullten Fragebogen (A32)
	\item ausgef"ulltes Formbl"atter Fehlverhalten/Fehler (A33, A34)
\end{itemize}


\newpage
\section*{Teil 2: Experiment}

\paragraph{Durchf"uhrung:} 
Das Experiment wird an drei Tagen durchgef"uhrt.
Genauere Information zu den Terminen und zur Aufgabeverteilung
befindet sich auf der n\"achsten Seite. 
Ein Treffen zwecks Vorbesprechung und Austeilung des Materials findet zu den
angegebenen Zeiten im Seminarraum 57/528 statt. Planen Sie einen Zeitbedarf
von ca.~3 Stunden pro Tag ein. {\bf Bitte seien Sie p"unktlich !}

\begin{quote}
{\bf Wenn Sie zu den angegebenen Terminen absolut keine Zeit haben, dann
melden Sie sich bitte umgehend bei Ihrem Betreuer.}
\end{quote}

\bigskip

Eine Nachbesprechung findet Mittwoch, 15.6., $17^{30}$ im Plenum statt.

\bigskip

Die Durchf"uhrung des Experiments entspricht ansonsten der der Einarbeitung,
Unterschiede bestehen nur im Material, da"s ausgeteilt wird.
F"ur jede Testtechnik ist auf den folgenden Seiten --zur "Ubersicht und
der Vollst"andigkeit halber-- nochmals das Proze"smodell angegeben.
Grunds"atzlich wird jede Testtechnik in vier
Schritten ausgef"uhrt:
\begin{enumerate}
	\item Testdaten erzeugen. Bei code reading: Kode lesen.
	\item Tests durchf"uhren. Bei code reading: entf"allt.
	\item Fehlverhalten erkennen. Bei code reading: Inkonsistenzen
			zwischen Spezifikation und Abstraktion erkennen.
	\item Fehler lokalisieren
\end{enumerate}

% \paragraph{Plenum:} Mittwoch, 15. Juni, 17.30

%------------------------------------------------------------------------------
\newpage
%\section{Durchf\"uhrungsplan des Experiments}

\subsection*{Termine und Aufgabenverteilung}

\begin{center}
\begin{tabular}{|r|c|l|l|}
\hline
\bf Termin    & \bf Gruppe  & \bf Aufgabe & Programm\\
\hline
Dienstag, 7. Juni         &  1          &  Strukturelles Testen  &  ntree\\
$10^{00}$ oder $15^{00}$  &  2          &  Funktionales  Testen  & \\
                          &  3          &  code reading          & \\
\hline
Mittwoch, 8. Juni         &  2          &  Strukturelles Testen  & cmdline \\
$15^{00}$                 &  3          &  Funktionales  Testen  & \\
                          &  1          &  code reading          & \\
\hline
Donnerstag, 9.Juni        &  3          &  Strukturelles Testen  & nametbl \\
$14^{00}$                 &  1          &  Funktionales  Testen  & \\
                          &  2          &  code reading          & \\
\hline
\end{tabular}
\end{center}


\subsection*{Gruppeneinteilung}

\begin{center}

\begin{tabular}{|l|c|}
\hline
\bf Kennung   &  \bf Gruppe \\
\hline
prakt01       &      1      \\
prakt02       &      3      \\
prakt03       &      2      \\
prakt04       &      1      \\
prakt06       &      3      \\
\hline
prakt07       &      2      \\
prakt08       &      2      \\
prakt09       &      1      \\
prakt10       &      1      \\
prakt11       &      3      \\
\hline
prakt12       &      3      \\
prakt13       &      1      \\
prakt15       &      3      \\
prakt17       &      3      \\
prakt18       &      2      \\
\hline
prakt19       &      1      \\
prakt20       &      3      \\
prakt21       &      2      \\
prakt22       &      1      \\
prakt23       &      2      \\
\hline
prakt24       &      1      \\
prakt25       &      1      \\
prakt26       &      2      \\
prakt27       &      2      \\
prakt28       &      1      \\
\hline
prakt29       &      2      \\
prakt30       &      3      \\
prakt31       &      3      \\
\hline
\end{tabular}

\end{center}

%------------------------------------------------------------------------------
\newpage
\section*{Code reading}

\paragraph{Der Teilnehmer erh"alt:}  
\begin{itemize}
	\item Anleitung "`Code reading"' (= {\bf E10}) zu Beginn des Experiments
	\item Quellkode der Komponente (= {\bf EQ$x$}) s.o.
	\item Fragebogen "`Code reading"' (= {\bf E12}) s.o.
	\item Formblatt f"ur Inkonsistenzen (= {\bf E13}) s.o.
	\item Formblatt f"ur Fehler (= {\bf E14}) s.o.
	\item Spezifikation der Komponente (= {\bf ES$x$}) im Verlauf des Experiments
				(siehe Beschreibung unten) 
\end{itemize}

\paragraph{Durchf"uhrung:} 
\begin{enumerate}
	\item Lesen
	\begin{itemize}
		\item Eingabe: Quellkode (EQ$x$), Fragebogen (E12)
		\item Aktivit"at: Lesen des Kodes nach der Methode "`code-reading by 
			stepwise abstraction"'
		\item Ausgabe: Notizen mit den Abstraktionen
	\end{itemize}
	Geben Sie eine Kopie Ihrer Notizen beim Betreuer ab und nehmen Sie die 
	Spezifikation (ES$x$) entgegen
	\item entf"allt
	\item Entdecken von Inkonsistenzen zw. Spezifikation und abstrahierter 
		Funktion
	\begin{itemize}
		\item Eingabe: Quellkode (EQ$x$), Spezifikation (ES$x$), Fragebogen (E12),
			Formblatt f"ur Inkonsistenzen (E13)
		\item Aktivit"at: Vergleich der abstrahierten Funktion mit der
			Spezifikation 
		\item Ausgabe: Formblatt mit erkannten Inkonsistenzen (= {\bf A13})
	\end{itemize}
	\item Lokalisieren von Fehlern
	\begin{itemize}
		\item Eingabe: Quellkode (EQ$x$), Spezifikation (ES$x$), Fragebogen (E12),
			Formblatt f"ur Fehler (E14)
		\item Aktivit"at: Isolieren von Fehlern
		\item Ausgabe: Formblatt mit erk.~Fehlern (= {\bf A14}), 
			vollst.~ausgef"ullter Fragebogen (= {\bf A12})
	\end{itemize}
\end{enumerate}

\paragraph{Der Teilnehmer gibt ab:}  
\begin{itemize}
	\item ausgef"ullten Fragebogen ({\bf A12})
	\item ausgef"ullte Formbl"atter  ({\bf A13, A14})
\end{itemize}

%------------------------------------------------------------------------------
\newpage
\section*{Funktionales Testen}

\paragraph{Der Teilnehmer erh"alt:}  
\begin{itemize}
	\item Erg"anzungsblatt "`Funktionales Testen des Programms $x$"' zu Beginn 
		des Experiments
	\item Anleitung "`Funktionales Testen"' (= {\bf E20}) s.o.
	\item Spezifikation der Komponente (= {\bf ES$x$});
		siehe Erg\"anzungsblatt.
	\item Fragebogen "`Funktionales Testen"' (= {\bf E22}) s.o.
	\item Formblatt f"ur Fehlverhalten (= {\bf E23}) s.o.
	\item Formblatt f"ur Fehler (= {\bf E24}) s.o.
	\item Dateipaket "`Funktionales Testen"'; siehe Erg\"anzungsblatt.
	\item Quellkode (= {\bf EQ$x$}) im Verlauf des Experiments
		(siehe Beschreibung unten und auch Erg\"anzungsblatt).
\end{itemize}

\paragraph{Durchf"uhrung:} 
\begin{enumerate}
	\item Testdaten erzeugen
	\begin{itemize}
		\item Eingabe: Spezifikation (ES$x$), Fragebogen (E22)
		\item Aktivit"at: Testdaten "Aquivalenzklassen- und Grenzwert-basiert
			ableiten
		\item Ausgabe: Testdaten
	\end{itemize}
	\item Tests Durchf"uhren
	\begin{itemize}
		\item Eingabe: ausf"uhrbares Programm, Testdaten, Fragebogen (E22)
		\item Aktivit"at: Test durchf"uhren
		\item Ausgabe: Testergebnisse
	\end{itemize}
	Loggen Sie sich aus, verlassen sie den Rechnerraum.
	\item Entdecken von Fehlverhalten
	\begin{itemize}
		\item Eingabe: Testergeb., Spez.~(ES$x$), Fragebogen (E22), 
			Formblatt f"ur Fehlverhalten (E23)
		\item Aktivit"at: Fehlverhalten diagnostizieren
		\item Ausgabe: Formblatt mit erkannten Fehlverhalten (= {\bf A23})
	\end{itemize}
	Geben Sie einen Ausdruck Ihrer Testergebnisse 
	beim Betreuer ab und nehmen Sie den Quellkode (EQ$x$) entgegen
	\item Lokalisierung von Fehlern
	\begin{itemize}
		\item Eingabe: Spez.~(ES$x$), Quellkode (EQ$x$), 
			Fragebogen (E22), Formblatt f"ur Fehler~(E24)
		\item Aktivit"at: Lokalisieren von Fehlern
		\item Ausgabe: Formblatt mit erk.~Fehlern (= {\bf A24}), 
			volls.~ausgef"ullter Fragebogen (= {\bf A22})
	\end{itemize}
\end{enumerate}

\paragraph{Der Teilnehmer gibt ab:}  
\begin{itemize}
	\item ausgef"ullten Fragebogen ({\bf A22})
	\item ausgef"ulltes Formbl"atter Fehlverhalten/Fehler ({\bf A23, A24})
\end{itemize}

%------------------------------------------------------------------------------
\newpage
\section*{Strukturelles Testen}

\paragraph{Der Teilnehmer erh"alt:}  
\begin{itemize}
	\item Erg"anzungsblatt "`Strukturelles Testen des Programms $x$"' zu Beginn 
		des Experiments
	\item Anleitung "`Strukturelles Testen"' (= {\bf E30}) s.o.
	\item Quellkode (= {\bf EQ$x$}); siehe Erg\"anzungsblatt.
	\item Fragebogen "`Strukturelles Testen"' (= {\bf E32}) s.o.
	\item Formblatt f"ur Fehlverhalten (= {\bf E33}) s.o.
	\item Formblatt f"ur Fehler (= {\bf E34}) s.o.
	\item Dateipaket "`Strukturelles Testen"'; siehe Erg\"anzungsblatt.
	\item Spezifikation (= {\bf ES$x$}) im Verlauf des Experiments (siehe 
		Beschreibung unten und auch Erg\"anzungsblatt)
\end{itemize}

\paragraph{Durchf"uhrung:} 
\begin{enumerate}
	\item Testdaten erzeugen
	\begin{itemize}
		\item Eingabe: Quellkode (EQ$x$), Fragebogen (E32)
		\item Aktivit"at: Testdaten erzeugen
		\item Ausgabe: Testdaten
	\end{itemize}
	\item Tests durchf"uhren
	\begin{itemize}
		\item Eingabe: ausf"uhrbares Programm, Testdaten, Fragebogen (E32)
		\item Aktivit"at: Kontrollkanten"uberdeckung mit dem Coverage-Tool GCT 
			bestimmen, Tests durchf"uhren
		\item Ausgabe: Testergebnisse
	\end{itemize}
	Loggen Sie sich aus, verlassen sie den Rechnerraum.
	Geben Sie einen Ausdruck Ihrer Testergebnisse beim Betreuer ab und nehmen 
	Sie die Spezifikation (E36) entgegen
	\item Entdecken von Fehlverhalten
	\begin{itemize}
		\item Eingabe: Testergeb., Spez.~(ES$x$), Quellkode (EQ$x$), 
			Fragebogen (E32), Formblatt f"ur Fehlverhalten (E33)
		\item Aktivit"at: Fehlverhalten diagnostizieren
		\item Ausgabe: Formblatt mit erkannten Fehlverhalten (= {\bf A33})
	\end{itemize}
	\item Lokalisierung von Fehlern
	\begin{itemize}
		\item Eingabe: Spez.~(ES$x$), Quellkode (EQ$x$), 
			Fragebogen (E32), Formblatt f"ur Fehler~(E34)
		\item Aktivit"at: Lokalisieren von Fehlern
		\item Ausgabe: Formblatt mit erk.~Fehlern (= {\bf A34}), 
			volls.~ausgef"ullter Fragebogen (= {\bf A32})
	\end{itemize}
\end{enumerate}

\paragraph{Der Teilnehmer gibt ab:}  
\begin{itemize}
	\item ausgef"ullten Fragebogen ({\bf A32})
	\item ausgef"ulltes Formbl"atter Fehlverhalten/Fehler ({\bf A33, A34})
\end{itemize}

%------------------------------------------------------------------------------

\newpage

\section*{Klassen von Fehlern}

Unterscheidung ob etwas vergessen wurde oder Geschriebenes falsch ist.
\begin{itemize}

\item Omission (O)

Es wurde etwas vergessen (z.B. eine Anweisung, die
Initialissierung einer Variablen, ein Parameter beim Aufruf einer
Routine, ...).

\item Comission (C)

Das Programmst\"uck ist falsch eingetippt (z.B. `+'
statt `-', falscher Initialisierungswert, Bedingung deckt die falschen
F\"alle ab, ...)

\end{itemize}

Weiterhin kann man Fehler dahingehend klassifizieren, welche
semantischen Teile einer Komponente betroffen sind.

\begin{itemize}

\item Initialisierung (I)

Der Wert eines Bezeichners ist beim ersten
Zugriff anders als erwartet (z.B.  Beginn eines Arrays bei 1 statt bei
0).

\item Berechnung (B)

Der Wert eines Ausdrucks wird falsch berechnet
(z.B. `+' statt `-' verwendet, Konstantenwert ist falsch, ...).

\item Kontrolle (K)

Der Kontrollpfad wird nicht wie erwartet begangen
(z.B. Bedingung in IF- Anweisung falsch, Schleifen-Invariante negiert,
...)

\item Schnittstelle (S)

Die Schnittstelle eines Elementes oder Moduls wurde
falsch verstanden (z.B. es wird erwartet, da{\ss} alle Stellen eines
array of char nach abschliessenden `$\backslash$0' mit Leerzeichen aufgef\"ullt
sind; Verhalten eines ADTs ist falsch verstanden worden; ...)

\item Daten (D)

Der Gebrauch von Daten ist falsch (z.B. falsche Berechnung
des Zugriffs auf ein Array-Element, in verketteter Liste wird Element
\"ubersprungen, ...)

\item Kosmetik (C, f"ur "`cosmetic"')

Pr\"asentation der Komponente gegen\"uber dem Benutzer
ist nicht korrekt (z.B.  falsch geschriebenes Wort in Fehlermeldung,
\"Uberschreiben von alten Bildschirminhalten in Window-Systemen, ...).

\end{itemize}

Die Einordnung in die Klassen ist jedoch meistens nicht eindeutig
durchf\"uhrbar. Sie unterliegt stark der Subjektivit\"at des
Testers.  
