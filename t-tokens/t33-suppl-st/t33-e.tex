
This is the supplemental sheet 
for applying technique ``structural testing''
to the program ``tokens.'' 



\subsection*{Necessary inputs}

\begin{itemize}

\item Which documents belong to this exercise?

\begin{enumerate}
\item Document ES7, the specification of the component, which you will
receive after you have created test cases and attempted to reach 100\%
coverage.
\item Document EQ7, the source code of the component
\end{enumerate}

\item How do I fetch the files which I need?

Do the following:

\begin{enumerate}

\item First create a new directory for this exercise with the
``mkdir'' command.
\begin{verbatim}
    mkdir st-tokens
\end{verbatim}

\item Then change to the new directory with the ``cd'' command.
\begin{verbatim}
    cd st-tokens
\end{verbatim}

\item Finally, enter the following command:
\begin{path}
\begin{verbatim}
    tar xf ~prakt00/Exercise5/st-tokens.tar
\end{verbatim}
\end{path}

\end{enumerate}

\item What should I have?

The following files must be available:
\begin{verbatim}
Makefile     gct-map      run-suite    test-dir    tokens
\end{verbatim}

\end{itemize}


\subsubsection*{Writing test cases}

All of the program's functions are fundamentally tested via the
invocation 
\begin{verbatim}
tokens 
\end{verbatim}
The programs reads its input from the standard input (stdin) and
can be influences using options.
A parameter file holds the options, the name of the input file plus
the necessary symbol to direct the contents of the file to the 
program (``<''). 
This is somewhat awkward, but necessary to allow applying 
the test driver consistently.
The format of the parameter file is as follows:
\begin{verbatim}
-Option1 -Option2 < inputfile
\end{verbatim}

Which options are understood, what input is allowed, and which
functions are triggered by that input can be seen in the functions 
in {\tt tokens.c}. 
