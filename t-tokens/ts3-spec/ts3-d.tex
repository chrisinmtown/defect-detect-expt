
\medskip
\textbf{\large Name}

\textbf{tokens} -- alphanumerische W\"orter sortieren und z\"ahlen

\medskip
\textbf{\large Verwendung}

\textbf{tokens} [ --ai ] [ --c Zeichen ] [ --m Zahl ]


\medskip
\textbf{\large Beschreibung}

\textbf{tokens} liest W\"orter von der Standardeingabe (``stdin'').
Ein Wort ist eine Folge von aufeinanderfolgenden Zeichen aus einem
eingeschr\"ankten Zeichensatz.
In diesem Zeichensatz sind sowohl die alphabetischen Zeichen
\mbox{A--Z} und \mbox{a--z} als auch die numerischen Zeichen
\mbox{0--9} erlaubt.  
Abh\"angig von den Optionen (siehe unten) k\"onnen die numerische
Zeichen entfernt bzw.\ andere Zeichen hinzugef\"ugt werden.
Unerlaubte Zeichen trennen W\"orter voneinander, werden aber 
ansonsten ignoriert.
Am Ende der Eingabe wird jedes Wort mit der H\"aufigkeit seines
Auftretens genau einmal ausgegeben.  Dabei sind die W\"orter  
in aufsteigender lexikographischer Ordnung sortiert.


\medskip
\textbf{\large Optionen}

\begin{itemize}

\item ``--a'': Ein Wort darf nur die alphabetischen Zeichen \mbox{A--Z}
und \mbox{a--z} beinhalten (keine Ziffer 0--9).

\item ``--c Zeichen'': Ein Wort darf zus\"atzlich die angegebenen
Zeichen beinhalten.

\item ``--i'':  Der Unterschied zwischen Gro{\ss}buchstaben und
Kleinbuchstaben wird ignoriert.  Alle Gro{\ss}buchstaben werden zu
Kleichbuchstaben umgewandelt.

\item ``--m Zahl'':  Das Programm gibt nur W\"orter aus, deren
H\"aufigkeit in der Eingabe gr\"o{\ss}er oder gleich dieser Zahl ist.

\end{itemize}



\medskip
\textbf{\large Beispiel}

Um alle alphabetischen W\"orter in Datei ``xyz'' zu sortieren und
z\"ahlen: 
{\small
\begin{verbatim}
tokens -a < xyz
\end{verbatim}
}


\medskip
\textbf{\large Einschr\"ankungen}

Die maximale Anzahl von verschiedenen W\"orter ist durch den Speicher
des Rechners begrenzt.
Die maximale Anzahl des Auftretens eines Worts ist durch
die L\"ange eines ``long integer'' begrenzt.


\medskip
\textbf{\large Autor}

Gary Perlman 

