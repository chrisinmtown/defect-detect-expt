%
% sample solution for exercise tokens
%

The following test cases attain the maximum coverage possible for the
program, given the faults described in the next section.

\begin{enumerate}
\item Erroneous flag

\begin{description}
\item[Command line:] -q
\item[Input:] none.
\item[Expected output:] Usage message for the program.
\end{description}

\item Empty file

\begin{description}
\item[Command line:] < /dev/null
\item[Input:] none.
\item[Expected output:] none.
\end{description}

\item Short file, few options

\begin{description}
\item[Command line:] -c /. < shortfile
\item[Input:] A file named ``shortfile'' with the single word ``nonl''
and no other characters before the end of file; i.e., no newline.
\item[Expected output:]  The following single line:

	\begin{quote}
	1 nonl\\
	\end{quote}
\end{description}

\item Acceptable input, many options

\begin{description}
\item[Command line:] -a -i -c / -m 1 < infile
\item[Input:] A file named ``infile'' with these lines:

	\begin{quote}
	word1\\
	word1\\
	word1\\
	word2\\
	word2\\
	$*$\\
	b\\
	a\\
	\end{quote}

\item[Expected output:] The following counts:

	\begin{quote}
	1 a\\
	1 b\\
	3 word1\\
	2 word2\\
	\end{quote}
\end{description}
\end{enumerate}
End of test cases.
\newpage
\noindent
After running the test cases shown above, the following coverage
values are attained:
{\small
\begin{verbatim}
% gsummary GCTLOG 
BINARY BRANCH INSTRUMENTATION (38 conditions total)
2 ( 5.26%) not satisfied.
36 (94.74%) fully satisfied.

SWITCH INSTRUMENTATION (5 conditions total)
0 ( 0.00%) not satisfied.
5 (100.00%) fully satisfied.

LOOP INSTRUMENTATION (21 conditions total)
8 (38.10%) not satisfied.
13 (61.90%) fully satisfied.

OPERATOR INSTRUMENTATION (15 conditions total)
0 ( 0.00%) not satisfied.
15 (100.00%) fully satisfied.

SUMMARY OF ALL CONDITION TYPES (79 total)
10 (12.66%) not satisfied.
69 (87.34%) fully satisfied.
\end{verbatim}
}

\noindent
The values that are less than 100\% are explained by invoking the
``greport'' command, which produces the following output:
{\small
\begin{verbatim}
% greport GCTLOG 
"tokens.c", line 38: if was taken TRUE 5, FALSE 0 times.
"tokens.c", line 64: loop zero times: 0, one time: 3, many times: 8.
"tokens.c", line 103: loop zero times: 0, one time: 1, many times: 1.
"tokens.c", line 118: loop zero times: 0, one time: 0, many times: 3.
"tokens.c", line 120: loop zero times: 0, one time: 0, many times: 3.
"tokens.c", line 122: if was taken TRUE 3, FALSE 0 times.
"tokens.c", line 123: loop zero times: 0, one time: 0, many times: 3.
\end{verbatim}
}

\noindent
These lines cannot be covered more thoroughly for the following reasons:

\begin{description}
\item[Line 38:] This ``if'' statement checks for a failure of calloc to
allocate memory.  It is difficult to make calloc fail, especially for
students who do not understand the ``limit'' commands supported by
various shells.  With perseverance, this might be made to fail.

\item[Line 64:] This is a ``forever'' loop (empty termination
criteria) so it will always be done at least once.

\item[Line 103:] This ``while'' loop is only begun if there is one or
more characters for it to process, so it will always be done at least
once.

\item[Lines 118, 120, and 123:]  These ``for'' loops have hardcoded,
fixed limits that are larger than 1, and therefore will always be
executed many times.

\item[Line 122:] This ``if'' statement is always true due to a bug in
the program (the ``Alpha'' variable is never set).

\end{description}

