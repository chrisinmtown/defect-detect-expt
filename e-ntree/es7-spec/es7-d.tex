
\subsection*{Name}

ntree -- Funktionen zur Verwaltung eines Baumes

\subsection*{Verwendung}

{\bf ntree} Eingabedatei

\subsection*{Beschreibung}

{\it ntree} liesst Kommandos aus einer Datei und bearbeitet sie, um
einige Funktionen auszutesten.
Die Funktionen implementieren gemeinsam einen Baum, wobei
jeder Knoten beliebig viele Kinder haben kann.  
Die Anzahl von Kinder wird durch das Programm nicht begrenzt. 
Ein Knoten enthaelt zwei Strings (Schluessel, Inhalt). 

\subsection*{Optionen}

Keine.

\subsection*{Eingabe}

Nur die folgenden Kommandos sind in der Eingabedatei zulaessig.
Jedes Kommando muss in einer neuen Zeile anfangen.
Klein- und Grossschreibung wird bei den Schluesseln unterschieden.
Trennzeichen zwischen Kommandos und Argumenten ist immer ein
Leerzeichen; d.h., Leerzeichen duerfen nicht innerhalb eines
Strings vorkommen.
In dieser Implementierung sind Kommandozeilen (also, Kommando plus
entsprechenden Parametern) auf einer maximalen Laenge von 1024
beschraenkt. 

\begin{itemize}

\item {\tt root <Schluessel> <Inhalt>}

Erzeugt und initialisiert die Wurzel des Baumes.
Dieses Kommando muss das erste Kommando in der Eingabedatei sein.
Falls es nicht als erstes vorkommt, oder falls das Kommando wiederholt
wird, ist das Verhalten des Programms nicht definiert.

\item {\tt child <Vater-Schluessel> <Kind-Schluessel> <Kind-Inhalt>}

Fuegt an den Vater-Knoten einen neuen Kind-Knoten 
mit den angegebenen Daten an.
Falls der Vater-Knoten im Baum nicht gefunden wird, wird eine
entsprechende Fehlermeldung ausgegeben.
Falls der Schluessel des Vater-Knotens mehrfach im Baum vorhanden
ist, wird der neue Sohn an den ersten gefundenen Knoten angehaengt
(siehe auch "search" unten).

\item {\tt search <Schluessel>}

Es wird im Baum nach dem Knoten mit dem <Schluessel> gesucht.
Der Baum wird von links nach rechts nach der "depth-first
search"-Methode durchsucht, wobei der Inhalt eines Knotens beim ersten
Besuch gecheckt wird ("preorder traversal").
Falls der Knoten mit dem angegebenen Schluessel gefunden wird, wird
sein Inhalt, falls nicht, wird eine entsprechende Fehlermeldung
ausgegeben.  Gibt es mehrere Knoten mit dem gleichen Schluessel, dann
wird nur der zuerst gefundene ausgegeben.  

\item {\tt sibs <Schluessel1> <Schluessel2>}

Stellt fest, ob die Knoten mit den angegebenen Schluessel Geschwister
sind (d.h., denselben Vater-Knoten teilen).  Falls einer oder der
andere Schluessel im Baum nicht gefunden wird, wird eine entprechende
Fehlermeldung ausgegeben.  Jeder Knoten zaehlt als sein eigener
Geschwister. 
Siehe auch "search" oben fuer eine Beschreibung der Such-Methode.

\item {\tt print}

Druckt den gesamten Inhalt des Baumes aus. Zu jedem Knoten wird in 
jeweils einer neuen Zeile Schluessel, Inhalt sowie seine Tiefe
ausgegeben. Korrespondierend zur Tiefe wird die Zeile um jeweils
4 Leerzeichen eingerueckt.

\end{itemize}

\subsection*{Beispiel}
Da die folgende Beispielausgabe des Programms durch ein Textsatzprogramm 
formatiert wurde, entspricht die hier dargestellte Einrueckung nicht der
tatsaechlichen Einrueckung. 

{\small
\begin{verbatim}
% cat eingabe
root Lamb SE:_Planning_for_Change
child Lamb Rombach Software_Specifications:_A_Framework
child Lamb Brackett Software_Requirements
child Rombach Parnas Predicate_Logic_for_SE
print

% ntree eingabe

Eingabedatei `eingabe' wird bearbeitet.

Die Zeile `root Lamb SE:_Planning_for_Change' wird ausgewertet:

Die Zeile `child Lamb Rombach Software_Specifications:_A_Framework' wird ausgewertet:

Die Zeile `child Lamb Brackett Software_Requirements' wird ausgewertet:

Die Zeile `child Rombach Parnas Predicate_Logic_for_SE' wird ausgewertet:

Die Zeile `print' wird ausgewertet:
Knoten (Ebene 0): Schluessel 'Lamb', Inhalt 'SE:_Planning_for_Change'
    Knoten (Ebene 1): Schluessel 'Rombach', Inhalt 'Software_Specifications:_A_Framework'
        Knoten (Ebene 2): Schluessel 'Parnas', Inhalt 'Predicate_Logic_for_SE'
    Knoten (Ebene 1): Schluessel 'Brackett', Inhalt 'Software_Requirements'

Ende der Eingabedatei `eingabe'.
\end{verbatim}
} % small

\subsection*{Autor}

Ch.\ Lott

\subsection*{Bugs}

Die Funktionalitaet zum Einlesen von Dateien und Erkennen von
Kommandos ist nur zu Testzwecken gedacht und ist deswegen nicht
besonders fehlertolerant.  Zum Beispiel kann man davonausgehen, 
dass falsch buchstabierte Kommandos oder fehlende bzw. zu
viele Argumente bei einem Kommando nicht besonders sorgfaeltig
behandelt werden. 
