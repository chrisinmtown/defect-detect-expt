
Dies ist das Erg\"anzungsblatt bez\"uglich des Programms "`cmdline"'
f\"ur funktionales Testen.

\subsection*{Notwendige Eingaben}

\begin{itemize}

\item Welche Dokumente geh"oren zu dieser Aufgabe?

\begin{enumerate}
\item Dokument ES8, die Spezifikation der Komponente
\item Dokument EQ8, der Quellcode der Komponente (erhalten Sie, nachdem Sie
	Testf"alle erstellt und Fehlverhalten diagnostiziert haben)
\end{enumerate}

\item Wie hole ich mir das Dateipaket, das ich brauche?

Gehen sie wie folgt vor:

\begin{enumerate}

\item Zuerst legen Sie ein neues Verzeichnis daf\"ur an mittels des 
"`mkdir"'-Kommandos.
\begin{verbatim}
    mkdir ft-cmdline
\end{verbatim}

\item Dann wechseln Sie in das neue Verzeichnis mittels des 
"`cd"'-Kommandos.
\begin{verbatim}
    cd ft-cmdline
\end{verbatim}

\item Zuletzt geben Sie folgendes Kommando ein:
\begin{verbatim}
    tar xf ~uebung00/Experiment/ft-cmdline.tar
\end{verbatim}

\end{enumerate}

\item Was mu"s vorhanden sein ?

Die folgenden Dateien m\"ussen vorhanden sein:
\begin{verbatim}
    Makefile        cmdline         run-suite       test-dir
\end{verbatim}

\end{itemize}

\subsection*{Bildung von "Aquivalenzklassen und Testf"allen}

Die Komponente "`cmdline"' kann fast direkt getestet werden. Es sind nur
f"ur die einzelnen Aufrufe von "`cmdline"' die Argumente in 
Parameterdateien abzulegen. Es werden keine (MIL/MDL-) Eingabedateien
ben"otigt.

\newpage
\section*{E28, Blatt 2 -- Wird mit dem Quellkode ausgegeben !}

\subsection*{Kurze Beschreibung der verwendeten Bibliotheks-Funktionen}

\begin{itemize}
   \item {\tt int strlen(char *s)}\\
      Liefert die L"ange des Strings {\tt s}
      zur"uck (Null-Zeichen am Ende wird nicht mitgez"ahlt).
   \item {\tt int strcmp(char *s1, char *s2)}\\
      Vergleicht zwei Strings. Liefert einen Wert gr"o"ser als, gleich, oder
      kleiner als 0 zur"uck, abh"angig davon, ob {\tt s1} lexikographisch
      (d.h.~ASCII-Zeichenwert) gr"o"ser, gleich, oder
      kleiner als {\tt s2} ist.
   \item {\tt int strncmp(char *s1, char *s2, int n)}\\
      Wie {\tt strcmp}, vergleicht maximal {\tt n} Zeichen.
   \item {\tt int fprintf(stderr,"Text"...)}\\
      Gibt den Text auf der Standardfehlerausgabe aus.
   \item {\tt int isdigit(char c)}\\
      Liefert {\em true\/}, wenn das Zeichen {\tt c} eine Ziffer ist,
      ansonsten {\em false\/}.
   \item {\tt int atoi(char *str)}\\
      Wandelt den String {\tt str} in einen Integer um. F"uhrende
      "`white-spaces"' werden ignoriert, das scannen das Strings wird mit dem
      ersten Zeichen, das keine Ziffer ist, abgebrochen. Konnte keine Ziffer
      gelesen werden, wird 0 zur"uckgegeben.
\end{itemize}
