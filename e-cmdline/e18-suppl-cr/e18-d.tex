
Dies ist das Erg\"anzungsblatt bez\"uglich des Programms "`cmdline"'
f\"ur "code-reading"'.

\subsection*{Kurze Beschreibung der verwendeten Bibliotheks-Funktionen}

\begin{itemize}
\item {\tt int strlen(char *s)}\\
	Liefert die L"ange des Strings {\tt s} 
	zur"uck (Null-Zeichen am Ende wird nicht mitgez"ahlt).
\item {\tt int strcmp(char *s1, char *s2)}\\
	Vergleicht zwei Strings. Liefert einen Wert gr"o"ser als, gleich, oder
	kleiner als 0 zur"uck, abh"angig davon, ob {\tt s1} lexikographisch
	(d.h.~ASCII-Zeichenwert) gr"o"ser, gleich, oder 
	kleiner als {\tt s2} ist.	
\item {\tt int strncmp(char *s1, char *s2, int n)}\\
	Wie {\tt strcmp}, vergleicht maximal {\tt n} Zeichen.
\item {\tt int fprintf(stderr,"Text"...)}\\
	Gibt den Text auf der Standardfehlerausgabe aus.
\item {\tt int isdigit(char c)}\\
	Liefert {\em true\/}, wenn das Zeichen {\tt c} eine Ziffer ist,
	ansonsten {\em false\/}.
\item {\tt int atoi(char *str)}\\
	Wandelt den String {\tt str} in einen Integer um. F"uhrende 
	"`white-spaces"' werden ignoriert, das scannen des Strings wird mit dem
	ersten Zeichen, das keine Ziffer ist, abgebrochen. Konnte keine Ziffer
	gelesen werden, wird 0 zur"uckgegeben.
\end{itemize}

\subsection*{Hinweis}

Machen Sie keine Abstraktionen von den Testumgebungs-Funktionen.

