
This is the supplemental sheet 
for applying technique ``functional testing''
to the program ``nametbl.'' 

\subsection*{Necessary inputs}

\begin{itemize}

\item Which documents belong to this exercise?

\begin{enumerate}
\item Document ES9, the specification of the component
\item Document EQ9, the source code of the component, which you will
receive after you have written test cases and diagnosed failures.
\end{enumerate}

\item How do I fetch the files which I need?

Do the following:

\begin{enumerate}

\item First create a new directory for this exercise with the
``mkdir'' command.
\begin{verbatim}
    mkdir ft-nametbl
\end{verbatim}

\item Then change to the new directory with the ``cd'' command.
\begin{verbatim}
    cd ft-nametbl
\end{verbatim}

\item Finally, enter the following command:
\begin{path}
\begin{verbatim}
    tar xf ~prakt00/Exercise5/ft-nametbl.tar
\end{verbatim}
\end{path}

\end{enumerate}


\item What should I have?

The following files must be available:
\begin{verbatim}
    Makefile        nametbl         run-suite       test-dir
\end{verbatim}

\end{itemize}

\subsection*{Creating equivalence classes and test cases}

The test scaffolding offers commands and their parameters
to simplify addressing the program's functions.
Do not create equivalence classes for syntactically incorrect
commands!  That would only test the test scaffolding.
Examples of unnecessary test cases:
\begin{verbatim}
delete all
\end{verbatim}
Command not defined.
\begin{verbatim}
ins
\end{verbatim}
Argument is missing.

\subsection*{Reminder}

Make sure that your test cases generate output so that you can
detect failures!


\newpage
\noindent {\Large \bf Document E29, part 2: To be given out with the code!}

\subsection*{Brief description of the library functions used}

\begin{itemize}

\item {\tt assert(expression)}

Macro that expects the expression to be true at the time the
program is executed and in that case does nothing.  Otherwise an error
message with the expression is printed and the program is ended.

\item {\tt char* malloc(unsigned size)}

Returns a pointer to a block of memory that is at least as large as
{\tt size} bytes.  In case of failure, NULL is returned.

\item {\tt int free(ptr)}

Gives a previously allocated block back to the system.  
{\tt ptr} must have been  allocated using {\tt malloc} or a similar
function.  If successful, 1 is returned; otherwise 0 is returned.

\item {\tt int strcmp(s1,s2)}

Compares two strings.  Returns a value greater than, equal to, or less
than 0 depending on whether {\tt st1} is lexicographically (i.e.,
ASCII value) greater than, equal, or less than {\tt s2}.

\item {\tt char *strdup(s1)}

Duplicates the string {\tt s1}.  Allocates memory for this purpose
using {\tt malloc()} and returns a pointer to the duplicate if
successful, otherwise NULL.

\item {\tt int sscanf(s,format,pointer)}

Reads characters from string {\tt s}, converts them according to
the specification string {\tt format}, and writes the result into
the variables pointed to by the {\tt pointer}s.
This is the reverse function of {\tt printf}; the format specification
strings are identical.

\item {\tt char *fgets(s,n,stream)}

Reads characters from the stream {\tt stream} and writes them in the
string {\tt s}, until either {\tt n-1} characters have been read or
one of NEWLINE or EOF are seen.
At EOF the value NULL is returned, otherwise {\tt s}.

\item {\tt char *tsearch(char *key, char **rootp, int (*compar)())}
\item {\tt char *tfind(char *key, char **rootp, int (*compar)())}
\item {\tt void twalk(char *root, void (*action)())}

These functions implement a binary search tree.  All required
comparisons are performed with the function {\tt compar}, which the
user must provide for use by the tree functions.
The function {\tt compar} is made known to the tree functions with the
help of a function pointer.  
{\tt compar} is invoked with two arguments, which point to the
elements to be compared.
The return value must be less than, equal to, or larger than 0,
depending on whether the first element is less than, equal to, or
larger than the second (analogous to {\tt strcmp}).

{\tt tsearch} can both build and search a tree.
If an element is found which has the same key as the element to which
{\tt key} points, a pointer is returned.  This pointer points to a
pointer which points to the element that was found.  
Otherwise the element pointed to by {\tt key} is inserted into the
tree and a pointer-to-a-pointer to the new element is returned.
An illustration about the double indirection:
\begin{verbatim}
    returned   +--+          +--------+ 
    ---------->|  |--------->|  node  |
    pointer    +--+ pointer  +--------+ 
                                /  \
\end{verbatim}
Only pointers are copied; i.e., the user is responsible for reserving
memory for the tree elements (e.g., with {\tt malloc}).
{\tt rootp} points to a variable and this variable points to the root
of the tree.
To produce the first node in the tree (the root), NULL must be stored
in the variable, thereafter the variable receives a pointer to the
root of the tree.

{\tt tfind} is identical to {\tt tsearch}, with the exception that if
the element is not found, NULL is returned and no new node is added to
the tree. 

{\tt twalk} traverses a tree or a subtree, because any node can be
used for the {\tt root} argument.  Take note that only a pointer to
{\tt root} can be given, not a pointer-to-a-pointer as used by
{\tt tsearch} and {\tt tfind}.
{\tt action} is the function which is invoked for each node in the
tree.
This function must also be provided by the user.  It receives
three arguments: a pointer to the tree element, information about when
the element is visited, and the depth of the tree element (root = 0).
The visit information is an enumerated type  {\tt typedef enum
\{preorder, postorder, endorder, leaf \} VISIT;} where {\tt preorder},
{\tt postorder}, and {\tt endorder} represent the first, second, and
third visit during a depth-first, left-to-right traversal of the
tree. 
{\tt leaf} refers to the leaf of a tree.

\end{itemize}
