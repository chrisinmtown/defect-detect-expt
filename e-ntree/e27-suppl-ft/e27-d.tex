
Dies ist das Erg\"anzungsblatt bez\"uglich des Programms "`ntree"'
f\"ur funktionales Testen.

\subsection*{Notwendige Eingaben}

\begin{itemize}

\item Welche Dokumente geh"oren zu dieser Aufgabe?

\begin{enumerate}
\item Dokument ES7, die Spezifikation der Komponente 
\item Dokument EQ7, der Quellcode der Komponente (erhalten Sie, nachdem
	Sie Testf"alle erstellt und Fehlverhalten diagnostiziert haben)
\end{enumerate}

\item Wie hole ich mir das Dateipaket, das ich brauche?

%Das Dateipaket befindet sich auf unseren Anlage in einer tar-Datei.
%Der folgende Proze{\ss} soll gefolgt werden.
Gehen Sie wie folgt vor:

\begin{enumerate}

\item Zuerst legen Sie ein neues Verzeichnis daf\"ur an mittels des 
"`mkdir"'-Kommandos.
\begin{verbatim}
    mkdir ft-ntree
\end{verbatim}

\item Dann wechseln Sie in das neue Verzeichnis mittels des 
"`cd"'-Kommandos.
\begin{verbatim}
    cd ft-ntree
\end{verbatim}

\item Zuletzt geben Sie folgendes Kommando ein:
\begin{verbatim}
    tar xf ~uebung00/Experiment/ft-ntree.tar
\end{verbatim}

\end{enumerate}

\item Was mu"s vorhanden sein ?

Die folgende Dateien m\"ussen vorhanden sein:
\begin{verbatim}
    Makefile        ntree           run-suite       test-dir
\end{verbatim}

\end{itemize}

\subsection*{Bildung von "Aquivalenzklassen und Testf"allen}

Die Kommandos mit ihren Parametern werden von der Testumgebung zur
Verf"ugung gestellt, um die Funktionen des Progamms m"oglichst einfach
ansprechen zu k"onnen. Bilden Sie keine "Aquivalenzklassen f"ur syntaktisch
falsche Kommandos ! Sie w"urden nur die Testumgebung testen.
Beispiele f"ur unsinnige Testf"alle:
\begin{verbatim}
delete all
\end{verbatim}
Kommando nicht definiert.
\begin{verbatim}
sibs Brackett
\end{verbatim}
Zweiter Parameter fehlt.

\newpage
\section*{E27, Blatt 2 -- Wird mit dem Quellkode ausgegeben !}

\subsection*{Kurze Beschreibung des verwendeten Bibliotheks-Funktionen}

\begin{itemize}
	\item {\tt assert(expression)}\\
		Makro, der erwartet, da"s {\tt expression} zum Zeitpunkt der Ausf"uhrung
		{\tt true} ist und in diesem Fall nichts tut. Ansonsten wird eine 
		Fehlermeldung mit der {\tt expression} ausgegeben.
	\item {\tt char* malloc(size)}\\
		Liefert ein Pointer auf einen Speicherblock zur"uck, der mindestens 
		{\tt size}
		Bytes gro"s ist. Bei Mi"serfolg wird NULL zur"uckgegeben.
	\item {\tt int free(ptr)}\\
		Gibt einen zuvor alloziierten Block zur"uck. {\tt ptr} mu"s mit 
		{\tt malloc} o."a. alloziiert worden sein. Bei Erfolg wird 1, sonst
 		0 zur"uckgegeben.
	\item {\tt int strcmp(s1,s2)}\\
		Vergleicht zwei Strings. Liefert einen Wert gr"o"ser als, gleich, oder
		kleiner als 0 zur"uck, abh"angig davon, ob {\tt s1} lexikographisch
		(d.h.~ASCII-Zeichenwert) gr"o"ser, gleich, oder 
		kleiner als {\tt s2} ist.	
	\item {\tt void bcopy(b1,b2,length)}\\
		Kopiert {\tt length} Bytes von String {\tt b1} nach String {\tt b2}. 
		"Uberlappende Strings werden korrekt behandelt.
	\item {\tt int sscanf(s,format,pointer)}\\
		Lie"st Zeichen von String {\tt s} und wandelt sie entsprechend dem
		spezifizierten {\tt format} um und schreibt die Resultate in die Variablen
		auf die die {\tt pointer} zeigen. Dies ist die Umkehrfunktion 
		von {\tt printf}, die Formatspezifikationen sind identisch. 
	\item {\tt char *fgets(s,n,stream)}\\
		Lie"st Zeichen aus dem Stream {\tt stream} bis zum NEWLINE oder EOF und
		schreibt sie in den String {\tt s}, bis {\tt n-1} Zeichen gelesen wurden.
		Bei EOF wird NULL zur"uckgegeben, sonst {\tt s}.
	\item {\tt char *strdup(s1)}\\
		Dupliziert den String {\tt s1}. Alloziiert zu diesem Zweck Speicher
		mit {\tt malloc()} und gibt bei Erfolg von {\tt malloc} den 
		Pointer darauf zur"uck, sonst NULL.
\end{itemize}

