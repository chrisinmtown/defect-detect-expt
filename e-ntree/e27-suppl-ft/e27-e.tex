
This is the supplemental sheet 
for applying technique ``functional testing''
to the program ``ntree.''


\subsection*{Necessary inputs}

\begin{itemize}

\item Which documents belong to this exercise?

\begin{enumerate}
\item Document ES7, the specification of the component
\item Document EQ7, the source code of the component, which you will
receive after you have written test cases and diagnosed failures.
\end{enumerate}

\item How do I fetch the files which I need?

Do the following:

\begin{enumerate}

\item First create a new directory for this exercise with the
``mkdir'' command.
\begin{verbatim}
    mkdir ft-ntree
\end{verbatim}

\item Then change to the new directory with the ``cd'' command.
\begin{verbatim}
    cd ft-ntree
\end{verbatim}

\item Finally, enter the following command:
\begin{path}
\begin{verbatim}
    tar xf ~prakt00/Exercise5/ft-ntree.tar
\end{verbatim}
\end{path}

\end{enumerate}


\item What should I have?

The following files must be available.
\begin{verbatim}
    Makefile        ntree           run-suite       test-dir
\end{verbatim}

\end{itemize}


\subsection*{Creating equivalence classes and test cases}

The test scaffolding offers commands and their parameters
to simplify addressing the program's functions.
Do not create equivalence classes for syntactically incorrect
commands!  That would only test the test scaffolding.
Examples of unnecessary test cases:
\begin{verbatim}
delete all
\end{verbatim}
Command not defined.
\begin{verbatim}
sibs Brackett
\end{verbatim}
Second parameter is missing.

\subsection*{Reminder}

Make sure that your test cases generate output so that you can
detect failures!

\newpage
\noindent {\Large \bf Document E27, part 2: To be given out with the code!}

\subsection*{Brief description of the library functions used}

\begin{itemize}

\item {\tt assert(expression)}

Macro that expects the expression to be true at the time the
program is executed and in that case does nothing.  Otherwise an error
message with the expression is printed and the program is ended.

\item {\tt char* malloc(unsigned size)}

Returns a pointer to a block of memory that is at least as large as
{\tt size} bytes.  In case of failure, NULL is returned.

\item {\tt int free(ptr)}

Gives a previously allocated block back to the system.  
{\tt ptr} must have been allocated using {\tt malloc} or a similar
function.  If successful, 1 is returned; otherwise 0 is returned.

\item {\tt int strcmp(s1,s2)}

Compares two strings.  Returns a value greater than, equal to, or less
than 0 depending on whether {\tt st1} is lexicographically (i.e.,
ASCII value) greater than, equal, or less than {\tt s2}.

\item {\tt void bcopy(b1,b2,length)}

Copies {\tt length} bytes from string {\tt b1} to string {\tt b2}.
Overlapping strings are treated correctly.

\item {\tt int sscanf(s,format,pointer)}

Reads characters from string {\tt s}, converts them according to
the specification string {\tt format}, and writes the result into
the variables pointed to by the {\tt pointer}s.
This is the reverse function of {\tt printf}; the format specification
strings are identical.

\item {\tt char *fgets(s,n,stream)}

Reads characters from the stream {\tt stream} and writes them in the
string {\tt s}, until either {\tt n-1} characters have been read or
one of NEWLINE or EOF are seen.
At EOF the value NULL is returned, otherwise {\tt s}.

\item {\tt char *strdup(s1)}

Duplicates the string {\tt s1}.  Allocates memory for this purpose
using {\tt malloc()} and returns a pointer to the duplicate if
successful, otherwise NULL.

\end{itemize}
