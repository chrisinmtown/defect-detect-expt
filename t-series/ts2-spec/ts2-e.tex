
\subsection*{Name}
series -- generate an additive series of numbers

\subsection*{Usage}

{\bf series} start end [stepsize]

\subsection*{Description}

{\bf series} prints the real numbers from
{\bf start} to {\bf end}, one per line.
{\bf series} begins with {\bf start} 
to which {\bf stepsize} is repeatedly added or subtracted,
as appropriate, to approach, possibly meet, but not pass
{\bf end}.

If all arguments are integers, only integers are 
produced in the output.
The {\bf stepsize} must be nonzero; if it is not specified,
it is assumed to be of unit size (1).
In all other cases,
{\bf series} prints an appropriate error message.

\subsection*{Example}
To count from 1 to 100:
\begin{verbatim}
	series 1 100 
\end{verbatim}
To do the same, but backwards:
\begin{verbatim}
	series 100 1
\end{verbatim}

\subsection*{Limitations}

The reported number of significant digits is limited.
If the ratio of the series range to the
{\bf stepsize} is too large, several numbers in a row will be equal. 

The maximum length of a series is limited to the size of the
maximum long integer that can be represented on the machine in use.
Exceeding this value has undefined results.

\subsection*{Author}

Gary Perlman
