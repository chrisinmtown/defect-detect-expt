

Die folgenden Tabellen zeigen eine Menge m\"ogliche Equivalenzklassen
f"ur das Programm ``series.''

\begin{table}[h]
\begin{center}
\begin{tabular}{|r|c|l|}
\hline
Nr. & G/I$^*$ & Equivalenzklasse \\
\hline
\hline
  1 &   G & Zahl der Argumente ist 2   \\
\hline
  2 &   G & Zahl der Argumente ist 3   \\
\hline
  3 &   U & Zahl der Argumente ist weder 2 noch 3\\
\hline
\hline
  4 &   U & 1. Argument ist keine Zahl  \\
\hline
  5 &   U & 2. Argument ist keine Zahl  \\
\hline
  6 &   U & 3. Argument ist keine Zahl  \\
\hline
  7 &   G & alle Argumente sind Zahlen  \\
\hline
\hline
  8 &   G & Startwert ist gr\"o{\ss}er als Endwert  \\
\hline
  9 &   G & Startwert ist kleiner als Endwert  \\
\hline
 10 &   G & Startwert ist gleich Endwert  \\
\hline
\hline
 11 &   U & Inkrement ist 0  \\
\hline
 12 &   G & Inkrement ist gr\"o{\ss}er als abs(Endwert - Startwert)  \\
\hline
 13 &   G & Inkement ist kleiner als abs(Endwert - Startwert)  \\
\hline
\hline
 14 &   G & Alle Argumente sind ganze Zahlen  \\
\hline
 15 &   G & Mindestens ein Argument ist Gleitpunktzahl  \\
\hline
\multicolumn{3}{l}{$^*$G = g\"ultige \"Aquivalenzklasse, U = ung\"ultige
\"Aquivalenzklasse}
\end{tabular}
\end{center}
\caption{Equivalenczklassen f"ur Programm ``series''}
\label{T:equiv-classes-series}
\end{table}

\begin{table}[h]
\begin{center}
\begin{tabular}{|r|l|l|l|}
\hline
Nr. & Testfall    & Equiv.-kl.      & Erwartete Ausgabe \\
\hline
\hline
1 & series 1       & 3               & Fehlermeldung \\
\hline
2 & series x 2     & 1, 4            & Fehlermeldung  \\
\hline
3 & series 1 x     & 1, 5            & Fehlermeldung  \\
\hline
4 & series 1 2 x   & 2, 6	     & Fehlermeldung  \\
\hline
5 & series 10 0    & 1, 7, 8, 14     & 10, 9, 8, ..., 1, 0\\
\hline
6 & series 1 2 0.1 & 2, 7, 9, 13, 15 & 1.0, 1.1, 1.2, ..., 1.9, 2.0\\
\hline
7 & series 7 7     & 1, 7, 10, 14    & die Zahl '7' \\
\hline
8 & series 1 2 0   & 2, 7, 9, 11, 14 & Fehlermeldung \\
\hline
9 & series 2 1 0.1 & 2, 7, 8, 13, 15 & 2.0, 1.9, 1.8, ..., 1.1, 1.0\\
\hline
10 & series 1 2 3  & 2, 7, 9, 12, 14 & die Zahl '1' \\
\hline
\end{tabular}
\end{center}
\caption{Testf\"alle f"ur Programm ``series''}
\label{T:tests-equiv-classes-series}
\end{table}



Fehlverhalten:
\begin{enumerate}
\item Testfall 2: Ein nicht-numerisches erstes Argument wird als
solches erkannt, aber in der Fehlermeldung wird das zweite Argument
(2) genannt (in der Fehlermeldung wird das erste fehlerhafte Argument
erwartet). 

\item Testfall 3: Es wird kein Fehler erkannt, wenn das zweite
Argument nicht-numerisch ist (erwartet wird eine Fehlermeldung).

\item Testfall 7: Falls der Abstand zwischen Startwert und Endwert
null ist, werden zwei Werte ausgegeben und der Endwert wird
\"uberschritten (nur der Startwert wird erwartet). 

\item Testall 5, 9: Wenn der Startwert gr\"o{\ss}er als der Endwert
ist, wird keine Ausgabe produziert (erwartet wird eine Folge von
Zahlen in absteigender Ordnung).  
\end{enumerate}

Fehler:
\begin{enumerate}
\item	Fehlverhalten:

Zeile 43: Die Variable ,endingstr" wird statt ,startstr" in printf()
verwendet 

\item Fehlverhalten:

Zeile 46: Die Variable ,startstr" wird statt ,endingstr" in der
if-Klausel verwendet 

\item	Fehlverhalten:

Zeile 81: Wert der Variablen ,nitems" mu� auf 1 statt 2 gesetzt werden

\item Fehlverhalten:

Zeile 74-80: Die Behandlung des Falls start > end wurde vergessen
(Step mu{\ss} negiert werden) 

\end{enumerate}
