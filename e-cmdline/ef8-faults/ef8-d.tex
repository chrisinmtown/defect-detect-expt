

Die Komponente ``cmdline`' weist die folgenden Fehlverhalten auf:

\begin{enumerate}

\item Die Option --ueber wird nicht erkannt.

\item In der Ausgabe kommt ein falsch buchstabiertes Wort
(``Suchption'') vor. 

\item Abgek"urzte Optionen werden generell als eindeutig interpretiert
und mit der ersten passenden Option assoziiert. 
Z.B. wird ``--m'' als ``--mass'' erkannt
obwohl der eindeutige Anfang nicht gesehen wurde.

\item Suchoptionen ohne Ma"sangabe stellen keine Fehleingabe dar.

\item Wenn Option ``-mass'' ohne Ma"sangabe angegeben wurde,
wird die Fehlermeldung ``es ist kein g"ultiges Ma"s angegeben''
ausgegeben, sofern weitere Argumente vorhanden waren, ansonsten
Absturz. 

\item Das Ma"s LKHM wird immer durch LKOM ersetzt.

\item Bei Angabe der Option --min erscheint die 
Fehlermeldung ``Zuviele Suchargumente angegeben'', da noch die
Option ``--unter'' ausgewertet wird.

\item Reele Zahlen werden zu Integern umgewandelt.

\item Negative Zahlen werden nicht als g"ultige Zahlen erkannt.

\item Wenn alle Optionen korrekt erkannt werden
konnten, wird das Fehlen von Dateien nicht bemerkt. Es werden der 
Kommandoname und die angegebenen Optionen als zu lesende Dateien
aufgef"uhrt. 

\end{enumerate}

Auf der n"achsten Seite werden die notwendigen "Aquivalenzklassen und
Testf"alle aufgelistet, die die obigen Fehlverhalten sichtbar machen.

\newpage

\textbf{"Aquivalenzklassen:}
\smallskip

\begin{tabular}{l|l}
g"ultige Klassen         		& ung"ultige Klassen \\ \hline
$G_1$ := \{genau 1 Ma{\ss}\}  		& $U_1$ := \{kein Ma{\ss}\} , $U_2$ := \{mehr als 1 Ma{\ss}\} \\
$G_2$ := \{max. 1 Option\}              & $U_3$ := \{mehr als eine Option\} \\
$G_3$ := \{mind. 1 Datei\}              & $U_4$ := \{keine Datei\} \\
$G_4$ := \{eindeutig verk"urzte Option\}& $U_5$ := \{nicht eindeutig verk"urzte Option\} \\
$G_5$ := \{"uber/unter-Wert reell\}     & $U_6$ := \{kein "uber/unter Wert\}  \\
$G_6$ := \{Reihenfolge Ma{\ss} Option\}    &                                     \\
$G_7$ := \{Reihenfolge Option Ma{\ss}\}    &                                     \\
$G_{81}$:= \{Option -max\}              &                                   \\
$G_{82}$:= \{Option -min\}              &                                   \\
$G_{83}$:= \{Option -durch\}            &                                   \\
$G_{84}$:= \{Option -unter\}            &                                   \\
$G_{85}$:= \{Option -ueber\}            &                                   \\
$G_{86}$:= \{Option -alle\}             &                                   \\
$G_{87}$:= \{Option -hilfe -?\}         &                                   \\
$G_{91}$:= \{Ma{\ss} -LKOM\}               & $U_7$ :=\{kein bekanntes Ma{\ss}\}   \\
$G_{92}$:= \{Ma{\ss} -GKOM\}               &                                   \\
$G_{93}$:= \{Ma{\ss} -GKHM\}               &                                   \\
$G_{94}$:= \{Ma{\ss} -LKHM\}               &                                   \\
$G_{95}$:= \{Ma{\ss} -GIHE\}               &                                   \\
$G_{96}$:= \{Ma{\ss} -LIHE\}               &                                   \\
\end{tabular}

\medskip
\textbf{Testf"alle:}
\smallskip

\begin{tabular}{l|l|l|l}
Nr. & abgedeckte Klassen  	& Aufruf des Testling 		& Fehlver.-Nr. \\ \hline
1   & $U_1$         		& cmdline -max Datei1                       & 4   \\
2   & $U_2$                     & cmdline -mass LKOM -mass GKOM -max Datei1 & --- \\
3   & $U_3$                     & cmdline -mass LKOM -alle -min Datei       & 7   \\
4   & $U_4$                     & cmdline -mass LKOM -durch                 & 10   \\
5   & $U_5$                     & cmdline -mass LKOM -m Datei1              & 3   \\
6   & $U_6$                     & cmdline -mass GIHE -unter Datei1          & --- \\
7   & $U_7$                     & cmdline -mass T Datei1                    & --- \\
8   & $G_1,G_2,G_3,G_{92},G_{82},G_6$& cmdline -mass GKOM -min  Datei1     & 7   \\
9   & $G_4, G_6, G_7 G_{91}, G_{83}$& cmdline -du -mass LKO Datei1          & --- \\
10  & $G_{93}, G_{85}, G_5$     & cmdline -mass GKHM -ueber 4.2 Datei1    & 1   \\
11  & $G_{94}, G_{81}, G_2$     & cmdline -mass LKHM -max Datei1 Datei2   & 6   \\
12  & $G_{95}, G_{84}, G_6$     & cmdline -mass GIHE -unter 3.14 Datei1   & 8   \\
13  & $G_{95}, G_{84}, G_6$     & cmdline -mass GIHE -unter -2 Datei1     & 9   \\
14  & $G_{96}, G_{86}$          & cmdline -mass LIHE -alle Datei1         & --- \\
15  & $G_{87}$                  & cmdline -hilfe                          & 2   \\
16  & $U_1, U_4$                & cmdline -mass                           & 5   \\
\end{tabular}

\medskip
Bemerkung 1: Bei g\"ultigen Klassen werden nur jeweils die genannt, die
zum ersten Mal abgedeckt werden

\smallskip
Bemerkung 2: Den Testfall Nummer 16 erh"alt man, indem man zwei
ung"ultige "Aquivalenzklassen in einem Testfall abdeckt. Dieses
widerspricht zwar dem Vorgehen nach dem Schema zur Verwendung von
\"Aquivalenzklassen, ist jedoch die einzige M"oglichkeit, das
Fehlverhalten 5 zu entdecken. Ein Beispiel daf"ur, da{\ss} keine Methode
perfekt ist.   

