
\subsection*{Name}
series -- eine additive Folge von Zahlen generieren

\subsection*{Verwendung}

{\bf series} Startzahl Stopzahl [ Schrittgr"o"se ]

\subsection*{Beschreibung}

{\bf series} druckt eine auf- bzw.\ absteigende Folge von
Real-Zahlen ausgehend von {\bf Startzahl} bis zur {\bf Stopzahl}
aus.
Jede Zahl erscheint jeweils in einer Zeile.
Der Wert $\mid$~{\bf Schrittgr"o"se}~$\mid$ spezifiziert den Abstand
zwischen aufeinanderfolgenden Zahlen.
Die Folge ist von maximaler L\"ange und bricht jeweils vor dem
"Uberschreiten der {\bf Stopzahl} ab.
Das bedeutet, abh"angig von der {\bf Schrittgr"o"se} kann die
{\bf Stopzahl} exakt erreicht werden.

Falls alle Argumente Integers sind, dann werden nur Integers in 
der Ausgabe produziert. 
Die {\bf Schrittgr"o"se} mu"s ungleich null sein;
falls sie nicht spezifiziert wird, wird eine Schrittgr"o"se
von 1 angenommen.
In allen anderen F"allen druckt {\bf series} eine entsprechende
Fehlermeldung aus.

\subsection*{Beispiel}
Um von 1 nach 100 zu z"ahlen:
\begin{verbatim}
	series 1 100 
\end{verbatim}
Das gleiche, aber umgekehrt:
\begin{verbatim}
	series 100 1
\end{verbatim}

\subsection*{Einschr"ankungen}

Die ausgegebene Anzahl signifikanter Ziffern ist eingeschr"ankt.
Falls das Verh"altnis vom $\mid$`{\bf Stopzahl -- Startzahl}~$\mid$
gegen"uber der {\bf Schrittgr"o"se} zu gro"s wird, dann werden
einige aufeinanderfolgende Zahlen identisch sein.

Die maximale L"ange einer Folge ist die maximale Gr"o"se eines 
``long integer'' auf dem verwendeten Rechner eingeschr"ankt.
Sollte dieser Wert "uberschritten werden, dann sind die Resultate
undefiniert. 

\subsection*{Autor}

Gary Perlman
