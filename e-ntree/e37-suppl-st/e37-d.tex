
Dies ist das Erg\"anzungsblatt bez\"uglich des Programms "`ntree"'
f\"ur strukturelles Testen.

\bigskip

In dieser Aufgabe werden haupts\"achlich einige C-Funktionen
ausgetestet.  
Um Ihnen Zeit und Arbeit bei dieser
Aufgabe zu sparen, wurde eine Testumgebung angelegt.  Die dazu notwendigen
Treiberfunktionen befinden sich in der schon kompilierten Datei
"`driver.o"' und werden beim Linken eingebunden.  
Die Treiber erledigen Funktionen wie \"Uberpr\"ufen und \"Offnen von
Eingabedateien, Allokierung von notwendigem Speicher usw.  Siehe auch
weitere Notizen bez\"uglich einer sichtbaren Treiberfunktion unten.

\subsection*{Notwendige Eingaben}

\begin{itemize}

\item Welche Dokumente geh"oren zu dieser Aufgabe?

\begin{enumerate}
\item Dokument ES7, die Spezifikation der Komponente (erhalten Sie, nachdem
   Sie Testf"alle erstellt und Fehlverhalten diagnostiziert haben)
\item Dokument EQ7, der Quellcode der Komponente
\end{enumerate}

\item Wie hole ich mir das Dateipaket, das ich brauche?

%Das Dateipaket befindet sich auf unseren Anlage in einer tar-Datei.
% Der folgende Proze{\ss} soll gefolgt werden.
Gehen Sie wie folgt vor:

\begin{enumerate}

\item Zuerst legen Sie ein neues Verzeichnis daf\"ur an mittels des 
"`mkdir"'-Kommandos.
\begin{verbatim}
    mkdir st-ntree
\end{verbatim}

\item Dann wechseln Sie in das neue Verzeichnis mittels des 
"`cd"'-Kommandos.
\begin{verbatim}
    cd st-ntree
\end{verbatim}

\item Zuletzt geben Sie folgendes Kommando ein:
\begin{verbatim}
    tar xf ~uebung00/Experiment/st-ntree.tar
\end{verbatim}

\end{enumerate}

\item Was mu"s vorhanden sein ?

Die folgende Dateien m\"ussen vorhanden sein:
\begin{verbatim}
    Makefile     gct-map     ntree     run-suite     test-dir
\end{verbatim}

% \item Wie wird die Umgebung initialisiert ?
% \begin{verbatim}
%    % make gct
% \end{verbatim}
% Es erscheint:
% \begin{verbatim}
%    gct-init
%    make ntree CC=gct
%    gct -g  -target sun4 -c  ntree.c
%    gct -o ntree ntree.o str-driver.o
% \end{verbatim}

\end{itemize}


\subsubsection*{Schreiben von Testf"allen}

Getestet werden alle Funktionen des Programms grunds"atzlich durch den
Aufruf 
\begin{verbatim}
ntree <Eingabedatei>
\end{verbatim}
Eine Parameterdatei enth"alt folglich nur den Namen der Eingabedatei. Dies
ist zwar etwas umst"andlich, aber aus Konsistenzgr"unden n"otig.
Das Format der Eingabedatei ist wie folgt:
\begin{verbatim}
Kommando1 Parameter1 Paramter2
Kommando2 ParamterX
...
\end{verbatim}

Welche Kommandos in der Eingabe erlaubt sind und welche Funktionen
dadurch aufgerufen werden, entnehmen Sie der Funktion 
{\tt fuehre\_kommandos\_aus} in der Datei {\tt ntree.c}.
Da diese Funktion zur Testumgebung geh"ort und nicht Bestandteil der 
eventuellen Anwendung sein wird, ist sie nicht besonders robust.
Sie k\"onnen davon ausgehen, da{\ss} diese Funktion, obwohl
relativ sensibel, doch fehlerfrei ist.
Eine T\"atigkeit des Treibers aus {\tt driver.o} ist es, 
die erste Datei auf der Kommandozeile zu \"offnen 
und die Funktion {\tt fuehre\_kommandos\_aus} mit einem Zeiger 
auf der ge\"offneten Datei aufzurufen. 
Schreiben Sie keine Testf"alle zur "Uberdeckung der Testumgebung !


\subsection*{Kurze Beschreibung des verwendeten Bibliotheks-Funktionen}

\begin{itemize}
	\item {\tt assert(expression)}\\
		Makro, der erwartet, da"s {\tt expression} zum Zeitpunkt der Ausf"uhrung
		{\tt true} ist und in diesem Fall nichts tut. Ansonsten wird eine 
		Fehlermeldung mit der {\tt expression} ausgegeben.
	\item {\tt char* malloc(size)}\\
		Liefert ein Pointer auf einen Speicherblock zur"uck, der mindestens 
		{\tt size}
		Bytes gro"s ist. Bei Mi"serfolg wird NULL zur"uckgegeben.
	\item {\tt int free(ptr)}\\
		Gibt einen zuvor alloziierten Block zur"uck. {\tt ptr} mu"s mit 
		{\tt malloc} o."a. alloziiert worden sein. Bei Erfolg wird 1, sonst
 		0 zur"uckgegeben.
	\item {\tt int strcmp(s1,s2)}\\
		Vergleicht zwei Strings. Liefert einen Wert gr"o"ser als, gleich, oder
		kleiner als 0 zur"uck, abh"angig davon, ob {\tt s1} lexikographisch
		(d.h.~ASCII-Zeichenwert) gr"o"ser, gleich, oder 
		kleiner als {\tt s2} ist.	
	\item {\tt void bcopy(b1,b2,length)}\\
		Kopiert {\tt length} Bytes von String {\tt b1} nach String {\tt b2}. 
		"Uberlappende Strings werden korrekt behandelt.
	\item {\tt int sscanf(s,format,pointer)}\\
		Lie"st Zeichen von String {\tt s} und wandelt sie entsprechend dem
		spezifizierten {\tt format} um und schreibt die Resultate in die Variablen
		auf die die {\tt pointer} zeigen. Dies ist die Umkehrfunktion 
		von {\tt printf}, die Formatspezifikationen sind identisch. 
	\item {\tt char *fgets(s,n,stream)}\\
		Lie"st Zeichen aus dem Stream {\tt stream} bis zum NEWLINE oder EOF und
		schreibt sie in den String {\tt s}, bis {\tt n-1} Zeichen gelesen wurden.
		Bei EOF wird NULL zur"uckgegeben, sonst {\tt s}.
	\item {\tt char *strdup(s1)}\\
		Dupliziert den String {\tt s1}. Alloziiert zu diesem Zweck Speicher
		mit {\tt malloc()} und gibt bei Erfolg von {\tt malloc} den 
		Pointer darauf zur"uck, sonst NULL.
\end{itemize}

