%
% faults in cmdline
%

The classification of each fault is given in braces.

\begin{enumerate}

% 1 
\item Fault in line 23, the initialization is unnecessarily done with
``-1'' (the -1 is superfluous).

\{Commission, Initialization\}

Causes failure: The last entry in the table is never found; i.e., the
option ``-ueber'' is never recognized.

% 2 
\item Fault in function ``usage'', line 28: In the word ``Suchoption''
the first ``o'' is missing.

\{commission, cosmetic\}

Causes failure: A misspelled word appears in the output.

% 3 
\item Fault in function ``is\_str\_keyword'', line 41: The if condition
does not compare the length of the string with the minimum required
length ({\tt min\_verg\_len}).

\{Commission, Interface\}

Causes failure: Abbreviated switches are generally interpreted to be
unique and associated with the first matching switch.
For example, ``-m'' is recognized as ``-mass'' although the unique
beginning was not seen.

% 4 
\item Fault in function ``print\_aufgabe'', line 68:  Expression
{\tt aufg->such == NULL} is superfluous

\{Commission, Control\}
	
Causes failure: A missing measurement option is not recognized as an
error if a search option is used.

% 5 
\item Fault in function ``process\_switches'', line 124: 
The variable ``i'' is incremented after the comparison instead of
before.

\{Commission, Data\}

Causes failure: If the switch ``-mass'' is used without a corresponding
legal measurement argument, no error message is printed provided that
a file name (really any text) follows.  However, if no argument
follows the switch, the program crashes. 

% 6 
\item Fault in function ``process\_switches'', line 137, LKOM is
assigned instead of LKHM.

\{Commission, Initialization\}

Causes failure: The switch LKHM is not processed correctly; instead,
the user always receives a report for LKOM.

% 7 
\item Fault in function ``process\_switches'', line 191: the
``break''-Statement is missing.

\{Omission, Control\}

Causes failure: If the ``-min'' option is given, an error message
complaining about too many search options appears because the code for
``--unter'' is also executed.

% 8 
\item Fault in function ``process\_switches'', line 205: the function 
``atoi()'' is used instead of ``atof()''.

\{Commission, Interface\}

Causes failure: Only the integer part of a real number is used.


% 9 
\item Fault in function ``process\_switches'', line 238/240: rc doesn't
receive the expected value from {\tt print\_dateien}.

\{Omission, Interface\}

Causes failure: If all switches were recognized correctly, missing
data files are not detected.  The command name and supplied switches
are listed as files to be read.

\end{enumerate}

Comments:

\begin{enumerate}
\item Line 28: The text does not mention the help options.
This is not treated as a fault.

\item Line 200/218: Only the first symbol of the arguments for the
switches ``-unter'' and ``-ueber'' is checked to see if it is a digit,
meaning that ``4c'' is accepted as a number.  This is not treated as a
fault. 

\end{enumerate}


The next page lists the necessary equivalence classes and test cases
to reveal the faults described above.

\newpage

\textbf{Equivalence Classes:}
\smallskip

\begin{tabular}{l|l}
Valid classes	         		& Invalid classes \\ \hline
$V_1$ := \{exactly 1 measure\} 		& $I_1$ := \{no measure\} , $I_2$ := \{more than 1 measure\} \\
$V_2$ := \{max. 1 option\}              & $I_3$ := \{more than one option\} \\
$V_3$ := \{min. 1 file\}              	& $I_4$ := \{no file\} \\
$V_4$ := \{uniquely abbreviated option\}& $I_5$ := \{not uniquely abbrev. option\} \\
$V_5$ := \{above/below value real\}     & $I_6$ := \{no above/below value\} \\
$V_6$ := \{order measure-option\}	&                                   \\
$V_7$ := \{order option-measure\}	&                                   \\
$V_{81}$:= \{option -max\}              &                                   \\
$V_{82}$:= \{option -min\}              &                                   \\
$V_{83}$:= \{option -durch\}            &                                   \\
$V_{84}$:= \{option -unter\}            &                                   \\
$V_{85}$:= \{option -ueber\}            &                                   \\
$V_{86}$:= \{option -alle\}             &                                   \\
$V_{87}$:= \{option -hilfe -?\}         &                                   \\
$V_{91}$:= \{measure -LKOM\}		& $I_7$ :=\{no known measure\}   \\
$V_{92}$:= \{measure -GKOM\}            &                                   \\
$V_{93}$:= \{measure -GKHM\}            &                                   \\
$V_{94}$:= \{measure -LKHM\}            &                                   \\
$V_{95}$:= \{measure -GIHE\}            &                                   \\
$V_{96}$:= \{measure -LIHE\}            &                                   \\
\end{tabular}

\medskip
\textbf{Test cases:}
\smallskip

\begin{tabular}{l|l|l|l}
Nr. & Covered classes	  	& Invocation of program 		   & Fault nr. \\ \hline
1   & $I_1$         		& cmdline -max Datei1                       & 4   \\
2   & $I_2$                     & cmdline -mass LKOM -mass GKOM -max Datei1 & --- \\
3   & $I_3$                     & cmdline -mass LKOM -alle -min Datei       & 7   \\
4   & $I_4$                     & cmdline -mass LKOM -durch                 & 10   \\
5   & $I_5$                     & cmdline -mass LKOM -m Datei1              & 3   \\
6   & $I_6$                     & cmdline -mass GIHE -unter Datei1          & --- \\
7   & $I_7$                     & cmdline -mass T Datei1                    & --- \\
8   & $V_1,V_2,V_3,V_{92},V_{82},V_6$& cmdline -mass GKOM -min  Datei1     & 7   \\
9   & $V_4, V_6, V_7 V_{91}, V_{83}$& cmdline -du -mass LKO Datei1          & --- \\
10  & $V_{93}, V_{85}, V_5$     & cmdline -mass GKHM -ueber 4.2 Datei1    & 1   \\
11  & $V_{94}, V_{81}, V_2$     & cmdline -mass LKHM -max Datei1 Datei2   & 6   \\
12  & $V_{95}, V_{84}, V_6$     & cmdline -mass GIHE -unter 3.14 Datei1   & 8   \\
13  & $V_{95}, V_{84}, V_6$     & cmdline -mass GIHE -unter -2 Datei1     & 9   \\
14  & $V_{96}, V_{86}$          & cmdline -mass LIHE -alle Datei1         & --- \\
15  & $V_{87}$                  & cmdline -hilfe                          & 2   \\
16  & $I_1, I_4$                & cmdline -mass                           & 5   \\
\end{tabular}

\medskip
Comment 1:  Valid equivalence classes are only named the first time they are covered.

\smallskip
Comment 2:  Test case 16 is attaind by covering two invalid equivalence classes 
at once.  This contradicts the approach discussed for using equivalence
classes but is the only possibility for revealing this failure.  This is
a good example that no method is perfect.
