Beachten Sie bitte auch das Erg"anzungsblatt "`Strukturelles Testen mit
Programm $x$"', was im Ihnen vorliegenden Material enthalten ist.

\begin{enumerate}

\item Lesen Sie sich den Fragebogen "`E32: Strukturelles Testen"'
durch. Tragen Sie Ihre Kennung ein und f"ullen Sie die Rubrik
"`Vorbetrachtungen"' aus. 

\subsection*{Test generieren}

\item Tragen Sie die Uhrzeit zu der Sie mit dem Experiment beginnen 
unter Frage 4 ein.

\item Lesen Sie sich den Kode der Komponente durch, um einen "Uberblick zu
gewinnen. Falls Ihnen jetzt oder in
den folgenden Schritten bereits m"ogliche Fehler auffallen, dann
markieren Sie diese. Investieren Sie bitte keine Zeit in die
genaue Analyse, isolieren Sie keine Fehler !!!
Fehlerisolation erfolgt ab Schritt 18.

\item Fangen Sie an, Testdaten zu generieren, die zu 100-prozentiger
"Uberdeckung der folgenden Kriterien f\"uhren werden: Zweig"uberdeckung
($C_{1}$), Schleifen"uberdeckung, Mehrfachbedingungs"uberdeckung
($C_{1}^{OP}$, $C_{1}^{AP}$) und Relationaloperator-"Uberdeckung.
Sollten Ihnen diese Kriterien unbekannt sein, dann schlagen Sie bitte auf
Seite \pageref{kriterien} dieser Anleitung nach.
Besondere Eigenschaften der Komponente entnehmen Sie dem
Erg\"anzungsblatt.
Entwerfen Sie Ihre Testdaten sorgf"altig, so vermeiden Sie sp"ater
zeitaufwendiges "`Testdaten hacking"'.

\item Nachdem Sie mit der Testdatengenerierung fertig sind, tragen 
Sie Ihren Zeitbedarf unter Frage 5 ein.

\item Loggen Sie sich bei einem Rechner bzw.~Xstation ein.  
Tragen Sie den Rechnername und Zeitpunkt unter Frage 6 ein.  
Holen Sie sich das notwendige Dateipaket (siehe Erg\"anzungsblatt).
% Tippen Sie
% \begin{quote}
% 	{\tt \% make gct}
% \end{quote}
% damit die Testumgebung initialisiert wird.
\"Uberzeugen Sie sich davon, da{\ss} alle Dateien bzw.~Verzeichnisse
vorhanden sind! 

\item Tippen Sie Ihre vorbereitete Testf\"alle in Dateien im
Verzeichnis "`test-dir"' ein. Um Ihnen die Arbeit zu erleichtern, ist die
Komponente mit einer Kommandozeilenschnittstelle versehen worden. Diese
Schnittstelle wird von einem automatischen Tester angesteuert. Der Tester
lie"st {\em Parameterdateien\/} und ruft die Komponente mit den darin
spezifizierten Parametern auf. Die Resultate werden in eine Datei geschrieben.
Auf diese Weise wird ein kompletter Test der Komponente durchf"uhrt.

Sie m"ussen die Parameter- und ggf.~{\em Eingabedateien} anlegen. Ein
Testfall $=$ eine Parameterdatei. Die
Ausdr"ucke in einer Parameterdatei spezifizieren einen Aufruf der
Komponente f"ur einen Lauf des automatischen Testers. Parameterdateien werden
mit ``.test'' als Dateikennung benannt. Beispiel: Soll die Komponente 'komp'
mit {\tt komp -i in\_file1} aufgerufen werden, dann geh"ort in
die Parameterdatei der Ausdruck {\tt -i in\_file1}.

{\em Eingabedateien\/} werden f"ur Komponenten angelegt, die aus
Dateien lesen. Eingabedateien sollten sinnvoll benannt werden,
z.B. ``leer'', sie d"urfen {\bf nicht} die Dateikennung ``.test'' aufweisen.

\"Uberzeugen Sie sich, da{\ss} in den Dateien auch wirklich das
enthalten ist, was Sie wirklich wollen!


% \item Tippen Sie Ihre vorbereitete Testf\"alle in Dateien im
% Verzeichnis "`test-dir"' ein. 
% Versehen Sie bitte jede Datei mit der Endung `.test'.
% \"Uberzeugen Sie sich, da{\ss} in den Dateien auch wirklich das
% enthalten ist, was Sie wirklich wollen!

\subsection*{Test durchf"uhren}

\item Wenden Sie die Testf\"alle auf die Komponente an indem Sie das
Kommando "`run-suite"' eingeben. 

\item Schauen Sie sich die Zusammenfassung der \"Uberdeckungen an
mit Hilfe der Kommandos "`gsummary"' und "`greport"'.
Zum Beispiel:
\begin{verbatim}
% gsummary test-dir/GCTLOG
% greport  test-dir/GCTLOG
\end{verbatim}

\item Versuchen Sie jetzt durch Generierung weiterer Tests alle Werte
auf 100\% zu bringen. Sie k\"onnen neue Tests (im Verzeichnis
`test-dir') hinzuf\"ugen.  Nachdem Sie die Testdaten ge\"andert haben,
testen Sie weiter indem Sie die folgende Befehlssequenz ausf\"uhren: 
\begin{verbatim}
% make clean 
% run-suite 
% gsummary test-dir/GCTLOG 
% greport  test-dir/GCTLOG
\end{verbatim}

\item Nachdem Sie ein \"Uberdeckungsma{\ss} von 100\% erreicht haben,
oder sich \"uberzeugt haben, da{\ss} 100-prozentige \"Uberdeckung aus
gutem Grunde nicht zu erreichen ist (mu{\ss} gut begr\"undet
werden!), sind Sie mit dem ersten Teil dieser Aufgabe fertig.

\item Tragen Sie die Anzahl der Testf\"alle, die Sie in dieser Aufgabe
generiert haben,  unter Frage 7 und die erreichten \"Uberdeckungswerte
unter Frage 8 ein.  
Bitte tragen Sie auch gleich Ihren Zeitbedarf unter Frage 9 ein.

\item Die Resultate stehen sowohl in einzelnen Dateien im Verzeichnis 
"`test-dir"' als auch in einer zusammenh\"angende Datei
"`test-results.summary"', worin die einzelnen Resultate getrennt sind.
Drucken Sie die Resultate zweimal auf dem Zeilendrucker aus. 

\item Loggen Sie sich aus und holen Sie Ihre Ausgabe vom Zeilendrucker.

\item Notieren Sie auf einem Ausdruck Ihrer Resultate Ihre Praktikumskennung,
mit welcher Testmethode Sie gearbeitet haben und in welchem Verzeichnis
Ihre Testf"alle zu finden sind. Geben Sie diesen Ausdruck
beim Betreuer ab und nehmen Sie die Spezifikation entgegen. Bitte l"oschen Sie
nicht Ihre Testf"alle !


%\item Geben Sie jetzt einen Ausdruck Ihrer Resultate beim Betreuer ab. Sie 
%erhalten dann die Spezifikation der Komponente. 

\subsection*{Fehlverhalten diagnostizieren}

\item Tragen Sie die momentane Uhrzeit unter Frage 10 ein.
Lesen Sie die Spezifikation und suchen Sie nach Fehlverhalten in den
Testergebnissen. Markieren Sie die entdeckten Fehlverhalten in der Ausgabe
durch Umkreisen, o."a. Tragen Sie die entdeckten Fehlverhalten in das 
Formblatt ein.  Nummieren Sie die Fehlverhalten von 1 bis $n$ in der
Spalte "`Fv-Nr."' (Fehlverhaltensnummer) des Formblatts und entsprechend
die Umkreisungen im Ausdruck.

%Vergessen Sie nicht, auf dem Formblatt Ihre Praktikums-Kennung anzugeben.

\item Wenn Sie meinen, alle Fehlverhalten entdeckt zu haben, 
tragen Sie bitte Ihren Zeitbedarf f"ur die Suche unter Frage 11 ein.

\subsection*{Fehler lokalisieren}

\item Tragen Sie die momentane Uhrzeit unter Frage 12 ein.
Versuchen Sie jetzt die Fehler, die zu den von Ihnen erkannten
Fehlverhalten gef\"uhrt haben, zu isolieren.
Tragen Sie die isolierten Fehler in das Formblatt ein, vergessen Sie
nicht, dort auch Ihre Praktikums-Kennung anzugeben.

Geben Sie an welche Fehler Sie entdeckt haben
[Zeilennummer]. Klassifizieren Sie dabei nach den vorgestellten
Fehlertypen [{Omission, Comission}, {Initialization, Control,
Interface, Data, Computation, Cosmetic}]. Charakterisieren Sie dann
kurz den Fehler, z.B. `Initialwert 1 statt 0'. Geben Sie auch
die Nummer des Fehlverhaltens an, zu dem der Fehler f"uhrt.

Sollten Sie zuf\"alligerwise auch Fehler erkennen, die nicht zu
Fehlverhalten bei Ihren Testergebnisse gef\"urt haben, d\"urfen
Sie sie auch auflisten.  In diesem Fall geh"ort in die Spalte
"`Fv-Nr."' ein Strich.

\item Nachdem Sie mit der Fehlerisolation fertig sind, tragen Sie
Ihren Zeitbedarf unter Frage 13 ein.

\item F"ullen Sie bitte den Rest des Fragebogens aus (Fragen 14--16).
Geben Sie den Fragebogen und alle Formbl"atter bei Ihrem Betreuer ab,
   sch\"onen Feierabend.

%\item Geben Sie den Fragebogen und alle Formbl"atter bei Ihrem Betreuer ab.
%R\"aumen Sie Ihren Arbeitsplatz auf. Verlassen Sie den
%Terminalraum, sch\"onen Feierabend.

\end{enumerate}

\subsection*{Anhang: "Uberdeckungskriterien}
\label{kriterien}

\subsubsection*{Zweig"uberdeckung}

Die Erf"ullung des Zweig"uberdeckungskriteriums stellt sicher, da"s jeder
Zweig in einer Komponente mindestens einmal ausgef"uhrt wurde. Zweige in
C-Programmen werden erzeugt durch die {\tt {if}}-, {\tt {?}}-, {\tt {for}}-,
{\tt {while}}- und {\tt {do-while}}-Anweisungen. Jede dieser Anweisungen
erzeugt zwei Bedingungen: Die Auswertung des Testausdrucks mu"s einmal
{\tt {true}} und einmal {\tt {false}} ergeben. In GCT Terminologie
hei"st diese Art der Zweig"uberdeckung {\bf binary branch instrumentation}.


Auch durch das {\tt {switch}}-Konstrukt entstehen verschiedene Programmzweige.
Um alle Zweige zu durchlaufen, m"ussen alle {\tt {case}}-Marken einmal
angesprungen werden. Das gilt auch f"ur die {\tt {default}}-Marke, selbst
wenn sie gar nicht im Quelltext aufgef"uhrt wird. Jede {\tt {case}}-Marke
erzeugt also eine Bedingung. In GCT Terminologie
hei"st diese Art der Zweig"uberdeckung {\bf switch instrumentation}.


\subsubsection*{Schleifen"uberdeckung (GCT: loop instrumentation)}

F"ur dieses Kriterium erzeugt eine {\tt {do-while}}-Schleife zwei Bedingungen:
eine f"ur das einmalige Ausf"uhren des Rumpfes und eine f"ur das mehrmalige
Ausf"uhren. {\tt {for}}- und {\tt {while}}-Schleifen erzeugen noch eine
dritte: der Testausdruck mu"s schon bei der ersten Auswertung {\tt {false}}
ergeben, der Rumpf darf also keinmal ausgef"uhrt werden.

Durch das Schleifen"uberdeckungskriterium werden Fehler in
Schleifen entdeckt, die sich erst bei einer bestimmten Anzahl von Durchl"aufen
bemerkbar machen.


\subsubsection*{Mehrfachbedingungs"uberdeckung (GCT: multiple condition 
instrumentation)}

Mehrfachbedingungen sind aus den Operatoren "`logisches und"' ({\tt {\&\&}}) und
"`logisches oder"' ({\tt {||}}) zusammengesetzte Ausdr"ucke. Zur Erf"ullung
dieses Kriteriums m"ussen beide Seiten eines solchen Ausdrucks einmal
{\tt {true}} und einmal {\tt {false}} werden. Eine Mehrfachbedingung erzeugt
also vier Bedingungen{\footnote{Geschachtelte Ausdr"ucke entsprechend mehr!}}:
linke und rechte Seite des Ausdrucks mu"s je einmal {\tt {true}}
und {\tt {false}} sein.

Bei der "Uberpr"ufung von Mehrfachbedingungen werden nicht nur die
Testausdr"ucke von {\tt {if}}-Anweisungen und Schleifen erfa"st, sondern
alle Ausdr"ucke, deren Auswertung einen booleschen Wert ergeben.


\subsubsection*{Relationaloperator-"Uberdeckung (GCT: operator instrumentation)}

Relationaloperatoren ({\tt {<}}, {\tt {<=}}, {\tt {>}}, {\tt {>=}}) sind oft
die Ursache von Fehlern, sei es aufgrund von Vertauschungen (z. B.
{\tt {<}} und {\tt {>}}), aufgrund falschen Gebrauchs (z. B. {\tt {<}}
statt {\tt {<=}}) oder falscher Grenzwerte (z.B. a < 99 statt a < 100).

Ein Relationaloperator erzeugt drei Bedingungen:

\begin{enumerate}

\item die linke Seite mu"s ungleich der rechten sein

\item die linke Seite mu"s gleich der rechten sein

\item 
	\begin{enumerate}
		\item F"ur die Operatoren {\tt {<=}} und {\tt {>}}:
			die linke Seite mu"s um "`epsilon"' kleiner als die rechte sein
		\item F"ur die Operatoren {\tt {>=}} und {\tt {<}}:
			die linke Seite mu"s um "`epsilon"' gr"o"ser als die rechte sein
	\end{enumerate}
	"`epsilon"' ist in der aktuellen Implementierung '1'.
\end{enumerate}

