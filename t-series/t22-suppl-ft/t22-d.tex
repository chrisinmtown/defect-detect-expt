
Dies ist das Erg\"anzungsblatt bez\"uglich des Programms "`series"'
f\"ur funktionales Testen.

\subsection*{Notwendige Eingaben}

\begin{itemize}

\item Welche Dokumente geh"oren zu dieser Aufgabe?

\begin{enumerate}
\item Dokument TS2, die Spezifikation der Komponente 
\item Dokument TQ2, der Quellcode der Komponente (erhalten Sie, nachdem
	Sie Testf"alle erstellt und Fehlverhalten diagnostiziert haben)
\end{enumerate}

\item Wie hole ich mir das Dateipaket, das ich brauche?

Gehen Sie wie folgt vor:

\begin{enumerate}

\item Zuerst legen Sie ein neues Verzeichnis daf\"ur an mittels des 
"`mkdir"'-Kommandos.
\begin{verbatim}
    mkdir ft-series
\end{verbatim}

\item Dann wechseln Sie in das neue Verzeichnis mittels des 
"`cd"'-Kommandos.
\begin{verbatim}
    cd ft-series
\end{verbatim}

\item Zuletzt geben Sie folgendes Kommando ein:
\begin{verbatim}
    tar xf ~prakt00/Aufgabe5/ft-series.tar
\end{verbatim}

\end{enumerate}

\item Was mu"s vorhanden sein ?

Die folgende Dateien m\"ussen vorhanden sein:
\begin{verbatim}
    Makefile        series          run-suite       test-dir
\end{verbatim}

\end{itemize}

