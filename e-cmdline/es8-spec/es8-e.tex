%
% specification for cmdline
%

\subsection*{Name}
cmdline -- Syntactic and semantic analysis of a command line

\subsection*{Usage}
{\bf cmdline} -mass <MASS> [Search-option] MIL/MDL-File [MIL/MDL-File...]

\subsection*{Description}
{\bf cmdline} analyzes a command line for syntactic and partially for
semantic correctness.
The user must give the program at a minimum one measure and one file
as argument; in addition, exactly one so-called search option may also
be given. 
The order of the measurement options and search options is not
important. 
When the command line is evaluated, all arguments beginning with the
first non-option are treated as file names.  The extistence of such
files is not checked.

If successful (i.e., if the syntax and meaning of the arguments is
found to be legal), a summary of the arguments is printed out,
otherwise an explanatory error message is printed.
The summary consists of the measure to be computed, the search options
if any were given (including a value for -unter and -ueber), and a
list of the MIL/MDL files to be read.

\subsection*{Options}
Options and arguments for options can be shortened to their unique
prefixes.
In the following text, the unique prefix of each option comes before
the square brackets.
A maximum of one search option may be used.

\begin{itemize}

\item {\tt -h[ilfe] }

Help option.  A helpful text is printed and nothing else is done.

\item {\tt -?}

Help option.  See above.

\item {\tt -mas[s] <MASS>}

Acceptable measures are GKO[M], LKO[M], GKH[M], LKH[M], GI[HE], and LI[HE].

\item {\tt -a[lle]}

Search option.  The measure should be evaluated for all units.

\item {\tt -max}

Search option.  Determines the maximum value of the measure for all
units.

\item {\tt -mi[n]}

Search option.  Determines the minimum value of the measure for all
units.

\item {\tt -du[rchschnitt]}

Search option.  Determines the average value of the measure for all
units.

\item {\tt -un[ter] <value>}

Search option.  All units for which the measure lies under the
boundary value will be identified.
The <value> may be a real number.

\item {\tt -ue[ber] <value>}

Search option.  All units for which the measure lies over the boundary
value will be identified.
The <value> may be a real number.

\end{itemize}


\subsection*{Example}

{\small
\begin{verbatim}
% cmdline -mass GKHM -alle datei1
Die Aufgabe ist:
Mass:  GKHM
Suche: -alle
Anzahl der zu lesenden Dateien: 1
Die Dateien sind:
   datei1
\end{verbatim}
}

\subsection*{Authors}
Baumg\"artner, Cla{\ss}en, Gieseke, Lott.
