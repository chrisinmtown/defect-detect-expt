
\begin{enumerate}

\item Fehler in Funktion ``init\_tree\_node'', Zeile 20: 
Parameter "`parent"' wird nicht in der Struktur gespeichert.

\{Omission, Daten\}

F\"uhrt zu Fehlverhalten: Diese Funktion wird vom Kommando sibs (Funktion
"`t\_are\_siblings"') benutzt und wirkt sich wie folgt aus: Sind beide Knoten
vorhanden, dann werden sie generell als Geschwister identifziert. (Da "`parent"'
vom Compiler gcc mit 0 initialisiert wird.)


\item Fehler in Funktion ``find\_node'', Zeile 61:
die Variable i f\"angt bei 1 statt 0 in der for-Schleife an.

\{Comission, Kontrolle\}

F\"uhrt zu Fehlverhalten: Ist der gesuchte Knoten das 
erste Kind eines (Vater-)Knotens, so wird er bei der
Suche nicht betrachtet. Diese Funktion wird von den Kommandos ``child'', 
`search'' und ``sibs'' benutzt und wirkt sich folgenderma"sen aus:
\begin{itemize}
	\item child: Erf"ullt der Vater oben genannte Eigenschaft,
		dann wird kein neuer Knoten eingef"ugt. Es erscheint eine Fehlermeldung,
		da"s der Vaterknoten nicht gefunden wurde, lt. Spezifikation h"atte
		keine Meldung erscheinen d"urfen. (Der Fehler hat zur Folge, 
		da"s alle ersten Kinder Bl"atter bleiben.)
	\item search: Hat der gesuchte Knoten oben genannte Eigenschaft, dann wird
		er nicht gefunden. Es wird keine Meldung ausgegeben, lt. Spezifikation 
		w"are die Meldung "Der Inhalt ist..." zu erwarten
		gewesen.
	\item sibs: Sind beide Knoten Geschwister und der erste 
		erf"ullt oben genannte Eigenschaft, dann erfolgt keine
		Meldung. Lt.~Spezifikation w"are die Meldung "Knoten sind / sind nicht
		Geschwister" zu erwarten gewesen.
\end{itemize}

%
% \item Fehler in Funktion ``t\_add\_child'', Zeile 86: 
% die Fehlermeldung, da"s der Vater-Knoten nicht gefunden wurde, fehlt.
%
% \{Omission, Kontrolle\} 
%
% F\"uhrt zum Fehlverhalten: Es wird keine Fehlermeldung ausgegeben,
% falls beim Einf\"ugen eines Kindes der entsprechende Vater-Knoten
% nicht gefunden wird.  
%

\item Fehler in Funktion ``t\_add\_child'', Zeile 90:
kein Speicher wird f\"ur den zu erweiternden Vater allokiert.

\{Omission, Daten\}

F\"uhrt zum Fehlverhalten:  Segmentation fault (core dumped)
falls ein Vater mehr als 4 Kinder bekommt (aber nicht in allen
F\"allen, Zeigerprobleme sind halt so).


% \item Fehler in Funktion ``t\_add\_child'', Zeile 81 / 99: 
% die Variable rc wird nicht initialisiert, um zu zeigen, da{\ss} die
% Funktion erfolgreich abgelaufen ist. 
% 
% \{Omission, Initialisierung\}
% 
% {\em F\"uhrt zu keinem Fehlverhalten\/}: Die R\"uckgabewerte werden
% nicht \"uberpr\"uft.
% 

\item Fehler in Funktion ``t\_search'', Zeile 118:
das Wort "`Inhalt"' wurde als ``Ihalt'' getippt.

\{Comission, Kosmetisch\}

F\"uhrt zum Fehlverhalten: Falsch buchstabiertes Wort in der Ausgabe. 


\item Fehler in Funktion ``t\_search'', Zeile 121:
Kode zum \"Uberpr\"ufung des R\"uckgabewerts ist unvollst\"andig.

\{Omission, Kontrolle\}

F\"uhrt zum Fehlverhalten: Es wird keine Fehlermeldung 
ausgegeben, falls der Knoten beim Suchen (``search'') nicht gefunden
wird, d.h. nicht existiert (siehe auch Fehler 2).

% \item Fehler in Funktion ``t\_are\_siblings'', Zeile 138:
% die Variable "`key1"' wird statt "`key2"' verwendet. 
% 
% \{Comission, Berechnung\}
% 
% F\"uhrt zum Fehlverhalten: Falls der erste Knoten gefunden wird, werden
% die Knoten als Geschwister erkannt (der zweite Knoten wird nicht "uberpr"uft).


\item Fehler in Funktion ``t\_are\_siblings'', Zeile 140:
die Fehlermeldung, da"s der Knoten nicht gefunden wurde, fehlt.

\{Omission, Kontrolle\}

F\"uhrt zum Fehlverhalten: Es wird keine Meldung 
ausgegeben, falls der erste Knoten nicht existiert. Lt.~Spezifikation
w"are eine Fehlermeldung zu erwarten gewesen.


% \item Fehler in Funktion ``t\_are\_siblings'', Zeile 143:
% die Fehlermeldung, da"s der Knoten nicht gefunden wurde, fehlt.
% 
% \{Omission, Kontrolle\}
% 
% {\em F\"uhrt zu keinem Fehlverhalten\/}: Da nicht nach dem zweiten Knoten
% gesucht wird (Fehler 8) kann auch nicht der Fall auftreten, da"s er nicht
% gefunden wurde.
% Es wird keine Fehlermeldung 
% ausgegeben, falls beim Geschwister-Check der rechte Knoten nicht
% gefunden wird (der zweite Parameter im ``sibs''-Kommando).


% 
% \item Fehler in Funktion ``print\_tree\_nodes'', Zeile 154:
% die Variable "`rc"' wird nicht initialisiert
% 
% \{Omission, Initialisierung\}
% 
% Alternativ: Zeile 153, 161, 162: "`rc"' ist \"uberfl"ussig und der
% Kode kann gestrichen werden.
% 
% \{Comission, Kontrolle\}
% 
% F\"uhrt zum Fehlverhalten: Der Baum wird nur bis zum ersten Blatt ausgegeben.
% Da das erste Blatt aufgrund von Fehler 2 das erste Kind von Root ist, werden
% maximal zwei Baumeintr"age ausgegeben: Root und sein erstes Kind.
% 

\item Fehler in Funktion ``print\_tree\_nodes'', Zeile 164:
die printf-Funktion gibt nur zwei Leerzeichen aus.

\{Comission, Kosmetik\}

F\"uhrt zum Fehlverhalten: Die Ebenen des Baumes werden nur um jeweils 2 statt
4 Leerzeichen einger"uckt.
 

\item Fehler in Funktion ``t\_print'', Zeile 180:
die Funktion wird mit 1 statt 0 aufgerufen.

\{Comission, Schnittstelle\}

F\"uhrt zum Fehlverhalten:  In der Ausgabe ist f"ur die
Wurzel die Ebene 1 statt 0 angegeben.

\end{enumerate}

Und einige Bemerkungen:

\begin{enumerate}

\item Bemerkung zu Funktion ``t\_are\_siblings'', Zeile 139: die
zwei Knoten werden nicht mit einander verglichen, um festzustellen, ob
ein und derselbe Knoten zweimal uebergegeben wurde.  Dies ist auch
kein Fehler, da die Spezifikation klar sagt, da{\ss} jeder Knoten sein 
eigener Geschwister ist.

\item Bemerkung zu Funktion ``fuehre\_kommandos\_aus'', Zeile 201/213: die
L\"ange wird nur f\"ur die Kommandozeile beachtet, nicht f\"ur die
einzelnen Teile.

\item Bemerkung zu Funktion ``fuehre\_kommandos\_aus'', Zeile 212, die
Kommandos werden nicht \"uberpr\"uft, ob ``root'' als erstes vorkam.
Dies ist ein Fehler aber es wurde auch klar gesagt, da{\ss} die
Test-Umgebung nicht zu testen war.

\end{enumerate}

