%
% instructions for structural testing
%

Please follow the directions on the supplemental sheet entitled 
``Structural testing with program $x$,'' which is included in this
packet of materials. 

\begin{enumerate}

\item Before beginning, read over the data-collection form ``E32:
Structural testing.'' 
Enter your identifier and fill out the section entitled ``Before beginning.''

\end{enumerate}

\subsection*{Produce test data}

\begin{enumerate}
\addtocounter{enumi}{1}

\item Enter the time when you began the experiment under question~4.

\item Briefly read through the code to gain an overview of this component.  
If you happen to notice faults during this or one of the following
steps, mark them.  However, please do not invest any time in a precise
analysis; i.e., don't isolate the faults.  Isolation of faults begins
in step 18. 

\item Begin to generate test data that will lead to 100\% coverage of
the following criteria:  branch coverage, loop coverage,
multiple condition coverage, and relational operator coverage.  
If you feel that you do not completely understand these criteria,
please turn to the end of these instructions and read the appendix
about coverage criteria.
Special properties of the component are noted on the supplemental sheet.
Think carefully in this step in order to avoid time-consuming 
``test-data hacking'' later.

\item After you are finished with generating test data, enter the time
you required under question~5.

\item Log yourself in on a computer/Xstation.  Enter the computer name
and time under question~6.
Fetch the necessary files (see the supplemental sheet).
Convince yourself that all necessary files and directories are there!

\item Type in your test cases in files under the directory ``test-dir''.
In order to make your task easier, the component has been equipped
with a command-line interface.
This interface is used by an automatic tester.  The tester reads the
parameter files and invokes the component with the parameters
specified in the files.
The results are written into a file.
A complete test of the component is accomplished in this way.

You must create the parameter and (when necessary) input files.
A test case $=$ a parameter file.
The expressions in a parameter file specify one invocation of the
program for a run of the automatic tester.
Parameter files are named with ``.test'' as the file suffix.
Example: If the component ``comp'' should be invoked as \mbox{\tt comp -i
in\_file1}, then the expression \mbox{\tt -i in\_file1} should be entered
into the parameter file.

Input files are only created for components which read from files.
Input files should be named sensibly, for example ``empty,'' but the
name may not use the suffix ``.test'' at all.

Convince yourself that the files hold what you really want them to
hold!

\end{enumerate}

\subsection*{Run tests}

\begin{enumerate}
\addtocounter{enumi}{7}

\item Apply the test cases to the component by typing in the command 
``run-suite'' and watching. 

\item Look at the summary of coverage with the help of the commands 
``gsummary'' und ``greport''.  For example:
\begin{verbatim}
% gsummary test-dir/GCTLOG
% greport  test-dir/GCTLOG
\end{verbatim}

\item Try to bring all coverage values up to 100\% by generating
further tests. 
You can add new tests in directory ``test-dir''.
After you have changed the test data, run the tests again by issuing
the following sequence of commands:
\begin{verbatim}
% make clean 
% run-suite 
% gsummary test-dir/GCTLOG 
% greport  test-dir/GCTLOG
\end{verbatim}

\item After you have reached coverage of 100\%, or you have convinced
yourself that various reasons prevent you from ever reaching 100\%
coverage (the reasons had better be good), you are done with the first
part of this exercise.

\item Enter the number of test cases that you generated in this
exercise under question~7 and the coverage values which you attained
under question~8.
Also please enter the time you required under question~9.

\item The results are in individual files under the directory
``test-dir'' and are also summarized in a file ``test-results.summary''.
Print out the summary on the line printer {\em twice}.

\item Log out and fetch your output from the printer.

\item Note on one copy of the print-out of your results your
identifier, which test method you worked with, and in which directory
your test cases can be found.  Give this print-out to the experiment
leader; in return, you will receive the specification.  Please do not
delete your test cases!

\end{enumerate}

\subsection*{Search for failures}

\begin{enumerate}
\addtocounter{enumi}{15}

\item Enter the current time under question~10.
Read the specification and search for possible failures by comparing
the expected results according to the specification with the output of
your test cases. 
Mark the detected failures in the output with circles, etc.
Enter the detected failures in the worksheet supplied for that purpose,
and don't forget to enter your identifier on that sheet as well.
Please number the failures which you found from 1 to $n$ in the
column labeled ``Fail. Nr.'' (failure number) in the worksheet for
failures and similarly number the failures which you marked in the 
output.

\item When you believe that you have detected all failures, please
enter the time you required under question~11.

\end{enumerate}

\subsection*{Isolate faults}

\begin{enumerate}
\addtocounter{enumi}{17}

\item Enter the current time under question~12.
Try to isolate the faults responsible for the failures which you
detected.
Enter the isolated faults in the worksheet  provided for that purpose,
and again don't forget to enter your identifier on that sheet.

Record the faults which you isolated by using the line number.
Classify the faults according to the different types and classes of
faults [\{Omission, Comission\}, \{Initialization, Control,
Interface, Data, Computation, Cosmetic\}]. 
Then characterize the fault briefly; e.g., ``initial value was 1 instead
of 0.''
Also give the number(s) of the failure(s) which resulted from this fault.

If you by chance recognize any faults that did not manifest themselves
in failures in your test results, you may list them also.  But please
mark clearly on the worksheet that these faults did not cause any
failures in your test results. 
These faults receive just a dash in the column ``Fail. Nr.''

\item When you are finished isolating faults, enter the time you
required under question~13.

\item Please fill out the rest of the form (questions~14--16).

\item Give the data-collection form and all other worksheets to the
experiment leader, and you're all done.

\end{enumerate}

\section*{GCT Coverage Criteria}

This appendix discusses the coverage criteria supported by GCT, the
Generic Coverage Tool.

\subsection*{Branch coverage}

Fulfilling the branch coverage criteria makes certain that each branch
in a component was executed at least once.  Branches in C programs are
created by  {\tt if}, {\tt ?}, {\tt for}, {\tt while}, and {\tt
do-while} statements.  Each of these statements creates two
conditions (branches): the evaluation of the test expression must
yield true once and false once.  In GCT terminology, this type of
branch coverage is called {\bf binary branch instrumentation}.

Branches are also created by the {\tt switch-case} construct.
To test all branches, all {\tt case} labels must be branched to at
least once.  This includes the {\tt default} label, even if it was not
explicitly written in the source code.  Each {\tt case} label produces
a single condition.  In GCT terminology, this type of branch coverage
is called {\bf switch instrumentation}.

\subsection*{Loop coverage}

A {\tt do-while} loop creates two conditions for this criteria: one
 one for executing the loop exactly one time, and one for executing
the loop more than one time. 
The {\tt for} and {\tt while} loop add a third condition: the idea of
executing the loop zero times; i.e., the test expression is false when
it is evaluated for the first time.

These loop coverage criteria reveal faults in loops that reveal
themselves only after a certain number of passes through the loop.


\subsection*{Multiple condition coverage}

Multiple conditions are expressions constructed using the operators
``logical and'' ({\tt \&\&}) and/or ``logical or''~({\tt{||}}).  To
fulfill these criteria for a two-part expression (i.e., an expression
with exactly one logical operator), each side of the expression must
be true once and false once.  In other words, a two-part
multiple-condition expression creates four conditions: false/false,
true/false, false/true, and true/true.  Nested expressions create
correspondingly more conditions.

When checking the coverage of multiple conditions, all expressions
that evaluate to a boolean value are checked, not only the test
expressions from {\tt if} statements.

\subsection*{Relational operator coverage}

The relational operators ({\tt <}, {\tt <=}, {\tt >}, {\tt >=}) 
are often the cause of faults, whether because of swapped
operators (e.g., {\tt <} instead of {\tt >}), incorrect use
(e.g., {\tt <} instead of {\tt <=}), or incorrect boundary
values (e.g., ``{\tt a < 99}'' instead of ``{\tt a < 100}'').

A relational operator creates three conditions:

\begin{enumerate}

\item The left side must not equal the right side.

\item The left side must equal the right side.

\item And for scalar values (integer, char):
        \begin{enumerate}
                \item for the operators {\tt <=} und {\tt >}:
                       the left side must be 1 smaller than the right side
                \item for the operators {\tt >=} und {\tt <}:
                       the left side must be 1 larger than the right side
        \end{enumerate}
\end{enumerate}
