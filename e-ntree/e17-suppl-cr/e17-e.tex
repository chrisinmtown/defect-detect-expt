
This is the supplemental sheet 
for applying technique ``code reading''
to the program ``ntree.''

\subsection*{Brief description of the library functions used}

\begin{itemize}

\item {\tt assert(expression)}

Macro that expects the expression to be true at the time the
program is executed and in that case does nothing.  Otherwise an error
message with the expression is printed and the program is ended.

\item {\tt char* malloc(unsigned size)}

Returns a pointer to a block of memory that is at least as large as
{\tt size} bytes.  In case of failure, NULL is returned.

\item {\tt int free(ptr)}

Gives a previously allocated block back to the system.  
{\tt ptr} must have been allocated using {\tt malloc} or a similar
function.  If successful, 1 is returned; otherwise 0 is returned.

\item {\tt int strcmp(s1,s2)}

Compares two strings.  Returns a value greater than, equal to, or less
than 0 depending on whether {\tt st1} is lexicographically (i.e.,
ASCII value) greater than, equal, or less than {\tt s2}.

\item {\tt void bcopy(b1,b2,length)}

Copies {\tt length} bytes from string {\tt b1} to string {\tt b2}.
Overlapping strings are treated correctly.

\item {\tt int sscanf(s,format,pointer)}

Reads characters from string {\tt s}, converts them according to
the specification string {\tt format}, and writes the result into
the variables pointed to by the {\tt pointer}s.
This is the reverse function of {\tt printf}; the format specification
strings are identical.

\item {\tt char *fgets(s,n,stream)}

Reads characters from the stream {\tt stream} and writes them in the
string {\tt s}, until either {\tt n-1} characters have been read or
one of NEWLINE or EOF are seen.
At EOF the value NULL is returned, otherwise {\tt s}.

\item {\tt char *strdup(s1)}

Duplicates the string {\tt s1}.  Allocates memory for this purpose
using {\tt malloc()} and returns a pointer to the duplicate if
successful, otherwise NULL.

\end{itemize}

\subsection*{Reminder}

Don't produce any abstractions for the test-scaffold functions.
