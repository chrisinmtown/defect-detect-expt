\centerline{VALIDATION}

\bigskip

\begin{itemize}

\item The goal of validation is to judge the quality of the software
from the user's point of view; e.g., reliability.

\item The goal of all validation techniques is:

\begin{itemize}
\item to reveal failures,
\item to localize the faults that caused the failures, and
\item ultimately, to correct the faults,
\end{itemize}

and thereby to achieve the highest possible confidence that the
component conducts itself according to its specification.

\item Validation can be considered an activity to improve the 
software and to evaluate the software.

However, the quality of poorly understood software can't be improved
via validation.

\end{itemize}



\newpage
\centerline{DIFFERENT TECHNIQUES}
\bigskip


\begin{list}{$\bullet$}{\labelwidth=1.5cm\leftmargin=\labelwidth}

\item[(A)] Execution-based validation (testing)

\begin{list}{$\bullet$}{\labelwidth=1.5cm\leftmargin=\labelwidth}

\item [(A1)] Black-box (functional testing)

Use the specification to develop test cases
(the code is invisible):
\begin{list}{$\bullet$}{\labelwidth=1.5cm\leftmargin=\labelwidth}
\item[(A11)] by constructing equivalence classes 
\item[(A12)] by performing boundary-value analysis

(ideally using both A11 and A12 together)
\end{list}

\item[(A2)] White-box (structural testing)

Use source code to develop test cases (the specification may be invisible)
in order to achieve:
\begin{list}{$\bullet$}{\labelwidth=1.5cm\leftmargin=\labelwidth}
\item[(A21)] statement coverage
\item[(A22)] control-path coverage
\item[(A23)] control-construct covering 
\item[(A24)] multiple-condition coverage

(...and many more)
\end{list}

\end{list}

\item [(B)] Non-execution-based validation (reading)

Really ``verification,'' this is presented here 
as a contrasting example to the execution-based techniques.

\begin{list}{$\bullet$}{\labelwidth=1.5cm\leftmargin=\labelwidth}
\item [(B1)] symbolic execution
\item [(B2)] reading

\begin{list}{$\bullet$}{\labelwidth=1.5cm\leftmargin=\labelwidth}
\item[(B21)] sequential
\item[(B22)] control-flow oriented
\item[(B23)] stepwise abstraction
\end{list}

\end{list}

\end{list}

\newpage

\centerline{TESTING}         % 3.2.6.2.
\addcontentsline{toc}{section}{Testing}
\bigskip

Goal: to define a finite set T of tests (T $\subseteq$ input x output)
where:
\begin{itemize}

\item the probability of revealing all failures is high
\item the belief that all failures were revealed is strong

\end{itemize}

There are different criteria:
\begin{itemize}
\item for choosing tests
\item for deciding about the ``completeness'' of the tests 

(i.e., when to stop testing)
\end{itemize}

Each test is specified through a pair of $T_i$:(TF, TD) where:
\begin{itemize}

\item TF represents the test case 

(i.e., a possible sequence of invocations of the component)

% Remark: Only one test case exists for components of type I.

\item TD represents the test data

(i.e., pairs (i $\in$ input, o $\in$ output))

Remark: o describes the expected result, according to the 
specification, for the input i.

\end{itemize}


\newpage
\centerline{TESTING Continued}         % 3.2.6.2.
\bigskip

Test Results:

\begin{itemize}

\item $T_i$: ($o \in$ output, $o' \in$ [Pgm](i))

\item $o$: expected result

\item $o'$: actual result

Remark: when testing is not done systematically, mismatches between
$o$ and $o'$ are frequently overlooked.

\end{itemize}


Documentation of a test case:
\begin{itemize}

\item $T_1$: comment (test's purpose)
\subitem Test case: $f_1$
\subitem Test data (input, expected result)
\subitem Test result (actual result)

\end{itemize}

Myers: A successful test case is one that causes a failure!


\newpage
\centerline{BLACK-BOX TESTING}

\bigskip


Example:
\begin{tabbing}
$ (x \ne 0 \rightarrow $ \= $( z > 1 \rightarrow y, z := 5, z/x \,|\, y, z := 5,0) \,|\,$ \\
                         \> $( z > 1 \rightarrow z := (z-x) / x \,|\, z := 0))$ \\
\end{tabbing}

Develop test data

\begin{enumerate}

\item[1.] Identify equivalence classes

Divide the inputs into equivalence classes for which
identical behavior is expected from the component.
\begin{enumerate}
\item $x \ne 0 \wedge z >    1$
\item $x \ne 0 \wedge z \leq 1$
\item $x  =  0 \wedge z >    1$
\item $x  =  0 \wedge z \leq 1$
\end{enumerate}

\end{enumerate}

\newpage
\centerline{BLACK-BOX TESTING Continued}

\bigskip

Develop test data continued:

\begin{enumerate}

\item[2.] Choose data points

\begin{itemize}
\item Using equivalence classes:

Use one (or more) tests per equivalence class.

\{(x=1, z=4), (x=1,z=0), (z=0,z=4), (x=0,z=0)\}

\item Using boundary value analysis:

Give special treatment to boundary values (boundaries
of equivalence classes).

\begin{enumerate}
\item \{($-\infty$,inc(1)),($+\infty$,inc(1)),($-\infty,+\infty$),($+\infty,+\infty$)\}
\item \{($-\infty$,1),($+\infty$,1),($-\infty,-\infty$),($+\infty,-\infty$)\}
\item \{(0, inc(1)), (0, $+\infty$)\}
\item \{(o, 1), (0, $-\infty$)\}

\end{enumerate}

Note: $+/-\infty$ corresponds to the largest/smallest representable number, 
inc(x) to the smallest number representable that is larger than x.


\end{itemize}

\end{enumerate}


\newpage
\centerline{PRINCIPLES OF THE BLACK-BOX TESTING PROCESS}
\bigskip

\begin{enumerate}
\item Detect failures

\begin{itemize}

\item Define the tests (TF, TD) using the component's specification:
\begin{itemize}
\item identify test cases 
\item identify test data
\end{itemize}

Criteria for completeness of the tests are:
\begin{itemize}
\item at least one test is defined for each equivalence class
\item at least one test is defined for every vague point in the
specification 

\end{itemize}

\item The program is executed for the input part of each test
(to obtain an actual result).

\item Failures are diagnosed in the output by comparing the expected
result with the actual result.

\end{itemize} 

\item Isolate faults
\begin{itemize}
\item Search for the cause (i.e., the fault in the code) of the
detected failure by reading/debugging.
\end{itemize}

\end{enumerate}


\newpage
\centerline{REQUIRED DOCUMENTS FOR BLACK-BOX TESTING}
\bigskip

\begin{center}
\begin{tabular}{|l||c|c|c|}
\hline
                 & Specification & Source code & executable\\
Steps            &     $f$       & $pgm$       & component\\
\hline
\hline
Generate         &               &             &       \\
test             &      X        &             &       \\
cases            &               &             &       \\
\hline
Execute          &               &             &       \\
test             &               &             &     X \\
cases            &               &             &       \\
\hline
                 &               &             &       \\
Diagnosis        &      X        &             &     X \\
                 &               &             &       \\
\hline
isolate          &               &             &       \\
faults           &      X        &      X      &    X  \\
                 &               &             &       \\
\hline
\end{tabular}
\end{center}





\newpage
\centerline{WHITE-BOX TESTING} 

\bigskip

\begin{minipage}[b]{8cm}
\begin{verbatim}
01:  if x != 0 then
02:      y := 5;
03:  else
04:      z := z - x;
05:  endif
06:  if z > 1 then
07:      z := z / x;
08:  else
09:      z := 0;
10:  endif
\end{verbatim}
\end{minipage}
\includegraphics[scale=1]{white-box} 



\newpage
\centerline{WHITE-BOX TESTING Continued} 

\bigskip

Test data (for source code)

\begin{enumerate}

\item statement coverage

Select tests to execute each statement at least once

\{(x=0, z=1), (x=1, z=3)\}

\item control-path coverage

Select tests to traverse each edge of the program's control-flow graph
at least once

\{(x=0, z=1), (x=1, z=3)\}

\item complete control-path coverage

Select tests to traverse each elementary path at least once

\{(x=0, z=1), (x=1, z=3), (x=0, z=3), (x=1, z=1)\}

\end{enumerate}

\newpage
\centerline{REFINEMENT OF COVERAGE CRITERIA}
\bigskip

Multiple-condition coverage

\begin{itemize}

\item In the case of combined boolean conditions\\
(e.g., a $\wedge$ b), make sure that all cases are tested:

\begin{center}
\begin{tabular}{p{2cm}|p{2cm}||p{2cm}}
\hline
a & b & a$\wedge$b \\
\hline
\hline
t & t & t \\
t & f & f\\
f & t & f\\
f & f & f\\
\hline
\end{tabular}
\end{center}

\item In the case of comparison operators (e.g., a $\leq$ b), make
sure that both possibilities are tested (a = b, a $<$ b).

\item In the case of loops (e.g., while $<$expr$>$ do S end), make sure
that the loop is executed 0 times, 1 time, and n $>$ 1 times.

\end{itemize}

\newpage
\centerline{PRINCIPLES OF THE WHITE-BOX TESTING PROCESS}
\bigskip

\begin{enumerate}

\item Detect failures

\begin{itemize}

\item Define the tests (TF, TD) using the source code
\begin{itemize}
\item identify test cases 
\item identify test data
\end{itemize}

Criteria for completeness of the tests are:

\begin{itemize}
\item According to the ``statement-coverage'' approach:

(\# XEQ statements / \# existing statements ) =  1

\item According to the ``control-path-coverage'' approach:

(\# XEQ paths / \# existing paths) = 1

$\mu$(G) := \# paths -- \# nodes + 2

\end{itemize}

\item The program is executed for the input part of each test
(to obtain an actual result)

\item Failures are diagnosed in the output by comparing the expected
result with the actual result

\item Coverage values attained are checked with a support tool.

\end{itemize}


\item Isolate faults

\begin{itemize}

\item Search for the cause (i.e., the fault in the code) of the
detected failure by reading/debugging

\end{itemize}

\end{enumerate}

\newpage
\centerline{REQUIRED DOCUMENTS FOR WHITE-BOX TESTING}
\bigskip

\begin{center}
\begin{tabular}{|l||c|c|c|}
\hline
                 & Specification & Source code & executable\\
Steps            &     $f$       & $pgm$       & component\\
\hline
\hline
Generate         &               &             &       \\
test             &               &     X       &       \\
cases            &               &             &       \\
\hline
Execute          &               &             &       \\
test             &               &     (X)     &     X \\
cases            &               &             &       \\
\hline
                 &               &             &       \\
Diagnosis        &      X        &     (X)     &     X \\
                 &               &             &       \\
\hline
isolate          &               &             &       \\
faults           &      X        &      X      &    X  \\
                 &               &             &       \\
\hline
\end{tabular}
\end{center}


\newpage
\centerline{READING VIA STEPWISE ABSTRACTION} % 3.2.6.1. 
\addcontentsline{toc}{section}{Reading via Stepwise Abstraction}
\bigskip


Example: 

\begin{verbatim}
01:  if x != 0 then
02:      y := 5;
03:  else
04:      z := z - x;
05:  endif
06:  if z > 1 then
07:      z := z / x;
08:  else
09:      z := 0;
10:  endif
\end{verbatim}

Develop abstractions of this code using Mill's functional notation.

\bigskip

Abstraction lines 01--05:  

\(x \ne 0 \rightarrow y := 5 \,|\, z := z - x\)

\bigskip

Abstraction lines 06--10:  

\(z > 1 \rightarrow z := z / x \,|\, z := 0\)

\bigskip

Abstraction lines 01--10:
\begin{tabbing}
$ (x \ne 0 \rightarrow $ \= $( z > 1 \rightarrow y, z := 5, z/x \,|\, y, z := 5,0) \,|\,$ \\
                         \> $( z > 1 \rightarrow z := (z-x) / x \,|\, z := 0))$ \\
\end{tabbing}

% Clearly documentation: ......        


\newpage


\centerline{PRINCIPLES OF THE CODE-READING PROCESS}
\bigskip

\begin{enumerate}

\item Detect how the program would fail if run 

\begin{itemize}

\item Determine the meaning (behavior) of a component through
reading. 

$\rightarrow [pgm]$

\begin{itemize}


\item define the meaning of every elemental program part, and describe
it as a conditional instruction.

Remark: conditions and parallel instructions should be described
as formally as possible. 

\item aggregate the elemental functions according to the control flow.

\end{itemize}

\item Compare the abstract meaning function with the given
specification to detect possible failures.

$\rightarrow$ is $ f \subseteq [pgm]$ ?

Remark: if specifications for single design parts exist, 
the diagnosis can proceed in a stepwise fashion.

\end{itemize}

\item Isolate the faults that would lead to failures

\begin{itemize}

\item Search out the cause of the inappropriate behavior (i.e., the fault)
in the code.

Remark: The deep understanding of the code gained in the reading
step should ease fault localization.

\end{itemize}


\end{enumerate}

\newpage
\centerline{REQUIRED DOCUMENTS FOR CODE READING}
\bigskip

\begin{center}
\begin{tabular}{|l||c|c|c|}
\hline
               & Specification & Source code & executable\\
Steps          &     $f$       & $pgm$       & component\\
\hline
\hline
1. detect likely &             &             &      \\
failures       &               &      X      &      \\
($\rightarrow [pgm]$) &        &             &      \\
\hline
Compare        &               &             &      \\
($f\subseteq [pgm]?)$&   x     &      X      &      \\
               &               &             &      \\
\hline
2. isolate     &               &             &      \\
faults         &      X        &      X      &      \\
               &               &             &      \\
\hline
\end{tabular}
\end{center}
