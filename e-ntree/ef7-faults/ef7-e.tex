
%
% faults in program ntree
%

The classification of each fault is given in braces.

\begin{enumerate}

% 1  (was 1)
\item Fault in function ``init\_tree\_node'', line 20: 
Parameter ``parent'' is not stored in the structure.

\{Omission, Data\}

This function is used by the command ``sibs'' and causes 
the following failure.  If both nodes used in the ``sibs'' command
exist, then they are recognized as siblings, because the value 
of the parent field is apparently initialized to 0 at run-time.

% 2  (was 2)
\item Fault in function ``find\_node'', line 62:
The variable i starts at 1 instead of 0 in the for-loop.

\{Comission, Control\}

Causes failure: Nodes in the subtree of the father node's first
(aka leftmost) child are not found during the search.  I.e., 
the leftmost subtree of the root is not searched, and so forth.
Causes problems for commands ``child,'' ``search,'' and ``sibs'' 
as follows: 

\begin{itemize}

\item child: if the father node meets the criteria stated above,
then no new node is inserted into the tree. 
An unexpected error message is printed that the father node could not
be found (a failure of the search function); the message is unexpected
because an existing node should have been found.

\item search: if the node meets the criteria stated above, it is not
found and no message is printed (not even an error message).

\item sibs: if the nodes are siblings and the first meets the criteria
stated above, then no message is printed (not even an error message).

a message is printed stating that they are not
siblings. 

\end{itemize}


%    (was 3)
% \item Fault in function ``t\_add\_child'', line 86: 
% The error message about not finding the father node is missing. 
% 
% \{Omission, Control\} 
% 
% Causes failure: No error message is printed if the father node cannot
% be found while trying to insert a child.
% 

% 3  (was 4)
\item Fault in function ``t\_add\_child'', line 91:
No memory is allocated for the father node that is to be extended.

\{Omission, Data\}

Causes failure:  Segmentation fault (core dumped)
if a father node receives more than 4 children (but not in all cases;
pointer problems are just like that).

%    (was 5)
% \item Fault in function ``t\_add\_child'', line 81 / 99: 
% the variable rc is not initialized in order to state that the function
% ran successfully.
% 
% \{Omission, Initialization\}
% 
% Causes no failure. The return values are not checked.
% 

% 4  (was 6)
\item Fault in function ``t\_search'', line 119:
The word ``Inhalt'' was incorrectly typed as ``Ihalt''.

\{Comission, Cosmetic\}

Causes failure: Misspelled word in the output.

% 5  (was 7)
\item Fault in function ``t\_search'', line 122:
Code for checking the return values is incomplete.

\{Omission, Control\}

Causes failure: No error message is printed if the node cannot be
found (command ``search''; see also fault 2).

%    (was 8)
% \item Fault in function ``t\_are\_siblings'', line 138:
% The variable ``key1'' is used instead of ``key2''.
% 
% \{Comission, Computation\}
% 
% Causes failure: If the first node is found, the second is ignored and
% the two are reported to be siblings.

% 6  (was 9)
\item Fault in function ``t\_are\_siblings'', line 140:
The error message about not finding the nodes is missing.

\{Omission, Control\}

Causes failure: No error message is printed if the left node (first
argument of command ``sibs'') is not found while checking for
siblings.  An error message was expected.

%    (was 10)
% \item Fault in function ``t\_are\_siblings'', line 143:
% The error message about not finding the nodes is missing.
% 
% \{Omission, Control\}
% 
% Causes no failure.  Because the program never searches for the second
% node (see fault 8), the case in which the node is not found never can
% occur. 
% If not for fault 8, the failure would be that no error message is
% printed if the right node (second argument of command ``sibs'') is not
% found while checking for siblings.

%    (was 11)
% \item Fault in function ``print\_tree\_nodes'', line 154:
% The variable ``rc'' is not initialized.
% 
% \{Omission, Initialization\}
% 
% Alternative: Line 153, 161, 162: The variable rc is superfluous and
% the code can be deleted.
% 
% \{Comission, Control\}
% 
% Causes failure: The tree is not printed in its entirety; only nodes up
% to the first leaf are printed.  Because fault 2 causes the leftmost
% child of the root to remain a leaf node, a maximum of two entries are
% printed. 

% 7  (was 12)
\item Fault in function ``print\_tree\_nodes'', line 165:
The printf invocation only prints two (2) blanks.

\{Comission, Cosmetic\}

Causes failure:  Each level in the tree is only indented 2 instead 
of 4 additional spaces. 

% 8  (was 13)
\item Fault in function ``t\_print'', line 181:
The function is called with 1 instead of 0.

\{Comission, Interface\}

Causes failure:  The root is marked as level 2 in the output instead
of level 0.

\end{enumerate}

And a few comments:

\begin{enumerate}

\item Comment about function ``t\_are\_siblings'', Line 139:  The two
nodes are not compared with each other to test whether one and the
same node was passed to the function twice.  This is not a fault
because the specification has nothing to say about this issue.

\item Comment about function ``fuehre\_kommandos\_aus'', line 201/213: 
The length is only checked for the command line, not for the
individual commands.  This deviates slightly from the specification.

\item Comment about function ``fuehre\_kommandos\_aus'', line 213:
The commands are not checked to see if ``root'' is the first one.
This might be seen as a fault, but it was clearly stated that the test
scaffold was not especially robust and should not be tested.

\end{enumerate}
