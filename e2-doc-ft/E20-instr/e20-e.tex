
Please follow the directions on the supplemental sheet entitled 
``Functional testing with program $x$,'' which is included in this
packet of materials. 

\begin{enumerate}

\item Before beginning, read over the data-collection form 
``E22: Functional Testing.'' 
Enter your identifier and fill out the section entitled ``Before beginning.''

\end{enumerate}

\subsection*{Produce test data}

\begin{enumerate}
\addtocounter{enumi}{1}

\item Enter the time when you began the experiment under question~4.

\item Read through the specification carefully.
Use it to derive the equivalence classes.
Enter the number of equivalence classes under question~5 in the
data-collection form, and please answer question~6 at the same time.

\item Generate test cases by choosing boundary values from the
equivalence classes.
While doing so, ignore any system (not program) dependencies such as
the upper limit for the number of command-line etc.~(see also the
supplemental sheet). 
When you are finished, enter the number of test cases under question~7
and the time you required under question~8 of the data-collection form.

\item Log yourself in on a computer/Xstation.  Enter the computer name
and time under question~9.
Fetch the necessary files (see the supplemental sheet).
Convince yourself that all necessary files and directories are there!

\item Type in your test cases in files under the directory ``test-dir''.
In order to make your task easier, the component has been equipped
with a command-line interface.
This interface is used by an automatic tester.  The tester reads 
parameter files and invokes the component with the parameters
specified in the files.
The results are written into a file.
A complete test of the component is accomplished in this way.

You must create the parameter and (when necessary) input files.
A test case $=$ a parameter file.
The expressions in a parameter file specify one invocation of the
program for a run of the automatic tester.
Parameter files are named with ``.test'' as the file suffix.
Example: If the component ``comp'' should be invoked as {\tt comp -i
in\_file1}, then the expression {\tt -i in\_file1} should be entered
into the parameter file.

Input files are only created for components which read from files.
Input files should be named sensibly, for example ``empty,'' but the
name may not use the suffix ``.test'' at all.

Convince yourself that the files hold what you really want them to
hold!

\end{enumerate}

\subsection*{Run tests}

\begin{enumerate}
\addtocounter{enumi}{6}

\item Apply the test cases to the component by typing in the command 
``run-suite'' and watching. 
If you made some typing mistakes which caused some test cases not to
test quite what you wanted to test, correct the mistakes, type ``make
clean'', and run the command ``run-suite'' again.
But please do not generate additional test data!

\item If you took any breaks while typing in or executing your test
cases, please enter just the time you spent actively typing or
executing test cases under question~10 (i.e., the time without
pauses).

\item The results are in individual files under the directory
``test-dir'' and are also summarized in a file ``test-results.summary''. 
Print out the summary on the line printer {\em twice}.

\item Log out and fetch your output from the printer.

\end{enumerate}

\subsection*{Search for failures}

\begin{enumerate}
\addtocounter{enumi}{10}

\item Look over the results carefully.  Enter your starting time under
question~11.
Find possible failures by comparing the expected results according to
the specification with the output of your test cases.
Mark the detected failures in both copies of the output with circles,
etc. 
Enter the detected failures in the worksheet supplied for that purpose,
and don't forget to enter your identifier on that sheet as well.
Please number the failures which you found from 1 to $n$ in the
column labeled ``Fail.~Nr.'' (failure number) in the worksheet for
failures and similarly number the failures which you marked in the 
output.

\item When you believe that you have detected all failures, please
enter the time you required under question~12.

\item Note on the print-out of your results your identifier, which
test method you worked with, and in which directory your test cases
can be found.  Give this print-out to the experiment leader; in
return, you will receive the source code.  Please do not delete your
test cases!

\end{enumerate}

\subsection*{Isolate faults}

\begin{enumerate}
\addtocounter{enumi}{13}

\item Please enter the current time under question~13.
Try to isolate the faults responsible for the failures which you
detected.
Enter the isolated faults in the worksheet provided for that purpose,
and again don't forget to enter your identifier on that sheet.

Record the faults which you isolated by using the line number.
Classify the faults according to the different types and classes of
faults [\{Omission, Comission\}, \{Initialization, Control,
Interface, Data, Computation, Cosmetic\}]. 
Then characterize the fault briefly; e.g., ``initial value was 1 instead
of 0.''
Also give the number(s) of the failure(s) which resulted from this fault.

If you by chance recognize any faults that did not manifest themselves
in failures in your test results, you may list them also.  But please
mark clearly on the worksheet that these faults did not cause any
failures in your test results. 
These faults receive just a dash in the column ``Fail. Nr.''

\item When you are done isolating faults, enter the time you
required under question~14.  Please enter the current time under
question~15 also. 

\item Enter your personal estimate under question~16, and fill out 
question~17.

\item Give the data-collection form and all other worksheets to the
experiment leader, and you're all done.

\end{enumerate}
