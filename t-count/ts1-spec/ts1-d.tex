
\subsection*{Name}

count -- Zeilen, Woerter und Zeichen zaehlen

\subsection*{Verwendung}

{\bf count} Eingabedatei [ Eingabedatei ... ]

\subsection*{Beschreibung}

{\it count} z\"ahlt die Anzahl der Zeilen, W\"orter und Zeichen in
den angegeben Dateien.
W\"orter sind Zeichenfolgen, getrennt durch ein oder mehrere
Leerzeichen, Tabulatoren oder Zeilenumbr\"uche (Carriage Return).

Ist eine Datei im Dateisystem nicht vorhanden, soll eine
entsprechende Fehlermeldung ausgegeben werden, mit der Bearbeitung der
anderen Dateien soll jedoch fortgefahren werden. Ist keine Datei
angegeben, so wird von der Standard-Eingabe gelesen.

Die ermittelten Werte sind sowohl fuer jede Datei einzeln anzugeben
(inkl. Name bei durch Parameter von `count' spezifizierten Dateien),
als auch als Summe der gesamten Werte.  Wird nur eine Datei bzw. nur
die Standard-Eingabe bearbeitet, wird dann keine Summe
ausgedruckt. Die Reihenfolge der Ausgabedaten ist Zeilenanzahl,
Woerteranzahl, Zeichenanzahl und eventuell Dateiname, `total' bei der
Summe. Bei Eingabe ueber Standard-Eingabe oder nur einer Datei soll
die vierte Information entfallen. 

\subsection*{Optionen}

Keine.

\subsection*{Beispiel}

\begin{verbatim}
  % count datei
       84     462    3621 datei
\end{verbatim}
