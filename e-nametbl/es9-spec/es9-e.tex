%
% specification for cmdline
%

\subsection*{Name}
nametbl -- Functions for a symbol table

\subsection*{Usage}
{\bf nametbl} input-file

\subsection*{Description}
{\bf nametbl}
reads commands from a file and processes them in order to
test a few functions. 
Considered together, the functions implement a symbol table for a
certain computer language.
The only scope for all symbols is ``global.''
No check of type semantics is done.

The symbol table stores the following information for each symbol:

\noindent
Symbol name.

\noindent
Object type of the symbol: {\tt \{OT\_NO\_INF, SYSTEM, RESOURCE\}}.

\noindent
Resource type of the symbol: {\tt \{RT\_NO\_INF, RT\_SYSTEM, FUNCTION, DATA\}}.

\subsection*{Options}
None.

\subsection*{Input Data}

Only the following commands are allowed in the input files.
Each command must start on a new line.
Lower and upper case is relevant in the symbols and in the object and
resource identifiers.
The separator between commands and arguments is a space; this means that
spaces may not appear within a symbol.

\begin{itemize}

\item {\tt ins <Symbol>}

Inserts the symbol in the table.
The object type of the symbol receives the value ``OT\_NO\_INF''
and the resource type receives the value ``RT\_NO\_INF.''
If the symbol already exists, an error message is printed.

\item {\tt tot <Symbol> <ObjTyp>}

Enters the object type <ObjTyp> for the symbol <Symbol>,
where <ObjType> must be either ``SYSTEM'' or ``RESOURCE'';
the value ``OT\_NO\_INF'' is not allowed.
The symbol must be entered in the table, otherwise a corresponding
error message is printed. 
The previous value for the object type is overwritten and lost.

\item {\tt trt <Symbol> <ResTyp>}

Enters the resource type <ResTyp> for the symbol <Symbol>,
where <ObjType> must be either 
``RT\_SYSTEM'', ``FUNCTION'', oder ``DATA'';
the value ``RT\_NO\_INF'' is not allowed.
(The value ``RT\_SYSTEM'' is used in the case where the object type
of the symbol has the value ``SYSTEM''  and information concerning the
resource type is therefore not required.)
The symbol must be entered in the table, otherwise a corresponding
error message is printed. 
The previous value for the resource type is overwritten and lost.


\item {\tt sch <Symbol>}

Searches for the symbol <Symbol> and prints out the corresponding
information. 
If the symbol is not found in the table, a corresponding error message
is printed. 

\item {\tt prt}

Prints out the count of entries in the symbol table and the complete
contents.

\end{itemize}


\subsection*{Example}
{\small
\begin{verbatim}
% cat eingabe
ins qwe
ins rty
tot qwe SYSTEM
prt
% nametbl eingabe
Eingabedatei `eingabe' wird bearbeitet.

Die Zeile `ins qwe' wird ausgewertet:

Die Zeile `ins rty' wird ausgewertet:

Die Zeile `tot qwe SYSTEM' wird ausgewertet:

Die Zeile `prt' wird ausgewertet:
Die Tabelle hat die folgenden 2 Eintraege:
Name    : qwe
oType   : SYSTEM
rType   : RT_NO_INF
-----
Name    : rty
oType   : OT_NO_INF
rType   : RT_NO_INF
-----
Ende der Eingabedatei `eingabe'.
\end{verbatim}
}

\subsection*{Authors}
Goldmann, Klemke, Knecht, Lott.

\subsection*{Bugs}
The functionality for reading files is only for test purposes and is
therefore not especially fault-tolerant.  For example, one can assume
that incorrectly spelled commands, commands with missing arguments, or
commands with too many arguments will not be handled especially
carefully. 

