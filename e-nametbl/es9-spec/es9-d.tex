
\subsection*{Name}

nametbl -- Funktionen fuer eine Symboltabelle 

\subsection*{Verwendung}

{\bf nametbl} Eingabedatei

\subsection*{Beschreibung}

{\it nametbl} liesst Kommandos aus der Eingabedatei und bearbeitet
sie, um einige Funktionen auszutesten.
Die Funktionen implementieren gemeinsam eine einfache Symboltabelle
fuer eine (gewisse) Computersprache. 
Es gibt fuer Symbole nur den Gueltigkeitsbereich "global". 
Es findet keine Pruefung von Semantik statt.

Die Symboltabelle speichert folgende Informationen zu jedem Symbol:\\
Symbolname,\\
Objekttyp des Symbols: \{OT\_NO\_INF, SYSTEM, RESOURCE\},\\
Resourcetyp des Symbols: \{RT\_NO\_INF, RT\_SYSTEM, FUNCTION, DATA\}.

\subsection*{Optionen}
Keine.

\subsection*{Eingabe}

Nur die folgenden Kommandos sind in der Eingabedatei zulaessig.
Jedes Kommando muss in einer neuen Zeile anfangen.
Klein- und Grossschreibung wird bei den Symbolen und den 
Objekt- und Resourcetypbezeichnern unterschieden.
Der Trenner zwischen Kommandos und Argumenten ist immer ein
Leerzeichen; d.h., Leerzeichen duerfen nicht innerhalb eines
Bezeichners vorkommen.

\begin{itemize}

\item {\tt ins <Symbol>}

F\"ugt das Symbol in die Tabelle ein.
Der Objekttyp des Symbols bekommt den Wert "OT\_NO\_INF" und der
Resourcetyp den Wert "RT\_NO\_INF".
Falls das Symbol schon in der Tabelle existiert, wird eine
Fehlermeldung ausgegeben. 

\item {\tt tot <Symbol> <ObjTyp>}

Traegt zum Symbol <Symbol> den Objekttyp <ObjTyp> ein,
wobei <ObjTyp> entweder "SYSTEM" oder "RESOURCE" sein muss; 
der Wert "OT\_NO\_INF" ist nicht zulaessig.
Das Symbol muss in der Tabelle eingetragen sein, ansonsten wird eine
entsprechende Fehlermeldung ausgegeben.
Der vorherige Objekttyp wird ueberschrieben und geht somit verloren.

\item {\tt trt <Symbol> <ResTyp>}

Traegt zum Symbol <Symbol> den Resourcetyp <ResTyp> ein,
wobei <ResTyp> entweder "RT\_SYSTEM", "FUNCTION" oder "DATA" sein
muss;
der Wert "RT\_NO\_INF" ist nicht zulaessig.
(Der Wert "RT\_SYSTEM" ist fuer den Fall vorgesehen, dass 
der Objekttyp des Symbols den Wert "SYSTEM" hat und Information
bezueglich des Resourcetyps deswegen nicht noetig ist.)
Das Symbol muss in der Tabelle eingetragen sein, ansonsten wird eine
entsprechende Fehlermeldung ausgegeben.
Der vorherige Resourcetyp wird ueberschrieben und geht somit verloren.

\item {\tt sch <Symbol>}

Sucht nach dem Symbol <Symbol> und druckt die dazugehoerigen
Informationen aus.
Falls das Symbol nicht in der Tabelle gefunden wird, wird eine
entprechende Fehlermeldung ausgegeben.

\item {\tt prt}

Druckt die Anzahl der Eintraege in der Symboltabelle und den kompletten
Inhalt aus.

\end{itemize}

\subsection*{Beispiel}
\begin{verbatim}
% cat eingabe
ins qwe
ins rty
tot qwe SYSTEM
prt
% nametbl eingabe
Eingabedatei `eingabe' wird bearbeitet.

Die Zeile `ins qwe' wird ausgewertet:

Die Zeile `ins rty' wird ausgewertet:

Die Zeile `tot qwe SYSTEM' wird ausgewertet:

Die Zeile `prt' wird ausgewertet:
Die Tabelle hat die folgenden 2 Eintraege:
Name    : qwe
oType   : SYSTEM
rType   : RT_NO_INF
-----
Name    : rty
oType   : OT_NO_INF
rType   : RT_NO_INF
-----
Ende der Eingabedatei `eingabe'.
\end{verbatim}

\subsection*{Authors}
Goldmann, Klemke, Knecht, Lott.

\subsection*{Bugs}

Die Funktionalitaet zum Einlesen von Dateien und Erkennen von
Kommandos ist nur zu Testzwecken gedacht und ist deswegen nicht
besonders fehlertolerant.  Zum Beispiel kann man davonausgehen, 
dass fehlende Argumente, zu viele Argumente oder falsch buchstabierte
Argumente bei einem Kommando nicht besonders sorgfaeltig behandelt
werden. 
