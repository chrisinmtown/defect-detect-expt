Beachten Sie bitte auch das Erg"anzungsblatt "`Code-reading mit Programm $x$"',
was im Ihnen vorliegenden Material enthalten ist.

\begin{enumerate}

\item Lesen Sie sich den Fragebogen "`E12: Code Reading"' durch.
Tragen Sie Ihre Kennung ein und f"ullen Sie die Rubrik
"`Vorbetrachtungen"' aus. 

\subsection*{Kode lesen}

\item Tragen Sie die Uhrzeit zu der Sie mit dem Experiment beginnen 
unter Frage 4 ein.

\item Lesen Sie den Kode der Komponente durch, um einen "Uberblick zu
gewinnen. Falls Ihnen jetzt oder in
den folgenden drei Schritten bereits m"ogliche Fehler auffallen, dann 
markieren Sie diese. Investieren Sie bitte keine Zeit in die
genaue Analyse, isolieren Sie keine Fehler !!! 
Fehlerisolation erfolgt ab Schritt 11.

\item Bestimmen Sie die Abh"angigkeiten zwischen den einzelnen Funktionen im
Quellkode (z.B. mit Hilfe eines Aufruf-Baumes). Normalerweise sind die
Funktionen im Quellkode bereits so angeordnet, da"s die "`low-level"'
Funktionen (Bl"atter) vorne stehen und die "`high-level"' Funktionen (Wurzel)
hinten. Beginnen Sie mit dem code-reading bei den Blatt-Funktionen und 
arbeiten sich dann bis zur Wurzel vor.

\item  Machen Sie sich mit der Struktur jeder einzelnen Funktion vertraut, 
indem Sie
die Elementarstrukturen (Sequenzen, bedingte Anweisungen, Schleifen) 
identifizieren und z.B. durch
Umrahmung kenntlich machen. Fassen Sie die Elementarstrukturen zu
gr"o"seren Strukturen zusammen, bis schlie"slich ein Rahmen die gesamte 
Funktion umfa"st.

\item	Versuchen Sie von innen heraus die Bedeutung eines jeden
Rahmens zu bestimmen und notieren Sie diese auf dem daf"ur vorgesehenen Blatt.
Verwenden Sie dabei die vorgegebenen Zeilennummern (von--bis Zeile). 
Vermeiden Sie implizit
Wissen zu verwenden, das nicht innerhalb dieses Rahmens liegt (z.B.
Werte bei der Initialisierung, Eingaben oder Parameterwerte).
Beschreiben Sie die Teile der Komponente so funktional wie m\"oglich.
Benutzen Sie dazu allgemein bekannte Prinzipien des Anwendungsbereichs, um
die Beschreibung kurz und verst"andlich zu halten, z.B.: bei B"aumen das
Prinzip "`Breitensuche"', anstatt die Breitensuche wortreich zu beschreiben.

\item Tragen Sie den Zeitbedarf f"ur die Erstellung der Abstraktionen
im Fragebogen unter Frage 5 ein.  Tragen Sie auch bitte gleich die
maximale Anzahl der Abstraktionsebenen und die Art der Beschreibung unter
Fragen 6 bzw.~7 ein.   

\item	Machen Sie eine Kopie Ihrer Notizen (das Original ben"otigen Sie noch)
und vermerken Sie bitte Ihre Praktikumskennung auf der Kopie.
Geben Sie die Kopie beim Betreuer ab. 
Sie erhalten dann die Spezifikation der Komponente.

\subsection*{Inkonsistenzen diagnostizieren}

\item Beantworten Sie Frage 8 auf dem Fragebogen.
Lesen Sie die Spezifikation aufmerksam durch. Die Spezifikation beschreibt die
Komponente als Ganzes, nicht die einzelnen Funktionen. Sie m"ussen daraus die 
Spezifikationen der einzelnen Funktionen ableiten.
(Dies ist notwendig, da bei allen Testtechniken aus Gr"unden der 
Vergleichbarkeit mit der gleichen Spezifikation 
gearbeitet werden soll). 
Finden Sie etwaige Inkonsistenzen, indem Sie die Spezifikation mit
Ihrer Beschreibung der Komponente vergleichen. 
Tragen Sie die entdeckten Inkonsistenzen in das Formblatt ein,
vergessen Sie nicht, dort auch Ihre Praktikums-Kennung anzugeben.

\item	Wenn Sie meinen, alle Inkonsistenzen entdeckt zu haben, dann
	tragen Sie den Zeitbedarf f"ur die Suche 
	im Fragebogen unter Frage 9 ein. Nummerieren Sie bitte Ihre Inkonsistenzen
	von 1 bis $n$ in der Spalte "`I-Nr."' (Inkonsistenznummer) des Formblatts.

%Eine Musterl\"osung ist ebenfalls bei Ihrem Betreuer erh\"altlich.
%Benutzen Sie diese jedoch so sp\"at wie m\"oglich. 

\subsection*{Fehler lokalisieren}

\item	Versuchen Sie die Fehler, die zu den von Ihnen erkannten
Inkonsistenzen gef\"uhrt haben, zu isolieren. Beantworten Sie bitte
zuvor Frage 10 auf dem Fragebogen.  Tragen Sie die isolierten
Fehler in das Formblatt ein, vergessen Sie nicht, dort auch Ihre
Praktikums-Kennung anzugeben. 

Geben Sie an welche Fehler Sie entdeckt haben
[Zeilennummer]. Klassifizieren Sie dabei nach den vorgestellten
Fehlertypen [{Omission, Comission}, {Initialization, Control,
Interface, Data, Computation, Cosmetic}]. Charakterisieren Sie dann 
kurz den Fehler, z.B. `Initialwert 1 statt 0'. Geben Sie auch die 
Nummer der Inkonsistenz an, zu der der Fehler f"uhrt.

Tragen Sie auf die beschriebene Weise auch die Fehler ein, die Sie zuf"allig 
gefunden haben. In die Spalte "`I-Nr"' geh"ort in diesem Fall ein Strich.

\item Tragen Sie den Zeitbedarf f"ur die Fehlerisolation im Fragebogen
unter Frage 11 ein.

\item F"ullen Sie bitte den Rest des Fragebogens (Fragen 12--14). 
Geben Sie den Fragebogen und alle Formbl"atter bei Ihrem Betreuer ab,
sch"onen Feierabend.

\end{enumerate}

