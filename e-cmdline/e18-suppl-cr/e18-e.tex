
This is the supplemental sheet 
for applying technique ``code reading''
to the program ``cmdline.''

\subsection*{Brief description of the library functions used}

\begin{itemize}

\item {\tt int strlen(char *s)}

Returns the length of the string {\tt s} (the null character on the end
is not counted).

\item {\tt int strcmp(char *s1, char *s2)}

Compares two strings.  Returns a value greater than, equal to, or less
than 0 depending on whether {\tt st1} is lexicographically (i.e.,
ASCII value) greater than, equal, or less than {\tt s2}.

\item {\tt int strncmp(char *s1, char *s2, int n)}

Like {\tt strcmp}, but compares maximum {\tt n} characters.

\item {\tt int fprintf(stderr, "Text"...)}

Prints the text on the standard-error output stream.

\item {\tt int isdigit(char c)}
Returns {\em true} if the character is a digit, otherwise {\em false}.

\item {\tt int atoi(char *str)}

Converts the string {\tt str} to an integer.
Leading white space is ignored, and scanning of the string ends
with the first non-numeric character.
If no numeric characters could be read, 0 is returned.

\end{itemize}

\subsection*{Reminder}

Don't produce any abstractions for the test-scaffold functions.
