%
% sample solution for training exercise series
%

The following tables show example equivalence classes and test cases
based on the equivalence classes for program ``series.'' These tables
may be useful as example solutions. 


\begin{table}[h]
\begin{center}
\begin{tabular}{|r|c|l|}
\hline
Nr. & V/I$^*$ & Equivalence class \\
\hline
\hline
  1 &   V & Argument count is 2   \\
\hline
  2 &   V & Argument count is 3   \\
\hline
  3 &   I & Argument count is neither 2 nor 3   \\
\hline
\hline
  4 &   I & First argument is not a number \\
\hline
  5 &   I & Second argument is not a number \\
\hline
  6 &   I & Third argument is not a number \\
\hline
  7 &   V & All arguments are numbers \\
\hline
\hline
  8 &   V & Start value is larger than end value \\
\hline
  9 &   V & Start value is smaller than end value \\
\hline
 10 &   V & Start value equals end value \\
\hline
\hline
 11 &   I & Increment is zero \\
\hline
 12 &   V & Increment is larger than abs(end -- start) \\
\hline
 13 &   V & Increment is smaller than abs(end--start) \\
\hline
\hline
 14 &   V & All arguments are integer \\
\hline
 15 &   V & At least one argument is a real\\
\hline
\multicolumn{3}{l}{$^*$V = equivalence class for valid input, 
I = e. c. for invalid input}
\end{tabular}
\end{center}
\caption{Equivalence classes for program ``series''}
\label{T:equiv-classes-series}
\end{table}

\begin{table}[h]
\begin{center}
\begin{tabular}{|r|l|l|l|}
\hline
Nr. & Test case    & Equiv. class    & Expected output \\
\hline
\hline
1 & series 1       & 3               & error message \\
\hline
2 & series x 2     & 1, 4            & error message  \\
\hline
3 & series 1 x     & 1, 5            & error message  \\
\hline
4 & series 1 2 x   & 2, 6	     & error message  \\
\hline
5 & series 10 0    & 1, 7, 8, 14     & 10, 9, 8, ..., 1, 0\\
\hline
6 & series 1 2 0.1 & 2, 7, 9, 13, 15 & 1.0, 1.1, 1.2, ..., 1.9, 2.0\\
\hline
7 & series 7 7     & 1, 7, 10, 14    & the number '7' \\
\hline
8 & series 1 2 0   & 2, 7, 9, 11, 14 & error message \\
\hline
9 & series 2 1 0.1 & 2, 7, 8, 13, 15 & 2.0, 1.9, 1.8, ..., 1.1, 1.0\\
\hline
10 & series 1 2 3  & 2, 7, 9, 12, 14 & the number '1' \\
\hline
\end{tabular}
\end{center}
\caption{Tests based on equivalence classes for program ``series''}
\label{T:tests-equiv-classes-series}
\end{table}
