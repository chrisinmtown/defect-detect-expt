%
% sample solution for training exercise ``count''
%

The following lists line numbers and abstractions (specifications) of
the source code from those lines.

\begin{tabbing}
mmmmmmmm   \= mmm \= \kill
{\bf Line(s)} \> {\bf Abstraction}							\\
\rule{\textwidth}{.5pt}\\ 
21--29	\> Sequence									\\
											\\
	\>The meaning of the components:						\\
											\\
21	\> charct := charct + 1								\\
22--23  \> c = {\em newline} $\rightarrow$ linect := linect + 1 $\mid$ ()  		\\
24--29	\> c = {\em whitespace} $\rightarrow$ inword:= 0 $\mid$ 				\\
	\> \>	(inword = 0 $\rightarrow$ inword, wordct := 1, wordct + 1 $\mid$ () ).	\\
											\\
	\> Because there are 4 paths through this sequence, the following results:	\\
											\\
	\> c = {\em newline} $\rightarrow$ charct, linect := charct + 1, linect + 1 $\mid$	\\
	\> c $\not=$ {\em newline} $\wedge$ c = whitespace $\rightarrow$ charct, inword := charct + 1, 0 $\mid$\\
	\> c $\not=$ {\em newline} $\wedge$ c $\not=$ whitespace $\wedge$ inword = 0 		\\
	\> \>	$\rightarrow$ charct, wordct, inword := charct + 1, wordct + 1, 1 $\mid$\\
	\> c $\not=$ {\em newline} $\wedge$ c $\not=$ whitespace $\wedge$ inword $\not=$ 0 $\rightarrow$ charct := charct + 1.\\
\\
14--39	\> Sequence\\
\\
	Once again, first the meaning of the components:\\
\\
14--17	\>argc > 1 $\rightarrow$ (file-open(argv[i]) = {\em failure} $\rightarrow$ {\em err-msg and halt} $\mid$ fp := {\em stream}) $\mid$ ().\\
\\
18--30	\> To determine the meaning of this portion sensibly, the lines 18--19 are \\
	\> included and the tasks of the variables are investigated:\\
	\> 1. Variable ``charct'' is incremented in all 4 cases; i.e., counts every character\\
	\> 2. Variable ``linect'' is only incremented if a {\em newline} is read; i.e., counts lines\\
	\> 3. Variable ``inword'' is a switch that takes on values 0 and 1.\\
	\> If whitespace is seen, the switch is set to 0. If other characters are seen,\\
	\> it is switched to 1 and at the same time ``wordct'' is incremented; i.e., counts words.\\
\\
	\> c $\not=$ EOF $\rightarrow$ 	charct, wordct, linect :=	\\
	\> \> character-count(stdin), word-count(stdin), line-count(stdin).\\
\\
	\> Note: because ``inword'' is initialized to 1, the first word is not counted\\
	\> if it comes at the beginning of a stream not preceded by whitespace.\\
\\
31	\> stdout := ``linect, wordct, charct''\\
32--35	\> argc > 1 $\rightarrow$ stdout := *argv {\em (i.e., program name) and newline} $\mid$ stdout := {\em newline}\\
36	\> {\em close stream}\\
37--39	\> tlinect, twordct, tcharct := tlinect + linect, twordct + wordct, tcharct + charct\\
\\
\\
\\
\\
	\> Because there are 3 paths through this sequence, the following results:\\
\\
	\> argc > 1 $\wedge$ open-file(argv[i]) = failure $\rightarrow$ stdout: = err msg {\em and halt} $\mid$\\
	\> argc > 1 $\vee$   open-file(argv[i]) = success $\rightarrow$\\
	\> \>	tcharct, tlinect, twordct, stdout := tcharct + character-count({\em stream}),\\
	\> \>	twortct + word-count({\em stream}), tlinect + line-count({\em stream}),\\
	\> \>   ``line count({\em stream}), word-count({\em stream}), character-count({\em stream}), pgm-name''\\
	\> argc $\leq$ 1 $\rightarrow$\\
	\> \>	tcharct, tlinect, twordct, stdout := tcharct + character-count(<nil>),\\
	\> \>	twortct + word-count(<nil>), tlinect + character-count(<nil>), \\
	\> \>	``line count(<nil>), word-count(<nil>), character-count(<nil>)''\\
\\
	\> Note: If argc is $\leq$ 1, tp is not initialized; i.e., the program\\
	\> reads from an undefined stream (label <nil> above).\\
\\
3--44	\> Sequence\\
\\
	Once again, first the meaning of the components:\\
\\
10	\> tlinect, twordct, tcharct := 1, 1, 1\\
12--40	\> for all indexes of command-line arguments from 1{\ldots}argc - 1 do [14--39]\\
41--42	\> argc > 1 $\rightarrow$ stdout := ``linect, twordct, tcharct'' $\mid$ ()\\
43	\> halt\\
\rule{\textwidth}{.5pt}
\end{tabbing}

\noindent
The behavior of the entire program is captured as follows:

\begin{itemize}

\item All command-line arguments are treated as file names.
Thee cases are distinguished:

\begin{enumerate}
\item Arguments are available, but no files exist with names 
corresponding to the arguments.
In this case, the program halts with an error message.

\item Arguments are available, and they correspond to existing files.
For each file, the number of lines, words, and characters
is counted and the sum along with the program name are printed
(the file name was expected).

\item  No arguments are available.  In this case, the program
attempts to read from an uninitialized stream and proceeds as in (2),
although no program name is printed (a reasonable initialization
of the stream would have been stdin).

\end{enumerate}

\item The count of words is determined depending on the first character 
in the stream.
If the first character is not a white space, then the first word is
not counted.

\item If the total count of arguments is at least 1, the total sum of 
all characters and words in all files and the count of lines in the last
file is printed.


\end{itemize}
